\documentclass [a4paper, 12pt]{article}

\usepackage{fullpage}
\usepackage[utf8]{inputenc}
\usepackage{polski}
\linespread{1.3}
\author{Stanisław Chmiela}
\title{Konjunktiv}

\begin{document}
	\section{Konjunktiv Imperfekt}
		Formy tworzone z Imperfektu.
		\begin{quote}
		\subsection{Czasowniki słabe}
			\begin{enumerate}
			\item \textbf{suchen} -- suchte
			\item \textbf{machen} -- machte
			\end{enumerate}
		\subsection{Czasowniki mocne}
			Niektóre samogłoski dostają przegłos.
			\begin{enumerate}
			\item \textbf{fahren} -- f\"uhr
			\item \textbf{kommen} -- k\"am
			\end{enumerate}
		\subsection{Czasowniki modalne}
			Przegłos.
			\begin{enumerate}
			\item \textbf{d\"urfen} -- ich d\"urfte
			\item \textbf{m\"ogen} -- ich m\"ochte
			\end{enumerate}
	\end{quote}
	\section{Konjuktiv Plusquamperfekt}
		Czas przeszły. Tworzony z pomocą czasownika posiłkowego (Konjunktiv Imperfekt) + Partizip Perfekt.
		\begin{enumerate}
		\item \textbf{haben} -- ich h\"atte gehabt
		\item \textbf{sein} -- ich w\"are gewesen
		\item \textbf{suchen} -- ich h\"atte gesucht
		\item \textbf{kommen} -- ich w\"are gekommen
		\end{enumerate}
	\section{Der Konditional}
		Bywa tak, że czasem używa się tego, \emph{w\"urde + Infinitiv}.
		\begin{enumerate}
		\item \textbf{ich machte} -- ich w\"urde machen
		\item \textbf{ich verst\"unde} -- ich w\"urde verstehen
		\item \textbf{er nennte} -- er w\"urde nennen
		\item \textbf{ich kaufte} -- ich w\"urde kaufen
		\end{enumerate}
\end{document}
