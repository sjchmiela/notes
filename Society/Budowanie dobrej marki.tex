\documentclass [a4paper, 11pt]{article}
\usepackage[utf8]{inputenc}
\usepackage{polski}
\usepackage{fullpage}
\usepackage{hyperref}
\usepackage[usenames,dvipsnames]{color}
\hypersetup{
    bookmarks=true,         % show bookmarks bar?
    unicode=false,          % non-Latin characters in Acrobat’s bookmarks
    pdftoolbar=true,        % show Acrobat’s toolbar?
    pdfmenubar=true,        % show Acrobat’s menu?
    pdffitwindow=false,     % window fit to page when opened
    pdfstartview={FitH},    % fits the width of the page to the window
    pdftitle={Budowanie dobrej marki},    % title
    pdfauthor={},     % author
    pdfsubject={Budowanie marki},   % subject of the document
    pdfcreator={Stanisław Chmiela},   % creator of the document
    pdfproducer={Stanisław Chmiela}, % producer of the document
    pdfkeywords={marka} {wos} {psychologia} {praca}, % list of keywords
    pdfnewwindow=false,      % links in new window
    colorlinks=true,       % false: boxed links; true: colored links
    linkcolor=Black,          % color of internal links
    citecolor=PineGreen,        % color of links to bibliography
    filecolor=RawSienna,      % color of file links
    urlcolor=MidnightBlue      % color of external links
}
\usepackage[pdftex]{graphicx}
\usepackage{wrapfig}
\usepackage{float}
\linespread{1.15}
\author{Stanisław Chmiela}
\title{Budowanie dobrej marki}
\begin{document}
\maketitle
\tableofcontents
\clearpage
\section{Wstęp}
Budowanie dobrej marki jest tematem bardzo złożonym. Nie da się zmieścić wszystkich rad oraz wiedzy na czterech stronach papieru, jednak ta praca podejmuje ten temat w najlepszy możliwy sposób. Mimo tego, że to tylko wierzchołek góry lodowej, opowiada o tradycyjnym budowaniu ,,\emph{brandu}'' (ang.) oraz najnowszych trendach.

\section{Co to jest marka?}
Słowo ,,\emph{marka}'' jest powszechne w naszym języku. Ale często znamy znaczenie danego słowa tylko intuicyjnie. Dlatego najpierw pochylmy się nad tym, czym w istocie marka jest. W marketingu istnieje wiele definicji tego terminu.

Jedną z nich jest definicja Jacka Kalla\cite{kall}: marka to ,,spełniona obietnica'' oraz ,,reduktor ryzyka''. Firmy obiecują wiele rzeczy w reklamach promujących marki, np. natychmiastowe uśmierzenie bólu głowy, jedwabiste włosy, superniskie spalanie samochodu itp. Jeśli wierzymy w te obietnice w większości dlatego, że dani producenci wywiązali się z innych obietnic w przeszłości. Kupiwszy toyotę, jeśli przez parę lat używania jej nie będziemy mieć żadnych problemów -- następnym samochodem prawdopodobnie będzie znów toyota. Ponieważ producent obiecał niezawodność, a produkt wywiązał się z obowiązków. Tego samego spodziewamy się po następnym produkcie marki, która spełniła obietnicę. A po kolejnym szczęśliwym zakupie będziemy mieli wyrobioną mocną opinię na temat samochodów tej marki -- wiemy, że są one niezawodne i że jest to powtarzalne. Kupując trzecią toyotę klient bierze pod uwagę mniejsze ryzyko -- skoro dwa poprzednie samochody okazały się fenomenalne, tego samego spodziewa się po kolejnym modelu.

Inną definicją marki jest ta zawarta w książce ,,Building Strong Brands''\cite[s.~68]{aaker} Davida Aakera. Według niego na określenie terminu marki składają się cztery składniki:
\begin{description}
	\item[Produkt albo usługa] --- Producent tworzy markę produktu lub usługi, która staje się mocniejsza i bardziej znana od marki wytwórcy.\\
	Przykład: iPod $>$ Apple, Plus GSM $>$ Polkomtel
	\item[Osoba] --- Dzieje się tak, gdy z danym produktem, usługą lub producentem natychmiast kojarzy nam się jakaś osobistość, która jest twarzą, a czasem nawet i duszą danej rzeczy.\\
	Przykład: Steve Jobs -- Apple, Walt Disney -- Disney
	\item[Organizacja] --- W tym przypadku marką jest zwykle sam producent-firma.\\
	Przykład: Fiat, Philips
	\item[Symbol] --- Silna marka może próbować stać się symbolem czasów, stylu życia, czy danej gałęzi nauk.\\
	Przykład: iPod -- wolność, IBM -- informatyka.
\end{description}

\subsection{Czym się różni w takim razie dobra marka od złej marki?}
Dobra marka to marka, którą ludzie zauważają i pamiętają. \cite{pogorzelski} Dla przykładu porównajmy dwie marki: FRUGO oraz Daewoo. Pierwsza prezentuje serię napojów, druga prezentowała samochody. Co się stało, gdy z polskich sklepów zniknęły napoje FRUGO? Na portalach społecznościowych zaczęły się protesty, wydarzenia, petycje o przywrócenie produktów. Czy gdy firma Daewoo zniknęła z rynku, czy ktoś protestował? Czy ktoś to w ogóle zauważył? Marka FRUGO jest przykładem marki, która zapadła w pamięć konsumentów, a Daewoo zniknęło zarówno z rynku jak i świadomości klientów.

\section{Jak budować markę?}
\subsection{Historyczne budowanie marek}
W obecnych czasach dobrze jest znać historię budowania marki W 2005 roku Steve Jobs na mowie rozpoczynającej rok szkolny opowiedział historię o ,,łączeniu kropek''.
\begin{quote}
Nie można łączyć kropek patrząc w przód, tylko patrząc w tył. \begin{flushright}\cite{jobs}\end{flushright}
\end{quote}

A więc jak budowano marki w dawnych czasach? Opowieściami. \cite{wyklad2} Gdy na bazar przychodził klient i chciał kupić kokosa -- sprzedawca zaczynał opowiadać mu historię, jak wysokie są palmy kokosowe, w jak pięknym środowisku wyrastają, jak ciężko jest zdjąć kokosa, jak daleko to jest\dots Przez to konsument czuł się wyjątkowy -- no w końcu taki kokos z taką historią -- wyjątkowy! Mimo ponad dwutysięcznego postępu opowiedzenie historii nadal jest podstawą marketingu.

\subsection{Przygotowanie i długotrwałe planowanie rozwoju}
Zanim zaczniemy budować markę, musimy jasno określić, czym ma być nasza marka, a czym ma na pewno nie być. \cite[s.~18]{tkaczykzakamarki} Jeśli zamierzamy stać się w oczach klientów specjalistą od napraw kosiarek, możemy w dalszej perspektywie zacząć je sprzedawać (domena działalności pozostaje mniej więcej taka sama), ale nie możemy pozwolić sobie na sprzedaż np. oczek wodnych. Wtedy cały wizerunek runie, a autorytet zniknie. \cite{wyklad2}
\subsection{Przydatność i użyteczność}
Gdy już jesteśmy gotowi ze swoim produktem i strategią marki, powinniśmy zająć się przyciągnięciem uwagi konsumentów. \cite{kall, tkaczykzakamarki} Na tym etapie musimy przekonać potencjalnych klientów, że nasz produkt jest przydatny oraz dostępny. W czasie opracowywania strategii marki musimy mieć świadomość, że nie da się stworzyć mocnej marki dla beznadziejnego produktu. Musimy udowodnić komuś kto dotychczas obchodził się bez naszego rewolucyjnego produktu, że nie może się więcej bez niego obejść. \cite[s.~109]{pogorzelski} Produkty są zwykle odbierane na dwóch poziomach w naszych mózgach.
\subsubsection{Poziom umysłu}
Poziom umysłu dotyka praktycznych, technicznych wręcz cech produktu. Za postrzeganie ich odpowiada rozum, ośrodek racjonalnego myślenia. Dzięki niemu wiemy na przykład, że to pióro jest wygodne, zapas atramentu starcza na długo, dobrze się je trzyma, ma trwałą stalówkę, nigdy nie cieknie itd.
\subsubsection{Poziom serca}
,,Serce'' natomiast podpowiada nam, że to pióro ładnie wygląda, że podnosi mój status społeczny, pisząc takim piórem czuję się zobowiązany do pisania rzeczy ważnych i mądrych.

Sprytnie oba te poziomy wykorzystuje TPSA w swoim cyklu reklam ,,Serce i Rozum''. Obaj bohaterowie prezentują humorystycznie różne argumenty, trafiające do obu naszych sfer decyzyjnych.

\subsection{Dotrzymywanie obietnic}
Promując zalety i korzyści produktu musimy zadbać, aby klient faktycznie miał poczucie spełnienia promowanych obietnic. Musimy pamiętać, że w pamięci klienta dużo mocniej zarysowują się złe wspomnienia, zwłaszcza na nowych markach. Wielokrotne badania pokazały, że klient pamięta średnią z ostatniego i najgorszego wrażenia w czasie korzystania z produktu bądź usługi. \cite{wyklad2}

\subsection{Umocnienie pozycji na rynku}
Budując długoterminową strategię musimy od samego początku wiedzieć, czy chcemy być marką-ekspertem, czy marką-parasolem. Czasem ze względu na produkt lub usługę wybór jest prosty, czasem jednak wymaga mocnego namysłu i dokładnego planu.

Marka-ekspert zyskuje coś niepowtarzalnego, status elity, specjalisty. Jeśli będzie dość konsekwentna, zyska monopol w danej branży. Udało się to na przykład markom Adidas i Pampers. Dzięki mocno ograniczonemu wachlarzowi produktów i ograniczeniu się do wysokiej jakości oraz specjalistycznej branży wszystkie sportowe buty stały się adidasami, a wszystkie pieluszki dla dzieci -- pampersami.

Marka-parasol staje się natomiast miejscem, do którego idziemy, gdy chcemy kupić np. cokolwiek związanego z elektroniką (Media Markt, Saturn), cokolwiek do wyposażenia domu (IKEA), książkę, płytę z muzyką (empik). Dzięki tej postawie specjalizacja znika, pojawia się natomiast marka parasolowata.

\section{Co dalej? -- \emph{social media}}
Tradycyjny marketing zamyka się na pewnej ograniczonej części wiedzy, jednak rozwój świata niesie ze sobą zmiany dotyczące budowania marek. Paręnaście lat temu nawet ,,marketingowi magicy'' nie wyobrażali sobie jak potoczą się losy komunikacji społecznej.

Do niedawna byliśmy zdani tylko na bierną komunikację z markami. Oglądaliśmy reklamy w gazetach, później przyszło radio. Wynaleziono telewizję, nadal klient nie miał możliwości odpowiedzi na reklamę. Dopiero gdy Internet dojrzał, pojawiła się możliwość komunikacji marki i konsumenta.

Teraz człowiek w galerii handlowej może w kilkanaście sekund sprawdzić jak inni użytkownicy oceniają daną restaurację, gdzie wybrać się, jeśli jest wielbicielem włoskiej kuchni, czy fryzjer na ulicy Słonecznej dobrze obcina itp. Pod koniec marca 2012 z serwisu Facebook korzystało 901 milionów użytkowników \cite{facebook}. Jest to potęga! W pewnych branżach marki niedostatecznie wykorzystują potencjału portali społecznościowych, które dają ogromne możliwości promocji marki. Dzięki nim mogą pojawić się za darmo w życiu konsumentów, reklamować się darmowymi pozytywnymi opiniami klientów, którzy chcą (!) dzielić się z innymi swoimi wrażeniami. Powinno to być wykorzystane.

\section{Najnowszy trend -- grywalizacja}
Do ludzkiego życia bardzo często wkrada się rutyna, a nawet nuda. Konsumenci szukają przez to ciekawych doznań, frajdy z zakupów. Grywalizacja, ogłoszona przez ,,\emph{Harvard Business Review}'' za jeden z wiodących trendów marketingowych drugiej dekady XXI wieku jest właśnie spełnieniem tego życzenia.\cite{tkaczykgrywalizacja}
\subsection{Jak dać klientowi radość z zakupów?}
Odejść od rutyny jest bardzo łatwo -- wystarczy wprowadzić element losowy. Weźmy choćby zupełnie zwykłą promocję w pubie -- ,,Kup jedno piwo, drugie gratis''. \cite{tkaczykwideo} Wyjątkowo pusta, zwykła promocja. Nie wzbudza żadnych emocji.

Ale ile się zmienia, gdyby wprowadzić element losowy. Zamiast prostych warunków -- losowanie. Gdy klient zamawia piwo, kelner proponuje mu rzut monetą. Jeśli wygra kelner, klient płaci za napój. Jeśli klient, dostaje piwo za darmo. Zwykła promocja nagle staje się czymś przyciągającym rzesze ,,szczęściarzy''. Koszt obu promocji jest bardzo podobny.

Kawiarnie mają podobny problem -- zwykłe menu, niewiele się zmienia, codziennie Ci sami kelnerzy, taka sama kawa, taka sama kolejka. Dodajmy do tego grywalizację, nagradzajmy lojalnych klientów. Po pewnej ilości zakupów nadajmy tytuł ,,ambasadora restauracji'' i stwórzmy osobną kolejkę dla takich ambasadorów. Każdy chciałby zostać takim specjalnym gościem, jak najszybciej wyrobić potrzebną ilość kaw, aby z uśmiechem na twarzy przejść ze swoją kawą rano przed kolejką zwykłych ludzi. Różne osiągnięcia dla różnych typów graczy tak, aby każdy znalazł coś dla siebie. Przyniosłoby to kawiarni wiele nowych i stałych klientów.

\section{Zakończenie}
Budowanie dobrej marki jest moim zdaniem bardzo ciekawym zagadnieniem, wymagającym umiejętności, starannego planowania, analiz oraz badań. Jest procesem skomplikowanym, długotrwałem oraz wymaga interdyscyplinarnej wiedzy. W życiu każdego przedsiębiorstwa decyzja o promowaniu marki jest bardzo ważna, może doprowadzić do złotego okresu w historii firmy lub do upadku.
\clearpage
\bibliographystyle{apasoft}
\bibliography{bibliografia}
\end{document}