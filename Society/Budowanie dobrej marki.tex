\documentclass [a4paper, 11pt]{article}
\usepackage[utf8]{inputenc}
\usepackage{polski}
%\usepackage{fullpage}
\usepackage{hyperref}
\usepackage[usenames,dvipsnames]{color}
\hypersetup{
    bookmarks=true,         % show bookmarks bar?
    unicode=false,          % non-Latin characters in Acrobat’s bookmarks
    pdftoolbar=true,        % show Acrobat’s toolbar?
    pdfmenubar=true,        % show Acrobat’s menu?
    pdffitwindow=false,     % window fit to page when opened
    pdfstartview={FitH},    % fits the width of the page to the window
    pdftitle={Budowanie dobrej marki},    % title
    pdfauthor={},     % author
    pdfsubject={Budowanie marki},   % subject of the document
    pdfcreator={Stanisław Chmiela},   % creator of the document
    pdfproducer={Stanisław Chmiela}, % producer of the document
    pdfkeywords={marka} {wos} {psychologia} {praca}, % list of keywords
    pdfnewwindow=false,      % links in new window
    colorlinks=true,       % false: boxed links; true: colored links
    linkcolor=Black,          % color of internal links
    citecolor=PineGreen,        % color of links to bibliography
    filecolor=RawSienna,      % color of file links
    urlcolor=MidnightBlue      % color of external links
}
\usepackage[pdftex]{graphicx}
\usepackage{wrapfig}
\usepackage{float}
\linespread{1.5}
\author{Stanisław Chmiela}
\title{Budowanie dobrej marki}
\begin{document}
\maketitle
\tableofcontents
\clearpage
\section{Wstęp}
Ta praca opowiada o tym, jak zbudować dobrą markę produktu/usługi. Zawiera zasady, wskazówki, opisuje proces budowania marki, nowopowstający ,,social media marketing''. Opowiada również o najnowszym trendzie w budowaniu marek i utrzymywaniu klientów przy danej firmie -- grywalizacji.

\section{Co to jest marka?}
Słowo ,,\emph{marka}'' jest powszechne w naszym języku. Ale często znamy znaczenie danego słowa tylko intuicyjnie. Dlatego najpierw pochylmy się nad tym, czym w istocie marka jest, a czym nie jest.

Według definicji Jacka Kalla\cite{kall} marka to ,,spełniona obietnica'' oraz ,,reduktor ryzyka''. Firmy obiecują wiele rzeczy w reklamach, natychmiastowe uśmierzenie bólu głowy, jedwabiste włosy, superniskie spalanie samochodu\dots Wierzymy w te obietnice w większości dlatego, że dani producenci wywiązali się z innych obietnic w przeszłości. Kupiwszy toyotę, jeśli przez parę lat jeżdżenia nią nie będziemy mieć żadnych problemów -- następnym samochodem prawdopodobnie będzie znów toyota. Ponieważ producent obiecał niezawodność, a produkt wywiązał się z obowiązków. Tego samego spodziewamy się po następcy. A po kolejnym szczęśliwym zakupie będziemy mieli wyrobioną mocną opinię na temat samochodów tej marki -- wiemy, że są one niezawodne i że jest to powtarzalne. Kupując trzecią toyotę redukujemy ryzyko -- skoro dwa poprzednie samochody okazały się fenomenalne, tego samego spodziewamy się po kolejnym modelu.

Spójrzmy na ten termin oczami Davida Aakera. W swojej książce ,,Building Strong Brands''\cite[s.~68]{aaker} na markę składają się cztery składniki:
\begin{description}
	\item[Produkt albo usługa] --- Producent tworzy produkt lub usługę, która staje się mocniejsza i bardziej znana od wytwórcy.\\
	Przykład: iPod $>$ Apple, Plus GSM $>$ Polkomtel
	\item[Osoba] --- Dzieje się tak, gdy z danym produktem, usługą lub producentem natychmiast kojarzy nam się jakaś osobistość, która jest twarzą, a czasem nawet i duszą danej rzeczy.\\
	Przykład: Steve Jobs -- Apple, Walt Disney -- Disney
	\item[Organizacja] --- W tym przypadku marką jest zwykle sam producent-firma.\\
	Przykład: Fiat, Philips
	\item[Symbol] --- Jeśli mamy mocną markę jako przynajmniej jeden z powyższych aspektów możemy próbować wykreować się na markę-symbol czasów, stylu życia, filozofii, czy danej gałęzi nauk.\\
	Przykład: iPod -- wolność, IBM -- informatyka.
\end{description}

\subsection{Czym się różni w takim razie dobra marka od złej marki?}
Dobra marka to ta marka, którą ludzie zauważają. \cite{pogorzelski} Porównajmy dwie marki: FRUGO oraz Guangzhou Bonisun Garment Co., Ltd. Pierwsza prezentuje serię napojów, druga garnitury. Co się stało, gdy z polskich sklepów zniknęły napoje FRUGO? Na portalach społecznościowych zaczęły się protesty, wydarzenia, petycje o przywrócenie produktów. Czy gdyby firma z Guangzhou zniknęła z rynku, czy ktoś by protestował? Czy ktoś by to w ogóle zauważył? To właśnie pokazuje przepaść pomiędzy mierną firmą, a ogólnokrajową marką.

\section{Jak budować markę?}
\subsection{Historyczne budowanie marek}
Aby dobrze podejść do budowania marki dziś, trzeba wiedzieć, jak robiono to kiedyś. W 2005 roku Steve Jobs na mowie rozpoczynającej rok szkolny opowiedział historię o ,,łączeniu kropek''.
\begin{quote}
Nie można łączyć kropek patrząc w przód, tylko patrząc w tył. \begin{flushright}\cite{jobs}\end{flushright}
\end{quote}

A więc jak budowano marki w dawnych czasach? Opowieściami. \cite{wyklad2} Gdy na bazar przychodził klient i chciał kupić kokosa -- sprzedawca zaczynał opowiadać mu historię, jak wysokie są palmy kokosowe, w jak pięknym środowisku wyrastają, jak ciężko jest zdjąć kokosa, jak daleko to jest\dots Przez to konsument czuł się wyjątkowy -- no w końcu taki kokos z taką historią -- wyjątkowy! Mimo ponad dwutysięcznego postępu opowiedzenie historii nadal jest podstawą marketingu.

\subsection{Przygotowanie}
Zanim zaczniemy budować markę, musimy jasno określić, czym ma być nasza marka, a czym ma na pewno nie być. \cite[s.~18]{tkaczykzakamarki} Jeśli zamierzamy stać się w oczach klientów specjalistą od napraw kosiarek, możemy w dalszej perspektywie zacząć je sprzedawać (domena działalności pozostaje mniej więcej taka sama), ale nie możemy pozwolić sobie na sprzedaż np. oczek wodnych. Wtedy cały wizerunek runie, a autorytet zniknie. \cite{wyklad2}
\subsection{Przydatność i użyteczność}
Gdy już jesteśmy gotowi ze swoim produktem i strategią marki, powinniśmy zająć się przyciągnięciem konsumentów. \cite{kall, tkaczykzakamarki} Na tym etapie musimy udowodnić potencjalnym klientom, że nasz produkt jest przydatny oraz dostępny. Ten punkt przypomina nam o tym, że nie da się stworzyć mocnej marki dla beznadziejnego produktu. Musimy udowodnić, że komuś, kto dotychczas obchodził się bez naszego rewolucyjnego produktu nie może się więcej bez niego obejść. A to nie jest łatwe. \cite[s.~109]{pogorzelski} Produkty są zwykle odbierane na dwóch poziomach w naszych mózgach. A oto jak je wykorzystać w praktyce:
\subsubsection{Poziom umysłu}
Poziom umysłu dotyka, możnaby rzec, sensownych cech produktu. Tych namacalnych. Podpowiada nam je rozum, ośrodek racjonalnego myślenia. Dzięki niemu wiemy na przykład, że to pióro jest wygodne, dobrze się je trzyma, ma cienką stalówkę itd.
\subsubsection{Poziom serca}
,,Serce'' natomiast podpowiada nam, że to pióro ładnie wygląda, że podnosi mój status społeczny, pisząc kartkówkę, czuję się jak szef w dużej firmie podpisujący umowę.

Bardzo ładnie te oba poziomy wykorzystuje TPSA w swoim cyklu reklam ,,Serce i Rozum''. Obaj bohaterowie prezentują różne argumenty, trafiające do obu naszych sfer decydujących o zakupie.

\subsection{Dotrzymywanie obietnic}
Gdy już przedstawiliśmy klientowi swój produkt i jakie korzyści zyska -- musimy zadbać o to, by faktycznie je zyskał. Niezbędny okaże się tutaj dobry produkt, sprawdzona oraz dopracowana usługa. Musimy pamiętać, że w pamięci klienta dużo mocniej zarysowują się złe wspomnienia -- koniecznie należy ich unikać. Ale jak wiadomo, zwłaszcza wśród branży usługowej nie da się takowych uniknąć. \cite{wyklad1} W takich momentach nie pomoże tylko świetna jakość finalnego obrazu usługi. Klient zapamięta przede wszystkim to, że coś na początku poszło nie tak. Wtedy jedynym ratunkiem pozostaje np. rabat na kolejną usługę. Badania pokazały, że klient pamięta średnią z ostatniego i najgorszego wrażenia w czasie korzystania z produktu. Skoro najgorszego nie da się poprawić -- pozostaje niespodzianka na koniec.

\subsection{Umocnienie pozycji na rynku}
Teraz trzeba wybrać, czy chcemy stać się marką-ekspertem, czy marką-parasolem. Czasem wybór jest prosty i nie da się zmienić, ale czasem wymaga namysłu. Marka-ekspert zyskuje coś nieprawdopodobnego, mianowicie zakorzenienie w kulturze. Jeśli będzie dość konsekwentna, zyska monopol w danej branży. Udało się to na przykład markom Adidas i Pampers. Dzięki mocno ograniczonemu wachlarzowi produktów i ograniczeniu się do specjalistycznej branży wszystkie sportowe buty stały się adidasami, a wszystkie pieluszki dla dzieci -- pampersami. Marka-parasol staje się natomiast miejscem, do którego idziemy, gdy chcemy kupić np. cokolwiek związanego z elektroniką (Media Markt, Saturn), cokolwiek do wyposażenia domu (IKEA), książkę, płytę z muzyką (empik). Dzięki tej postawie specjalizacja znika, pojawia się natomiast ,,wszystko o\dots''.

\section{Co dalej? -- \emph{social media}}
Tradycyjny marketing zamyka się na pewnej ograniczonej części wiedzy, jednak świat zmieniając się, niesie również zmiany dotyczące budowania marek. Paręnaście lat temu ,,marketingowi magicy'' nie wyobrażali sobie jak potoczą się losy komunikacji.

Dotychczas byliśmy zdani tylko na bierną komunikację z markami. Oglądaliśmy reklamy w gazetach, później przyszło radio. Wynaleziono telewizję, nadal klient nie miał możliwości odpowiedzi na reklamę. Dopiero gdy Internet dojrzał, pojawiła się możliwość komunikacji marki i konsumenta.

Teraz człowiek w galerii handlowej może w 19 sekund sprawdzić jak inni użytkownicy oceniają daną restaurację, gdzie wybrać się, jeśli jestem wielbicielem włoskiej kuchni, czy fryzjer na ulicy Słonecznej dobrze obcina\dots Pod koniec marca 2012 z serwisu Facebook korzystało 901 milionów użytkowników \cite{facebook}. Jest to potęga. W pewnych branżach marki nie wykorzystują dość potencjału portali społecznościowych, które dają ogromne możliwości firmom. Dzięki nim mogą pojawić się za darmo w życiu konsumentów, reklamować się darmowymi pozytywnymi opiniami klientów, którzy chcą (!) dzielić się z innymi swoimi wrażeniami. Trzeba to wykorzystać.
\section{Najnowszy trend -- grywalizacja}
Dobrze, mamy klientów. Ale do ludzkiego życia bardzo często wkrada się nuda i rutyna. Konsumenci przez to szukają ciekawych alternatyw, frajdy z zakupów. Grywalizacja, ogłoszona przez ,,\emph{Harvard Business Review}'' za jeden z wiodących trendów marketingowych drugiej dekady XXI wieku jest właśnie spełnieniem tego życzenia.\cite{tkaczykgrywalizacja}
\subsection{Jak dać klientowi radość z zakupów?}
Odejść od rutyny jest bardzo łatwo -- wystarczy wprowadzić element losowy. Weźmy choćby zupełnie zwykłą promocję w pubie -- ,,Kup jedno piwo, drugie gratis''. \cite{tkaczykwideo} Wyjątkowo pusta, zwykła promocja. Nie wzbudza żadnych emocji.

Ale ile się zmienia, gdyby wprowadzić element losowy. Zamiast prostych warunków -- losowanie. Gdy klient zamawia piwo, kelner proponuje mu rzut monetą. Jeśli wygra kelner, klient płaci za napój. Jeśli klient, dostaje piwo za darmo. Zwykła promocja nagle staje się czymś przyciągającym rzesze ,,szczęściarzy''. A restauracja płaci tyle samo.

Kawiarnie mają podobny problem -- zwykłe menu, niewiele się zmienia, codziennie Ci sami kelnerzy, taka sama kawa, taka sama kolejka. Dodajmy do tego grywalizację, nagradzajmy lojalnych klientów. Po pewnej ilości zakupów nadajmy tytuł ,,ambasadora restauracji'' i stwórzmy osobną kolejkę dla takich ambasadorów. Każdy chciałby zostać takim specjalnym gościem, jak najszybciej wyrobić potrzebną ilość kaw, aby z uśmiechem na twarzy przejść ze swoją kawą rano przed kolejką zwykłych ludzi. Różne osiągnięcia dla różnych typów graczy tak, aby każdy znalazł coś dla siebie. Przyniosłoby to kawiarni wiele nowych i stałych klientów.

\section{Zakończenie}
Proces budowania marki jest skomplikowany, a często i długi. Ta praca objęła sobą maksymalnie wiele wiedzy potrzebnej do stworzenia mocnego \emph{brand}-u. Od starych marketingowych prawideł po najnowsze trendy.
\clearpage
\bibliographystyle{apasoft}
\bibliography{bibliografia}
\end{document}