\documentclass [a4paper, 12pt, oneside]{article}
\usepackage{fullpage}
\usepackage[utf8]{inputenc}
\usepackage{polski}
\usepackage{hyperref}
\usepackage[usenames,dvipsnames]{color}
\usepackage[pdftex]{graphicx}
\usepackage{wrapfig}
\usepackage{float}
\usepackage{amsmath}
\usepackage{amsfonts}
\linespread{1.3}
\hypersetup{
    bookmarks=true,
    unicode=false,
    pdftoolbar=true,
    pdfmenubar=true,
    pdffitwindow=false,
    pdfstartview={FitH},
    pdftitle={Najważniejsze fakty z konstytucji},
    pdfauthor={Stanisław Chmiela},
    pdfsubject={Konstytucja},
    pdfcreator={Stanisław Chmiela},
    pdfproducer={Stanisław Chmiela},
    pdfkeywords={konstytucja} {wos},
    pdfnewwindow=false,
    colorlinks=true,
    linkcolor=BrickRed,
    citecolor=PineGreen,
    filecolor=RawSienna,
    urlcolor=MidnightBlue
}
\author{Stanisław Chmiela}
\title{Najważniejsze fakty z konstytucji}
\newcommand{\comment}[1]{\textcolor{Gray}{\textsf{\emph{(#1)}}}}
\begin{document}
\section{Podziękowania}
\begin{enumerate}
    \item Dominika Salawa -- autorka
    \item Aleksandra Bańkowska -- skanowaczka
    \item Stanisław Chmiela -- \TeX owacz
\end{enumerate}
\clearpage
\section{Rzeczpospolita Polska} % (fold)
\label{sec:rzeczpospolita_polska}
    Konstytucja jest najwyższym prawem. Władze kolejno sprawują:
    \begin{itemize}
        \item Ustawodawcza -- Sejm, Senat
        \item Wykonawcza -- Prezydent, Rada Ministrów
        \item Sądownicza -- sądy, trybunały
    \end{itemize}
\comment{\href{http://www.youtube.com/watch?v=C-u5WLJ9Yk4}{Klikniesz tu, czy nie klikniesz z ciekawości? xD}}
    \paragraph{Partie} wolno tworzyć, finansowanie ma być jawne. Dodatkowo można tworzyć związki zawodowe, zrzeszenia, fundacje itp. Za to mówimy zdecydowane NIE partiom nazistowskim, faszystowskim, komunistycznym, totalitarnym.
    \paragraph{Samorząd terytorialny} stanowi ogół mieszkańców terytorium.

    Można tworzyć samorządy zawodowe \comment{albo coś podobnego graficznie :D}.

    \paragraph{Małżeństwo, rodzina, macierzyństwo, rodzicielstwo} są pod ochroną RP. Inwalidzi wojenni są pod specjalną opieką państwa.

    \paragraph{Ustrój gospodarczy} społeczna gospodarka rynkowa jest jego podstawą. \comment{I coś tam jeszcze było, że rynek wolnym ma być -- tu jeszcze chciało mi się czytać Konstytucję.} Wywłaszczenie jest możliwe tylko na cele publiczne, za odszkodowaniem, a ograniczenie wolności działalności gospodarczej może być tylko w interesie publicznym. Podstawową jednostką ustroju rolnego jest gospodarstwo rodzinne. Kościół ma być niezależny.
% section rzeczpospolita_polska (end)

\section{Prawa i obowiązki obywatela} % (fold)
\label{sec:prawa_i_obowi_zki_obywatela}
    \begin{itemize}
        \item Brak dyskryminacji
        \item Kobieta = Mężczyzna
        \item Obywatel = syn Polaków
        \item Obywatelstwa można się zrzeknąć
        \item Zakaz eksperymentów na ludziach, tortur, kar cielesnych
        \item Każdy ma prawo do obrony, domniemanej niewinności
        \item Zbrodnie wojenne i przeciwko ludzkości nie mogą zostać przedawnione
        \item Wolność i ochrona tajemnicy komunikowania się
        \item Nienaruszalność mieszkania, pojazdu
        \item Władze publiczne nie mogą gromadzić nieniezbędnych informacji o obywatelu \comment{Redukcja!}
        \item Prawo do sprostowania informacji nieprawdziwych
        \item Prawo do posiadania świątyń własnej religii w kraju
        \item Cenzura zakazana
        \item Azyl dla cudzoziemców
        \item Związki zawodowe mogą organizować strajki
        \item Obywatel może uzyskiwać informacje o działaniu organów władzy
        \item Obywatel może uczestniczyć w posiedzeniach władzy i \comment{coś jakby ,,nadgryzać''}
        \item Aby uczestniczyć w głosowaniu lub referendum trza mieć 18 lat
        \item Prawo do dziedziczenia
        \item Wolność wyboru pracy
        \item Pracować można od 16 lat
        \item Mamy mieć zapewnione bezpieczne i higieniczne warunki pracy
        \item Wolne dni i urlopy też
        \item Do 18 lat mamy się uczyć \comment{hłe-hłe-hłe, i jeszcze czego}
        \item Działania przeciwdziałające bezdomności
        \item Trybunał Konstytucyjny rozwiązuje spór jeśli coś jest niezgodne z konstytucją
        \item Mamy płacić podatki
        \item I bronić ojczyzny
    \end{itemize}
% section prawa_i_obowi_zki_obywatela (end)

\section{Źródła prawa} % (fold)
\label{sec:_r_d_a_prawa}
    \begin{enumerate}
        \item Konstytucja
        \item Ustawy
        \item Umowy międzynarodowe
        \item Rozporządzenia
    \end{enumerate}
    Prezydent ratyfikuje ustawy międzynarodowe. Rzeczpospolita Polska może przekazać organizacji międzynarodowej kompetencje władzy państwowej.

    \paragraph{Ratyfikacja} wymaga zgody $\frac23$ z $\frac12$ posłów i $\frac23$ z $\frac12$ senatorów.

    Uchwały Rady Ministrów są tylko wewnętrzne.
% section _r_d_a_prawa (end)

\section{Sejm i Senat} % (fold)
\label{sec:sejm_i_senat}
    Sejm kontroluje Radę Ministrów. W jego skład wchodzi 460 posłów \comment{Na moje oko to ich za dużo jest.}, a w skład Senatu 100 senatorów.

    \paragraph{Kadencje} Wybierani są na 4-letnie kadencje. Wybory powinien zarządzić Prezydent co najmniej 90 dni przed końcem kadencji. Sejm może skrócić kadencję Sejmu lub Senatu, jeśli zgodzi się $\frac23$ posłów. 45 dni od skrócenia kadencji odbywają się wybory.

    \paragraph{Kandydaci} Minimum wiekowe do Sejmu wynosi 21 lat, do Senatu -- 30 lat. Nie można jednocześnie kandydować do Sejmu i Senatu. Posłowie ślubują. Pierwsze posiedzenie odbywa się 30 dni po wyborach. Sejm wybiera sobie Marszałka Sejmu i wicemarszałków.

    \paragraph{Zgromadzenie Narodowe} = Sejm + Senat

    \paragraph{Wojna} Sejm decyduje o wojnie, jeśli zagłosuje przeciwko decyzja przechodzi na Prezydenta.

    \paragraph{Władza ludu} $100~000$ obywateli może uchwalić ustawę. Do 30 dni Senat rozpatruje taką ustawę, do 21 dni Prezydent podpisuje taką ustawę. Referendum zarządza Sejm lub Prezydent za zgodą Senatu.
% section sejm_i_senat (end)

\section{Prezydent} % (fold)
\label{sec:prezydent}
    \paragraph{Kadencja} 5-letnia, wybrany może być tylko raz.

    \paragraph{Kandydat} Minimum 35 lat. Minimum $100~000$ popierających. \comment{Także jak będziecie zdobywać władzę w kraju siłą, to zostawcie sobie 100 000 ludzi, żeby Was poparli.}

    \paragraph{Wybory} zarządza Marszałek Sejmu. Sąd Najwyższy zatwierdza wybór. Dopóki Prezydent nie sprawuje władzy, pełniącym obowiązki Prezydenta jest Marszałek Sejmu. \comment{Hłe-hłe-hłe, jakbym ja bym był Marszałkiem to bym nie robił tych wyborów, hłe-hłe-hłe.}

    \paragraph{Kim jest?} Zwierzchnikiem Sił Zbrojnych, doradcą w Radzie Bezpieczeństwa Narodowego, razem z Radą Ministrów tworzą Radę Gabinetową.

    \paragraph{Pomocnik} -- Kancelaria Prezydenta.

    \paragraph{Przestępstwa} Jeśli takowe się pojawi Prezydent staje przed Trybunałem Stanu.
% section prezydent (end)

\section{Rada Ministrów i administracja rządowa} % (fold)
\label{sec:rada_ministr_w_i_administracja_rz_dowa}
    Wojewoda -- przedstawiciel Rady Ministrów. Prezydent $\rightarrow$ Prezes Rady Ministrów $\rightarrow$ Ministrowie. Sejm $\Rightarrow$ wotum zaufania.
% section rada_ministr_w_i_administracja_rz_dowa (end)

\section{Samorząd terytorialny} % (fold)
\label{sec:samorz_d_terytorialny}
    Mają prawo ustalania wysokości podatków. Prezes Rady Ministrów i wojewoda nadzorują.
% section samorz_d_terytorialny (end)

\comment{Ale muuuuuza, waka waka! Djangoooo eee eee djangooo ee ee!}

\section{Sądy i Trybunały} % (fold)
\label{sec:s_dy_i_trybuna_y}
    Prezydent na wniosek Krajowej Rady Ministrów powołuje sędziów na czas nieoznaczony. Pierwszego prezesa Sądu Najwyższego powołuje Prezydent na 6 lat.

    \paragraph{Trybunał Konstytucyjny} składa się z 15 sędziów (9-letnia kadencja) wybieranych przez Sejm.

    \paragraph{Trybunał Stanu} składa się z $2^4$ posłów lub senatorów, również wybieranych przez Sejm.
% section s_dy_i_trybuna_y (end)

\comment{Booooooomba! [King Africa -- La Bomba]}

\section{Organy Państwowej Ochrony Prawa} % (fold)
\label{sec:organy_pa_stwowej_ochrony_prawa}
    Najwyższa Izba Kontrolu kontroluje jednostki administracji rządowej. Poza tym analizuje wykonanie budrzetu \comment{Co ja pacze, błond ortograficzny!} państwa przez Sejm.

    Do tego dodajmy Rzecznika Praw Obywatelskich i Krajową Radę Radiofonii i Telewizji.
% section organy_pa_stwowej_ochrony_prawa (end)

\section{Finanse publiczne} % (fold)
\label{sec:finanse_publiczne}
    Sejm uchwala budżet. Rada Ministrów składa Sejmowi sprawozdanie. NBP emituje pieniądze.
% section finanse_publiczne (end)

\paragraph{Klęska żywiołowa} Rada Ministrów decyduje o stanie klęski żywiołowej.

\comment{We're no strangers to looooove\dots You know the ruuules and sooooo dooo IIIIIII!}
\end{document}