\documentclass [a4paper, 10pt, oneside]{article}
\usepackage{fullpage}
\usepackage[utf8]{inputenc}
\usepackage{polski}
\usepackage{hyperref}
\usepackage[usenames,dvipsnames]{color}
\usepackage[pdftex]{graphicx}
\usepackage{wrapfig}
\usepackage{float}
\usepackage{amsmath}
\usepackage{listings}
\definecolor{Brown}{cmyk}{0,0.81,1,0.60}
\definecolor{OliveGreen}{cmyk}{0.64,0,0.95,0.40}
\definecolor{CadetBlue}{cmyk}{0.62,0.57,0.23,0}
\definecolor{lightlightgray}{gray}{0.9}
\usepackage{amsfonts}
\linespread{1.3}
\hypersetup{
    bookmarks=true,
    unicode=false,
    pdftoolbar=true,
    pdfmenubar=true,
    pdffitwindow=false,
    pdfstartview={FitH},
    pdftitle={Zadanie z PHP - 18 kwi 2012},
    pdfauthor={},
    pdfsubject={PHP},
    pdfcreator={},
    pdfproducer={},
    pdfkeywords={php} {informatyka},
    pdfnewwindow=false,
    colorlinks=true,
    linkcolor=BrickRed,
    citecolor=PineGreen,
    filecolor=RawSienna,
    urlcolor=MidnightBlue
}
\author{}
\title{Zadanie z PHP - 18 kwi 2012}
\begin{document}
\lstset{
language=HTML,                            % Code langugage
basicstyle=\ttfamily,                % Code font, Examples: \footnotesize, \ttfamily
keywordstyle=\color{OliveGreen},        % Keywords font ('*' = uppercase)
commentstyle=\color{Gray},            % Comments font
numbers=left,                          % Line nums position
numberstyle=\tiny,                    % Line-numbers fonts
stepnumber=1,                          % Step between two line-numbers
numbersep=5pt,                        % How far are line-numbers from code
%backgroundcolor=\color{lightlightgray}, % Choose background color
frame=2pt,                           % A frame around the code
tabsize=2,                            % Default tab size
captionpos=b,                          % Caption-position = bottom
breaklines=true,                        % Automatic line breaking?
breakatwhitespace=false,                % Automatic breaks only at whitespace?
showspaces=false,                      % Dont make spaces visible
showtabs=false,                      % Dont make tabls visible
columns=flexible,                      % Column format
morekeywords={__global__, __device__}  % CUDA specific keywords
}
\maketitle
\tableofcontents
\section{Część statyczna -- HTML}
W pliku \texttt{zapytaj.php} znajduje się szablon HTML (to zwyczajne, czyli \texttt{doctype}, \texttt{head}, \texttt{body} itd.) oraz kod poniżej:
\begin{lstlisting}
<form action="./odpowiedz.php" method="post">
    <label for="ssak">Ssak</label>
    <input type="text" size="30" maxlength="100" id="ssak" name="ssak" />

    <label for="rozmiar">Rozmiar</label>
    <input type="text" size="10" maxlength="10" id="rozmiar" name="rozmiar" />

    <label for="liczba">Liczba</label>
    <input type="text" size="12" maxlength="12" id="liczba" name="liczba" />

    <input type="submit" value="Zapytaj" />
</form>
\end{lstlisting}
Pochylmy się nad poszczególnymi linijkami, aby dowiedzieć się, co każda z nich robi.
\subsection{Formularz}
\paragraph{Znacznik \texttt{form}} Tenże znacznik oznacza, że w nim znajduje się formularz. Atrybut \texttt{action} definiuje do jakiego pliku ma zostać wysłana zawartość formularza (tam musi się znajdować skrypt z odpowiedzią), a \texttt{method} -- jaką \hyperlink{metody}{metodą}.
\paragraph{Znacznik \texttt{label}} służy do oznaczania etykiet dla pól formularza. Po prostu -- taki znacznik istnieje, więc się go używa. Jako atrybut możemy dać \texttt{for}, w które wpisujemy identyfikator pola, dla którego etykieta jest przeznaczona.
\paragraph{Znacznik \texttt{input}} to zwyczajnie pole do wprowadzania. Różne są jego rodzaje (określane przez \texttt{type}): text, email, submit, reset\dots My wykorzystujemy głównie \texttt{text} -- służy do wpisywania tekstu oraz \texttt{submit} -- przycisk do wysłania formularza.

\subparagraph{Text} Dla tego pola możemy nadać przeróżne atrybuty, oto najważniejsze:
\begin{itemize}
    \item \texttt{size} -- długość pola mierzona w znakach,
    \item \texttt{maxlength} -- maksymalna długość ciągu jaki można wpisać w pole,
    \item \texttt{id} -- identyfikator pola,
    \item \texttt{name} -- nazwa pola, wykorzystywana później w pliku odpowiedzi.
\end{itemize}

\subparagraph{Submit} Najważniejszym atrybutem jest \texttt{value}, które zwyczajnie mówi jaki tekst ma się wyświetlić na przycisku.

\section{Kod PHP}
\lstset{
language=PHP,                            % Code langugage
basicstyle=\ttfamily,                % Code font, Examples: \footnotesize, \ttfamily
keywordstyle=\color{OliveGreen},        % Keywords font ('*' = uppercase)
commentstyle=\color{Gray},            % Comments font
numbers=left,                          % Line nums position
numberstyle=\tiny,                    % Line-numbers fonts
stepnumber=1,                          % Step between two line-numbers
numbersep=5pt,                        % How far are line-numbers from code
%backgroundcolor=\color{lightlightgray}, % Choose background color
frame=2pt,                           % A frame around the code
tabsize=2,                            % Default tab size
captionpos=b,                          % Caption-position = bottom
breaklines=true,                        % Automatic line breaking?
breakatwhitespace=false,                % Automatic breaks only at whitespace?
showspaces=false,                      % Dont make spaces visible
showtabs=false,                      % Dont make tabls visible
columns=flexible,                      % Column format
morekeywords={__global__, __device__}  % CUDA specific keywords
}
\begin{lstlisting}
<?php
$result = "<p>Nie ma.</p>";
$file = fopen("nazwy.txt", "r");
while(!feof($file))
{
    if(trim(fgets($file)) == trim($_POST['ssak']) && isset($_POST['ssak']) && $_POST['ssak'] != '')
    {
        $result =  "<p>OK.</p>";
    }
}
fclose($file);
echo $result;
if(is_integer($_POST['rozmiar']) && $_POST['rozmiar'] >= 10 && $_POST['rozmiar'] <= 100)
{
    echo "<p>OK.</p>";
}
else
{
    echo "<p>Rozmiar spoza zakresu.</p>";
}
$file = fopen("liczby.txt", "a");
fwrite($file,"\n".$_POST['liczba']);
fclose($file);
echo "<p>".$_POST['rozmiar']*$_POST['liczba']."</p>";
?>
\end{lstlisting}

Przypomnijmy sobie najpierw o co chodzi w zadaniu. Mamy sprawdzić, czy w pliku \texttt{nazwy.txt} istnieje nazwa gatunku wysłana przez użytkownika, sprawdzić parę rzeczy jeśli chodzi o rozmiar podany w formularzu, a na koniec dopisać liczbę podaną przez użytkownika do pliku \texttt{liczby.txt}.

\subsection{Funkcje}

W kodzie tym znajduje się dużo interesujących funkcji, więc najpierw opiszę co każda z nich robi, a później będziemy oglądać na przykładzie tegoż pliku:

\begin{description}
    \item[\texttt{fopen(\textit{nazwa-pliku}, \textit{tryb})}] Otwiera plik \texttt{\textit{nazwa-pliku}}. Musimy zapisać to co funkcja zwraca do zmiennej, aby móc później korzystać z tego otwartego pliku. Aby poczytać o różnych trybach otwierania pliku, \hyperlink{tryby}{kliknij tu}.
    \item[\texttt{feof(\textit{otwarty-plik})}] Sprawdza, czy wskaźnik czytający plik doszedł już do końca pliku. Zwraca \texttt{true} lub \texttt{false}.
    \item[\texttt{fgets(\textit{otwarty-plik})}] Czyta plik linia po linii, przesuwając wskaźnik. Zwraca kolejną linię pliku.
    \item[\texttt{fclose(\textit{otwarty-plik})}] Zamyka plik, zapisuje zmiany.
    \item[\texttt{fwrite(\textit{otwarty-plik}, \textit{\texttt{tekst-do-wpisania}})}] Dopisuje do \texttt{\textit{otwartego-pliku}} \texttt{\textit{tekst-do-wpisania}}.
    \item[\texttt{trim(\textit{ciag-znakow})}] Usuwa ,,białe znaki'' z początku i końca ciągu znaków. Białe znaki to spacje, znaki nowej linii i takie inne.
    \item[\texttt{isset(\textit{nazwa-zmiennej})}] Sprawdza, czy zmienna o danej nazwie została już gdzieś zdefiniowana.
    \item[\texttt{echo(\textit{tekst})}] Wypisuje tekst.
    \item[\texttt{is\_integer(\textit{zmienna})}] Sprawdza, czy zmienna jest liczbą całkowitą. Zwraca \texttt{true} lub \texttt{false}.
\end{description}

\subsection{Czytanie pliku przez PHP}
Zacznijmy od tego tematu. Gdy PHP musi przeczytać jakiś plik, ustawia sobie najpierw wskaźnik, a następnie przesuwa go w te, czy wewte. Wskaźnik -- uznajmy to za metaforyczną kreseczkę, jaką się widuje w edytorach tekstu. Gdy PHP otwiera plik w trybie ,,r'' ustawia sobie wskaźnik na początek pliku, a następnie gdy wywołujemy funkcję \texttt{fgets} bierze wskaźnik i przesuwa go coraz to dalej, aż natrafi na znak nowej linii. Wyobraźmy to sobie jako przesuwanie strzałką w prawo kursora w tekście. Gdy znajdzie koniec linii, całą linię kopiuje do schowka, a później wkleja nam i wypluwa.
\subsection{Wyjaśnienie kodu}
Stąd taka konstrukcja kodu. Najpierw otwieramy odpowiedni plik w odpowiednim trybie, a później dopóki wskaźnik nie doszedł do końca pliku (\texttt{feof}) pobieramy kolejne linie (\texttt{fgets}) i sprawdzamy, czy nie są równe gatunkowi ryby. Używamy tam funkcji \texttt{trim}, ale czemu. Otóż mam wrażenie, że \texttt{fgets} podaje linię z pliku razem ze znakiem nowej linii na końcu (której nie ma w danej z formularza), więc tą funkcją zwyczajnie usuwamy zbędną linię. Dodatkowo sprawdzamy, czy w ogóle formularz został wysłany i czy podany gatunek nie jest pusty. Takie zabezpieczające linie. No i jeśli linia z pliku równa jest gatunkowi, to odkryliśmy, że podany gatunek jest w pliku.

Kolejna linia sprawdza warunki na podanym rozmiarze.
\begin{enumerate}
    \item Sprawdza, czy jest liczbą całkowitą.
    \item Sprawdza, czy jest większe lub równe 10.
    \item Sprawdza, czy jest mniejsze lub równe 100.
\end{enumerate}
Jeśli liczba spełnia te warunki -- wypisujemy OK. W przeciwnym wypadku co innego.

No i na koniec przechodzimy do dodania liczby do pliku tekstowego. Podobnie jak przy wczytywaniu:
\begin{enumerate}
    \item Otwieramy plik w odpowiednim \hyperlink{tryby}{trybie}.
    \item Dopisujemy do niego ciąg \texttt{"\textbackslash n".\$\_POST['liczba']}, czyli znak nowej linii połączony z liczbą.
    \item Zamykamy edycję pliku.
\end{enumerate}

No rzecz ostatnia -- wypisujemy pomnożone rozmiar i liczbę.

\section{Różności o różnościach}
\subsection {Metody wysyłania formularzy}
\hypertarget{metody}{Istnieją dwie metody wysyłania formularzy -- GET i POST.} Różnica jest prosta.

Dane formularza wysyłane metodą GET są dodawane do linku. Czyli, jeśli formularz wysyłamy metodą GET, a mamy w nim pole tekstowe ,,dlugosc'', w które wpisaliśmy 144, to po kliknięciu przycisku wysłania formularza (zakładając, że \texttt{action="./response.php"}) przejdziemy na stronę\\ \texttt{http://costam/folder/response.php?dlugosc=144}.

Dla formularza:
\begin{verbatim}
<form action="./odpowiedz.php" method="get">
    <input type="text" name="interesujace-pole" /> <!-- Wyp. zawartosc1 -->
    <input type="text" name="nieinteresujace-pole" /> <!-- Wyp. zawartosc2 -->
    <input type="submit" value="Wyślij, proszę. :)" />
</form>
\end{verbatim}
Po nacisnięciu przycisku przejdziemy na stronę\\
\texttt{plepleple/odpowiedz.php?interesujace-pole=zawartosc1\&nieinteresujace-pole=zawartosc2}.

W pliku odpowiedzi możemy dostać tą wartość za pomocą zmiennej \texttt{\$\_GET['nazwa-pola']}. Jak choćby w powyższym przypadku mamy dostęp do \texttt{\$\_GET['interesujace-pole']} oraz do \texttt{\$\_GET['nieinteresujace-pole']}.

Jeśli chodzi natomiast o metodę POST -- dane wysyłane są ,,tajemnymi drogami'', nie widzimy ich w adresie strony. A z poziomu skryptu odpowiedzi korzystamy z tablicy \texttt{\$\_POST}. Jak choćby:
\begin{verbatim}
<form action="./odpowiedz-inna-a-co.php" method="post">
    <input type="text" name="laaaaal" /> <!-- Wyp. zawartoscnumerjeden -->
    <input type="text" name="dziaala" /> <!-- Wyp. zawartoscnumerdwa -->
    <input type="submit" value="Wyślij, proszę. :)" />
</form>
\end{verbatim}
W pliku \texttt{odpowiedz-inna-a-co.php} mamy dostęp do dwóch zmiennych:
\begin{itemize}
    \item \texttt{\$\_POST['laaaaal'] = 'zawartoscnumerjeden'}
    \item \texttt{\$\_POST['dziaala'] = 'zawartoscnumerdwa'}
\end{itemize}
\subsection{Tryby otwierania pliku}
\hypertarget{tryby}{} Można różnie otwierać plik w PHP. Dzięki temu wiemy, że np. otwierając ten plik nie nadpiszemy sobie nic, że przypadkiem ktoś niepowołany nie przeczyta go itd. A oto jak wyglądają i co robią poszczególne tryby:
\begin{center}\begin{tabular}{|p{1.5cm}|p{10cm}|}
\hline
r & Pozwala na czytanie pliku. Wskaźnik na początku pliku.\\
\hline
r+ & Pozwala na czytanie i zapisywanie w pliku. Wskaźnik na początku.\\
\hline 
w & Pozwala na zapisywanie do pliku. Jeśli plik istnieje -- otwiera go i czyści. Jeśli nie istnieje -- tworzy nowy.\\
\hline
w+ & Pozwala na czytanie i zapisywanie pliku. Jeśli plik istnieje -- otwiera go i czyści. Jeśli nie istnieje -- tworzy nowy.\\
\hline
a & Pozwala na dodawanie tekstu do istniejącego pliku. Jeśli plik istnieje -- otwiera go i wskaźnik daje na koniec. Jeśli nie istnieje -- tworzy nowy.\\
\hline
a+ & Pozwala na dodawanie tekstu do istniejącego pliku oraz czytanie jego zawartości. Jeśli plik istnieje -- otwiera go i wskaźnik daje na koniec. Jeśli nie istnieje -- tworzy nowy.\\
\hline
x & Pozwala na zapisywanie do pliku. Tworzy nowy.\\
\hline
x+ & Pozwala na zapisywanie do pliku i czytanie jego zawartości. Tworzy nowy.\\
\hline
\end{tabular}\end{center}

\end{document}