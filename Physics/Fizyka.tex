\documentclass [a4paper, 11pt, oneside]{book}
\usepackage{fullpage}
\usepackage[utf8]{inputenc}
\usepackage{polski}
\usepackage{hyperref}
\usepackage[usenames,dvipsnames]{color}
\hypersetup{
	bookmarks=true,         % show bookmarks bar?
	unicode=false,          % non-Latin characters in Acrobat’s bookmarks
	pdftoolbar=true,        % show Acrobat’s toolbar?
	pdfmenubar=true,        % show Acrobat’s menu?
	pdffitwindow=false,     % window fit to page when opened
	pdfstartview={FitH},    % fits the width of the page to the window
	pdftitle={Fizyka},    % title
	pdfauthor={Stanisław Chmiela},     % author
	pdfsubject={Notatki - fizyka},   % subject of the document
	pdfcreator={Stanisław Chmiela},   % creator of the document
	pdfproducer={Stanisław Chmiela}, % producer of the document
	pdfkeywords={fizyka} {notatki}, % list of keywords
	pdfnewwindow=false,      % links in new window
	colorlinks=true,       % false: boxed links; true: colored links
	linkcolor=BrickRed,          % color of internal links
	citecolor=PineGreen,        % color of links to bibliography
	filecolor=RawSienna,      % color of file links
	urlcolor=MidnightBlue      % color of external links
}
\usepackage[pdftex]{graphicx}
\usepackage{wrapfig}
\usepackage{float}
\usepackage{amsmath}
\linespread{1.3}
\author{Stanisław Chmiela}
\title{Fizyka}
\begin{document}
\maketitle
\tableofcontents
\part{Semestr I}
\chapter{2 września 2011 -- \textit{Wymagania edukacyjne z fizyki w klasie drugiej}}
	\section{Wymagane materiały}
		\begin{description}
		\item[Podręczniki] \textit{,,Wybieram fizykę''} część II
		\item[Dodatkowa książka] \textit{,,Podstawy fizyki''}, Heilmana
		\item[Zapisać się na ,,moodla'']
		\end{description}
	\section{Wymagania edukacyjne}
		\begin{enumerate}
			\item Posiadamy zeszyt, co by zapisywać w nim zadania domowe itd. (Jedynek z zadań domowych nie poprawiamy.)
			\item Wagi ocen:
			\begin{itemize}
				\item Sprawdziany: 3
				\item Odpowiedź ustna, kartkówka: 2
				\item Oceny z aktywności, typu referaty, zadania domowe, aktywność: 1
			\end{itemize}
			\item Do oceny śródrocznej bądź końcoworocznej dokładamy $0.2$ za frekwencję powyżej $90\%$.
			\item Za udział w konkursach (z sukcesem) również $0.2$.
			\item Za uczestnictwo w projektach takich jak Feniks, kółku fizycznym dostaje się $0.2$.
			\item Na ocenę celującą Pan wymaga udziału w jednej z olimpiad fizycznych z sukcesem.
			\item Sprawdzian obejmuje najmniej 3 tematy, zapowiadany z conajmniej tygodniowym wyprzedzeniem. Zostaną sprawdzone w terminie dwóch tygodni od napisania. Kto dostaje jedynkę -- poprawia, kto dwójkę -- nie może poprawiać.
			\item Kartkówki są niezapowiadane, obejmują do 3 tematów lekcyjnych.
			\item Do wykorzystania dwa braki zadań i dwa nieprzygotowania. Zgłaszane będą po sprawdzeniu obecności, werbalnie, po pytaniu nauczyciela.
			\item Oceny śródroczne i końcoworoczne mogą poprawić osoby, które mają minimum $80\%$ frekwencji.
			\item W trakcie sprawdzianu uczeń może korzystać z kalkulatorów (jakichśtam niespecjalnych) oraz z niepopisanych tablic wzorów.
		\end{enumerate}
\chapter{6 września 2011 -- \textit{Teoria Wielkiego Wybuchu}}
	\section{Wstęp}
	Nazwa jest nazwą prześmiewczą, wymyśloną przez przeciwnika teorii. Wymyślona w latach 30. XX wieku. Oficjalna nazwa teorii to \textbf{\textit{alfa--beta--gamma}}. Człony odpowiadają nazwiskom autorów.

	Przestrzeń i czas zostały stworzone w Wielkim Wybuchu. Na początku powstania Wszechświata przestrzeń była \textbf{całkowicie wypełniona materią}. Materia była początkowo bardzo gorąca i gęsta, rozszerzając się ulegała ochłodzeniu, aby w końcu utworzyć gwiazdy i galaktyki, które widzimy dzisiaj we Wszechświecie.

	W chwili Wielkiego Wybuchu Wszechświat miał zerowy promień, a zatem nieskończenie wysoką temperaturę. W miarę jak wzrastał promień Wszechświata, temperatura promieniowania spadała.
	$$T = \frac{10^{10}}{t^{\frac12}}$$
	\begin{flushright}
		\textit{$t$ w sekundach, $T$ w Kelwinach}
	\end{flushright}

	\section{Era Plancka}
	Trwała od 0 do $10^{-43}$ sekundy.

	Stan Wszechświata w erze Plancka nie może być opisany za pomocą równań klasycznej ogólnej teorii względności, gdyż efekty kwantowe odgrywają wówczas zasadniczą rolę i do poprawnego opisu potrzebna jest kwantowa teoria grawitacji, której obecnie nie ma, choć do jej miana aspiruje kilka teorii np. pętlowa grawitacja kwantowa, M--teoria, teoria strun.

	Z erą Plancka związanych jest kilka parametrów, opisujących stan Wszechświata w trakcie ery:
	\begin{description}
		\item[Czas Plancka] $5.391\times 10^{-44}~s$
		\item[Długość Plancka] $7.4\times 10^{-26}~m$
		\item[i inne$\dots$]
	\end{description}

	\section{Era plazmy kwarkowo--gluonowej (hadronowa)}
	Od $10^{-43}$ do $10^{-4}$ sekundy.

	Na początku wszystkie oddziaływania, z wyjątkiem grawitacyjnego, czyli elektromagnetyczne, słabe i silne miały jednakowe znaczenie i były nierozróżnialne. Między tymi oddziaływaniami występowała symetria. Ten okres nazywa się \textbf{wielką unifikacją}.

	Po obniżeniu się temperatury cięższe kwarki zaczęły się rozpadać, a lżejsze zaczęły łączyć się w hadrony. Najrozmaitsze odmiany hadronów znajdowały się w równowadze termodynamicznej ze sobą, nie tylko te najbardziej trwałe takie jak protony, neutrony, hiperony, piony, kaony, ale wiele krótkożyjących rezonansów.

	\textbf{Symetria została złamana w chwili $10^{-35}$ sekundy}, kiedy temperatura spadła do wartości $10^{28}~K$. Oddziaływanie silne oddzieliło się wtedy od oddziaływania słabego i elektromagnetycznego, a jego moc zaczęła przewyższać moc dwóch pozostałych, jak ma to miejsce dzisiaj. Zjawisko nazywa się \textbf{inflacją Wszechświata}. Konsekwencją złamania symetrii było wydzielenie się wielkiej ilości energii.

	Gdyby liczby cząstek materii i antymaterii były dokładnie take same całość uległaby anihilacji fotonów. Jednak powstała asymetria. 

	\section{Era leptonowa}
	Od $10^{-4}$ sekundy do 10 sekund.

	W poprzedniej erze istniały również leptony, ale stanowiły jedynie nic nie znaczącą domieszkę. Obecnie to leptony synęły się na pierwsze miejsce. Powstawały pary:
	$$\gamma\gamma \leftrightarrow e^+ e^-$$
	elektron--pozyton mion--antymion, taon--antytaon neutrino--antyneutrino

	Wraz ze spadkiem temperatury malał proces powstawania par lepton--antylepton, a więcej było procesów anihilacji.

	W pierwszej kolejności zanihilowały cięższe cząstki, czyli miony i taony. W tej erze neutrina praktycznie przestały oddziaływać z pozostałą materią i rozproszyły się. Jest więc nadzieja, że w przyszłości wykryjemy je w postaci ,,reliktowych neutrin tła''.

	We wczesnym wszechświecie była duża temperatura, a więc również były wysokie energie, wypełniających go fotonów. Mogły więc zachodzić reakcje kreacji par cząstka--antycząstka.

	$\gamma\gamma \leftrightarrow e^+ e^-$ zakończyła się, gdy $kT \ge m_e c^2 = 0.5MeV$

	Analogicznie wcześniejsza \textit{epoka hadronowa} skończyła się, gdy anihilacje nawet najlżejszych hadronów (tj. mezonów $\pi$) stały się nieodwracalne, a zatem gdy $kT \approx m_\pi c^2\approx150MeV$, czyli $T\approx10^{12}~K$. Oznacza to czas trwania tej epoki rzędu $t\approx 10^{-4}~s$.

	Pod koniec tej ery keptonowej zaczęły rozpadać się neutrony, które są cząstkami nietrwałymi. Część z nich uniknęła zagładzie, łącząc się z protonami w stabilne jądra (nukleosynteza).

	\section{Era promieniowania}
	Od 10 sekundy do 300000 lat

	Po około 10 sekundach elektrony i ich antycząstki zanihilowały, pozostawiając niewielką nadwyżkę elektronów, której istnienie tłumaczymy również z zasady łamania symetrii. Zaczęła się era promieniowania, w której Wszechświat był wypełniony głównie foronami z niewielką domieszką protonów i neutronów oraz minimalnymi ilościami helu. Cząstki te nieustannie oddziaływały ze sobą i temperatura promieniowania była równa temperaturze materii, Wszechświat był nieprzezroczysty.

	\subsection{Promieniowanie tła, reliktowe}
	Widmo mikrofalowego promieniowania tła okazało się widmem ciała doskonale czarnego o temperaturze $2.7~K$. Potwierdziło to przewidywania, że jst ono pozostałością (reliktem) po początkowych fazach ewulucji Wszechświata, gdy wypełniała go bardzo gęsta i gorąca materia oraz będące z nią w równowadze promieniowanie o widmie Plancka.

	W przybliżeniu 300 000 lat po Wielkim Wybuchu promieniowanie przestało oddziaływać z materii, a do dziś wskutek nieustannej ekspansji Wszechświata temperatura promieniowania spadła do poziomu $2.7~K$.

	\section{Prawdziwość Teorii Wielkiego Wybuchu}
	\begin{enumerate}
	\item Zastosowanie teleskopów radiowych oraz potwierdzenie przewidywań, co do faktu, że im dalej w głąb i co za tym idzie w przeszłość Wszechświata, tym materia Wszechświata jest gęstsza, a galaktyki są skupione bliżej siebie.
	\item Odkrycie kwazarów (quasi--stellar) -- obiektów o ogromnej masie, skupionych w niewielkiej przestrzeni i emitujących poteżnej ilośc energii. Obecnie uważa się, że kwazary są aktywnymi jądrami młodych galaktyk i powstały one ogromnie dawno temu, mają -- prawdopodobnie ponad 10 miliardów lat. W obecnym Wszechświecie takich obiektów już nie ma, ani żadnych im podobnych. Ich widma są przesunięte daleko w kierunku czerwieni, co świadczy o ich wielkiej prędkości.
	\item Potwierdzenie istnienia promieniowania tła (za pomocą anteny Echo1). Odkrycia dokonali Arno Penzias i Robert Wilson w 1964 roku.
	\end{enumerate}
	Ostatecznie słuszność teorii Wielkiego Wybuchu potwierdził wybitny naukowiec George Smoot.

	\section{Problem helu}
	Helu we wszechświecie jest za dużo, jeżeli tylko byłby produkowany w syntezie termojądrowej w gwiazdach to byłoby go we Wszechświecie tylko $4\%$, ale jest go $25\%$, co oznacza, że Wszechświat kiedyś musiał być na tyle gorący, by powstała reszta brakującego helu.

\chapter{8 września 2011 -- \textit{Ogólna teoria względności}} % (fold)
\label{cha:8_wrze_nia_2011_textit}
	\section{Albert Einstein} % (fold)
	\label{sec:albert_einstein}
		Żył w latach 1879--1955. Gdy ogłaszał światu (1915) teorię względności, był stosunkowo młody.
	% section albert_einstein (end)
	\section{Podstawowe założenia} % (fold)
	\label{sec:podstawowe_za_o_enia}
		\subsection{Zasada kosmologiczna} % (fold)
		\label{sub:zasada_kosmologiczna}
			Wszechświat w wielkiej skali jest jednorodny (\textit{taki sam ze względu na translację}) i izotropowy (\textit{taki sam ze względu na obrót}).
		% subsection zasada_kosmologiczna (end)

		\subsection{Postulat ogólnej kowariantności} % (fold)
		\label{sub:postulat_ogolnej_kowariantnosci}
			Wszystkie układy odniesienia są równoprawne, a podstawowe równania fizyki są niezmiennicze względem transformacji z jednego układu do drugiego.
		% subsection  (end)

		\subsection{Zasada równoważności} % (fold)
		\label{sub:zasada_r_wnowa_no_ci}
			Nieinercjalny układ odniesienia o przyspieszeniu $a$, będący poza polem grawitacyjnym jest lokalnie równoważny (nieodróżnialny) układowi inercjalnemu, a w polu grawitacyjnym o przyspieszeniu $|g| = |a|$.

			Inaczej mówiąc -- ,,sztuczna grawitacja'', powstająca NUO jest (lokalnie) nieodróżnialna od grawitacji ,,prawdziwej'' powstającej wokół mas.

			Masa grawitacyjna i masa bezwładna są sobie równe. Swobodny spadek w polu grawitacyjnym z przyspieszeniem $g$ jest więc (lokalnie) równoważny swobodnemu, jednostajnemu (inercjalnemu) ruchowi z dala od pól grawitacyjnych.

			Należy podkreślić lokalność opisanej powyżej równoważności.
		% subsection zasada_r_wnowa_no_ci (end)
	% section podstawowe_za_o_enia (end)

	\section{Oddziaływanie na czasoprzestrzeń} % (fold)
	\label{sec:oddzia_ywanie_na_czasoprzestrze_}
		Stojąc w polu grawitacyjnym powinniśmy zaobserwować efekty, świadczące o nieeuklidesowości czasoprzestrzeni wokół nas (np. ugięcie trajektorii promienia świetlnego). OTW sprowadziła pole grawitacyjne do nieeuklidesowej czasoprzestrzeni. Innymi słowy -- rozkład masy (energii) w czasoprzestrzeni determinuje jej geometrię, ta zaś określa ruch mas w czasoprzestrzeni.
		\begin{center}
			\textbf{Masy zakrzywiają czasoprzestrzeń, a grawitacja wynika z czasoprzestrzeni.}
		\end{center}
		W czasoprzestrzeni zakrzywionej światło porusza się po torach, które są liniami o najmniejszej długości\dots (linie geodezyjne).
	% section oddzia_ywanie_na_czasoprzestrze_ (end)

	\section{Grawitacja} % (fold)
	\label{sec:grawitacja}
		Promień Schwarzschilda określa rozmiar horyzontu zdarzeń.
		\[
			r_s = \frac{2GM}{c^2}
		\]
		Zakrzywienie czasoprzestrzeni oznacza, że nie tylko przestrzeń jest zakrzywiona, a również czas.
		\[
			\Delta t_1 = \sqrt{1-\frac{r_s}{r}}\cdot\Delta t
		\]
		\begin{flushright}
			\it $\Delta t_1$ --- Odstęp czasu w polu grawitacyjnym, $\Delta t$ --- Odstęp czasu mierzony bez pola grawitacyjnego
		\end{flushright}

		Konsekwencją spowolnienia czasu jest grawitacyjne przesunięcie ku czerwieni światła emitowanego z powierzchni gwiazdy. Na przykład:
		\begin{quote}Na powierzchni Słońca spowolnienie czasu, a więc i $z = \frac{\Delta\lambda}{\lambda}$ wynosi $2\cdot10^{-6}$. Jim Brault potwierdził to doświadczenie w latach 60 XX wieku.\end{quote}

		Nieskończenie rozległe \textit{ciasto z rodzynkami} (galaktykami) puchnące wszędzie równomiernie przy ogrzewaniu (drożdze są w nim wszędzie) -- jest to 3-wymiarowy model wszechświata otwartego ($V = \infty$) i płaskiego ($k=0$).

		Odległość dowolnej pary punktów może być zawsze zapisana jako:
		\[
			r(t) = r(t_0)R(t)
		\]
		gdzie $R(t)$ nazywa się \textit{funkcją skalującą}. Podanie modelu wszechświata oznacza przede wszystkim podanie konkretnej postaci matematycznej funkcji $R(t)$.

		Fundamentalne prawo Hubble'a $v = Hr$ staje się teraz automatyczną i wizualnie oczywistą konsekwencją rozszerzania izotropowego (poprzednie przykłady). Przy tym obliczeniowo:
		\[
			v = r(t) = r(t_0)R(t) = r(t_0)R(t)\frac{R(t)}{R(t)} = \frac{R(t)}{R(t)}r = Hr
		\]
		skąd otrzymujemy dodatkową informację o stałej (tylko przestrzennie!) Hubble'a:
		\[
			H(t) = \frac{R}{R}
		\]
		Gdzieś tam były kropki nad niektórymi $R$.
	% section grawitacja (end)

	W teorii względności podstawowym obiektem matematycznym charakteryzującym geometryczne własności czasoprzesterzeni jest tzw. interwał czasoprzestrzenny, opisujący różniczkową odległość pomiędzy dwoma punktami (zdarzeniami) w czasoprzestrzeni. W STW, gdzie obowiązuje geometria (pseudo)euklidesowa i we współrzędnych kartezjańskich ma on znaną prostą postać:
	\[
		ds^2 = (dx^0)^2 - (dx^1)^2 - (dx^2)^2 - (dx^3)^2
	\]
	gdzie $x^0 = ct$. Ogólniej interwał można zapisać w postaci sumy:
	\[
		ds^2 = \sum^3_{i,j = 0} g_{ij}dx^idx^i
	\]
	Podstawowe równania kosmologiczne, opisujące geometrię czasoprzestrzeni wszechświata i jej ewolucję w czasie (równania Friedmanna):
	\[
		\dot R - \frac83\pi G\rho R^2 = kc^2
	\]
	\[
		2\ddot R R+\dot R^2+8\dots =-kc^2
	\]
	Rozważmy punkt materialny (galaktykę) o masie $m$ w odległości $R(t)$ od dowolnego centrum obserwacji (np. dla nas), oddalający się z prędkością $v$. Przyjmując, że efektywnie działa na niego grawitacyjnie jedynie masa $M$ kuli o promieniu $R$, dostajemy z zasady zachowania energii:
	\[
		\frac{mv^2}2 + \left(-\frac{GmM}R\right) = E-\textrm{const.}
	\]
	Ponieważ $v = \dot R$, zaś $M = \rho\frac43\pi R^3$ mamy
	\[
		\dot R^2 - \frac83\pi G\rho R^2 = \frac{2E}M
	\]
	co już jest w zasadzie równaniem Friedmanna (I-szym i ,,głównym'') na funkcję skalującą $R(t)$.

% chapter 8_wrze_nia_2011_textit (end)

\chapter{9 września 2011 -- \textit{Ogólna teoria względności}} % (fold)
\label{cha:9_wrze_nia_2011_textit}
	\section{Podejście ogólnej teorii względności do grawitacji} % (fold)
	\label{sec:podej_cie_og_lnej_teorii_wzgl_dno_ci_do_grawitacji}
		Według ogólnej teorii względności grawitacja wywoływana jest przez rozkład masy, co wywołuje zakrzywienie czasoprzestrzeni. Masy zakrzywiają czasoprzestrzeń, powodują, że poprzez nią mamy oddziaływanie grawitacyjne.
	% section podej_cie_og_lnej_teorii_wzgl_dno_ci_do_grawitacji (end)
	
	\section{Transformacja Galileusza} % (fold)
	\label{sec:transformacja_galileusza}
		Jest to transformacja z układu stacjonarnego, do układu poruszającego się ze stałą prędkością. Jeżeli mamy jakiś punkt w układzie poruszającym się o współrzędnych $(x', y')$, to:
		\[
			\left\{\begin{array}{ccc}
				x' & = & x-vt\\
				y' & = & y\\
				t' & = & t
			\end{array}\right.
		\]
		Transformacja przestaje działać do zjawisk elektromagnetycznych.
	% section transformacja_galileusza (end)

	\section{Transformacja Lorenza} % (fold)
	\label{sec:transformacja_lorenza}
		\subsection{Postulaty (Einsteina), o które się oparła} % (fold)
		\label{sub:postulaty_o_kt_re_si_opar_a}
			\begin{enumerate}
				\item Prędkość światła jest jednakowa we wszystkich inercjalnych układach odniesienia.
				\item Prawa fizyki są jednakowe we wszystkich układach inercjalnych.
			\end{enumerate}
		% subsection postulaty_o_kt_re_si_opar_a (end)
		\subsection{Współrzędne punktów w transformacji} % (fold)
		\label{sub:wsp_rz_dne_punkt_w_w_transformacji}
			\[
				\left\{\begin{array}{ccc}
					x' & = & (x-vt)\gamma\\
					y' & = & y\\
					t' & = & \gamma(t-\frac{v}c \cdot \frac{x}c)
				\end{array}\right.
			\]
			\[
				\gamma = \frac{1}{\sqrt{1-\frac{v^2}{c^2}}}
			\]
		% subsection wsp_rz_dne_punkt_w_w_transformacji (end)
		\subsection{Efekty} % (fold)
		\label{sub:efekty}
			\subsubsection{Dylatacja czasu} % (fold)
			\label{ssub:dylatacja_czasu}
				Czas płynie wolniej w układach poruszających się. Tam zmierzymy mniejszy odstęp czasu.
				\[
					\Delta t = \gamma\Delta t'
				\]
				Najkrótszy odstęp czasu zmierzymy zawsze tam, gdzie następuje zjawisko.
			% subsubsection dylatacja_czasu (end)
			\subsubsection{Skrócenie Lorenza} % (fold)
			\label{ssub:skr_cenie_lorenza}
				\[
					\Delta l = \frac{\Delta l'}\gamma
				\]
				Jeżeli ciało się porusza wzdłuż osi $OX$, to wymiar ten ulega skróceniu.

				Jeśli mamy gościa na rakiecie, poruszającej się z prędkością $v$. Dla zewnętrznego obserwatora rakieta jest krótsza, niż dla gościa na rakiecie.
			% subsubsection skr_cenie_lorenza (end)

			\subsubsection{Względność równoczesności} % (fold)
			\label{ssub:wzgl_dno_r_wnoczesno_ci}
				Dwa zdarzenia równoczesne w układzie stacjonarnym nie są równoczesne w układzie poruszającym się.

				Popatrzmy na pociąg ,,Einsteina''. Mamy dwóch obserwatorów, w środku i na zewnątrz. Nagle w oba końce pociągu uderzają pioruny. Dla obserwatora wewnątrz zdarzenia stały się jednocześnie, natomiast dla zewnętrznego pierwszy piorun uderzył piorun z przodu.
			% subsubsection wzgl_dno_r_wnoczesno_ci (end)

			\subsubsection{Przyrost masy} % (fold)
			\label{ssub:przyrost_masy}
				Jeżeli ciało się porusza, to jego masa wzrasta.
				\[
					m = m_0\gamma
				\]
			% subsubsection przyrost_masy (end)

			\subsubsection{Relatywistyczny wzór na energię kinetyczną} % (fold)
			\label{ssub:relatywistyczny_wz_r_na_energi_kinetyczn_}
				Energia kinetyczna cząstki to:
				\[
					E_K = E-E_0 = mc^2 - m_0c^2 = m_0\gamma c^2 - m_0 c^2 = m_0 c^2(\gamma-1) = E_0(\gamma-1)
				\]
				\[
					E_K = E_0(\gamma-1)
				\]
			% subsubsection relatywistyczny_wz_r_na_energi_kinetyczn_ (end)
		% subsection efekty (end)
	% section transformacja (end)
% chapter 9_wrze_nia_2011_textit (end)

\chapter{13 września 2011 -- \textit{Efekty relatywistyczne}} % (fold)
\label{cha:13_wrze_nia_2011_textit_efekty_relatywistyczne}
	\section{Rozwiązywanie zadań} % (fold)
	\label{sec:rozwi_zywanie_zada_}
		\subsection{Zadanie 1} % (fold)
		\label{sub:zadanie_1}
			\paragraph{Treść}
			Rakieta porusza się z szybkością $0.8c$. Obserwator na Ziemi zmierzył czas trwania pewnego zjawiska i otrzymał wynik $800~s$.
				\subparagraph{a)}Jak długo trwało to zjawisko według pilota rakiety?
				\subparagraph{b)}Pilot zmierzył długość swojej rakiety i uzyskał wynik $120~m$. Oblicz jaki wynik tego samego pomiaru uzyska obserwator na Ziemi.
			\paragraph{Rozwiązanie}
			Oznaczmy rakietę przez literę $B$, a Ziemię jako $A$. Wiemy, że:
			\begin{eqnarray*}
				v_B &=& 0.8c\\
				t_A & = & 800~s = \Delta t'\\
				l_B & = & 120~m\\
				\gamma &=& \frac{1}{\sqrt{1-\frac{v^2}{c^2}}}\\
				&=&\frac{1}{\sqrt{1-\frac{0.64c^2}{c^2}}}\\
				&=&\frac{1}{\sqrt{0.36}}\\
				&=&\frac{1}{0.6}\\
				&=&\frac{10}{6}\\
				\Delta t &=& \gamma\cdot800~s\\
				& = & \frac{10}{6}\cdot800~s\\
				&=& 1333.3~s\\
				\Delta l &= & \frac{120~m}{\frac{10}6}\\
				&=& \frac{920}{10}\\
				&=&72~m
			\end{eqnarray*}
		% subsection zadanie_1 (end)
		\subsection{Zadanie 2} % (fold)
		\label{sub:zadanie_2}
			\paragraph{Treść}
			Miony utworzone w górnych warstwach atmosfery przebywają do chwili rozpadu odległość $5~km$ z prędkością $0.99c$.
				\subparagraph{a)}Jak długi jest czas życia mionu mierzony przez nas, a jaki czas życia mierzony w jego własnym układzie odniesienia?
				\subparagraph{b)}Jaka jest grubość atmosfery przebyta przez mion, mierzona w jego własnym układzie odniesienia.
			\paragraph{Rozwiązanie}
			\begin{eqnarray*}
				s &=& 5~km\\
				v_m & = & 0.99~c\\
				\Delta t &=& \frac{s}{v}\\
				&=& \frac{5000~km}{0.99\cdot3\cdot10^8~\frac{m}{s}}\\
				&=& \frac{1}{59400}~s\\
				\Delta t' & = & \Delta t\cdot\sqrt{1-\frac{v^2}{c^2}}\\
				&=& 1.68\cdot10^{-5}~s\cdot\sqrt{1-\frac{(0.99c)^2}{c^2}}\\
				& = & 1.68\cdot10^{-5}~s\cdot0.14\\
				&=& 2.37\cdot10^{-6}~s\\
				\Delta l' &=& v\cdot\Delta t'\\
				&=& 2.37\cdot0.99\cdot3\cdot10^{12}\\
				&=& 703.69~m
			\end{eqnarray*}
		% subsection zadanie_2\\
\section{Relatywistyczne składanie prędkości} % (fold)
\label{sec:relatywistyczne_sk_adanie_pr_dko_ci}
	\[
		v_{wzgl} = \frac{v_1\mathbf{\pm}v_2}{1\mathbf{\pm}\frac{v_1\cdot v_2}{c^2}}
	\]
	Jeżeli cząstki poruszają się w różnych kierunkach mamy plus w pogrubionych miejscach, jeśli w tym samym, to minus.
% section relatywistyczne_sk_adanie_pr_dko_ci (end)
\section{Rozwiązywanie zadań -- cd.} % (fold)
\label{sec:rozwi_zywanie_zada_cd_}

% section rozwi_zywanie_zada_cd_ (end)
		\subsection{Zadanie 3} % (fold)
		\label{sub:zadanie_3}
			\paragraph{Treść}
			Dwa akceleratory dają cząstki poruszające się w przeciwne strony. Z prędkością $v_1 = v_2 = 0.9~c$. Oblicz względną prędkość cząstek.
			\paragraph{Rozwiązanie}
			\begin{eqnarray*}
				v & = & \frac{v_1\mathbf{\pm}v_2}{1\mathbf{\pm}\frac{v_1\cdot v_2}{c^2}}\\
				&=& \frac{0.9~c+0.9~c}{1+\frac{0.81~c^2}{c^2}}\\
				&=& \frac{1.8~c}{1.81}\\
				 &=& 0.99447~c
			\end{eqnarray*}
		% subsection zadanie_3 (end)
	% section rozwi_zywanie_zada_ (end)
		
		
% chapter 13_wrze_nia_2011_textit_efekty_relatywistyczne (end)

\chapter{15 września 2011 -- \textit{Relatywistyka: rozwiązywanie zadań}} % (fold)
\label{cha:15_wrze_nia_2011_textit}
	\section{Zadanie 1} % (fold)
	\label{sec:zadanie_1}
		\paragraph{Treść}Energia kinetyczna pewnej niestabilnej cząstki jest równa $70~MeV = 70\cdot1.6\cdot10^{-19}~J$. Ile razy czas jej połowicznego zaniku jest większy od czasu połowicznego zaniku w spoczynku, jeżeli jej masa spoczynkowa wynosi $0.3~u = 0.3\cdot1.66\cdot10^{-27}~kg$.
		\paragraph{Rozwiązanie}
		\begin{eqnarray*}
			E_k &=& m_0c^2(\gamma-1)
		\end{eqnarray*}
		\[
			\begin{array}{ccc}
				1~u & - & 930~MeV\\
				0.3 & - & x
			\end{array}
		\]
		\begin{eqnarray*}
			(\gamma-1) &=& \frac{E_K}{E_0}\\
			(\gamma-1) & = & \frac{70}{279}\\
			\gamma & = & \frac{349}{279}\\
			\gamma & = & 1.25
		\end{eqnarray*}
	% section zadanie_1 (end)
	\section{Zadanie 2} % (fold)
	\label{sec:zadanie_2}
		\paragraph{Treść}Elektrony są przyspieszane do energii $1~MeV$ w akceleratorze liniowym.
		\subparagraph{a)}Z jaką prędkością się poruszają?
		\subparagraph{b)}O ile krótszy jest dla nich każdy metr rury akceleratora?
		\subparagraph{c)}O ile procent zmieni się masa elektronu, poruszającego się z tą prędkością.
		\paragraph{Rozwiązanie}
		\subparagraph{a)}
		\begin{eqnarray*}
			E_K & = & E_0(\gamma-1)\\
			\gamma & = & \frac{1~MeV}{0.511~MeV}+1\\
			\gamma & = & 2.957\\
			\frac{1}{\sqrt{1-\frac{v^2}{c^2}}} & = & 2.95\\
			\sqrt{1-\frac{v^2}{c^2}} & = & 0.338\\
			1-\frac{v^2}{c^2} &=& 0.114\\
			\frac{v^2}{c^2} & = & 0.886\\
			v^2 & = & 0.886\cdot c^2\\
			v & = & 0.941\cdot c
		\end{eqnarray*}
		\subparagraph{b)}
		\begin{eqnarray*}
			\Delta l &=& \frac{\Delta l'}{\gamma}\\
			\Delta l & = & \frac{1~m}{2.957}\\
			\Delta l & = & 0.33~m\\
			1~m-0.33~m & = & 0.66~m
		\end{eqnarray*}
		\subparagraph{c)}
			\begin{eqnarray*}
				m & = & m_0\gamma\\
				\frac{m}{m_0} &=& \gamma\\
				\%m &=& 295.7\%
			\end{eqnarray*}
	% section zadanie_2 (end)
	\section{Zadanie 3} % (fold)
	\label{sec:zadanie_3}
		\paragraph{Treść} Objętość sześcianu spoczywającego w układzie laboratoryjnym wynosi $1~m^3$. Ile będzie wynosiła objętość tego sześcianu, gdy będzie się on poruszał z prędkością $0.5~c$ wzdłuż osi $OX$.
		\paragraph{Rozwiązanie}
		\begin{eqnarray*}
			\gamma & = & \frac{1}{\sqrt{1-\frac{v^2}{c^2}}}\\
			\gamma & = & \frac{1}{\sqrt{1-\frac{(0.5~c)^2}{c^2}}}\\
			\gamma & = & \frac{1}{\sqrt{1-0.25}}\\
			\gamma & = & \frac{1}{\sqrt{0.75}}\\
			\gamma & = & \frac{1}{0.866}\\
			\gamma & = & 1.15\\ \\
			\Delta V & = & \frac{\Delta V'}{\gamma}\\
			\Delta V & = & \frac{1~m^3}{\gamma}\\
			\Delta V & = & \frac{1~m^3}{1.15}\\
			\Delta V & = & 0.866~m^3
		\end{eqnarray*}
		\begin{quote}
			\bf Uwaga, tutaj skorzystaliśmy z tego, że objętość bryły skaluje się tak jak odległość.
		\end{quote}
	% section zadanie_3 (end)
	
	\section{Zadanie 4} % (fold)
	\label{sec:zadanie_4}
		\paragraph{Treść}Średni czas życia spoczywającej cząstki wynosi $10^{-12}~s$. Oblicz szybkość, z którą musi poruszać się cząstka, by pozostawić w emulsji fotograficznej ślad o długości $1~cm$.
		\paragraph{Rozwiązanie}
		\begin{eqnarray*}
			\Delta t' &=& 10^{-12}~s\\
			\Delta l & = & 0.01~m\\
			v &=& \frac{s}{\Delta t'}\\
			\Delta t &=& \gamma\Delta t'\\
			\Delta t &=& \frac{1}{\sqrt{1-\frac{v^2}{c^2}}}\cdot\Delta t'\\
			v &=& \frac{s\sqrt{1-\frac{v^2}{c^2}}}{\Delta t'}\\
			\frac{v\Delta t'}{s} &=& \sqrt{1-\frac{v^2}{c^2}}\\
			\frac{v^2\Delta t'^2}{s^2} &=& 1-\frac{v^2}{c^2}\\
			v^2(\frac{\Delta l'^2}{s^2} + \frac{1}{c^2}) &=& 1\\
			v &=& \frac{1}{\sqrt{10^{-20}+\frac19\cdot10^{-16}}}\\
			v & = & 2.9995\cdot10^8\frac{m}{s}
		\end{eqnarray*}
	% section zadanie_4 (end)

	\section{Teoria} % (fold)
	\label{sec:teoria}
		W trójosiowym układzie współrzędnych odległość między punktami $A$ i $B$.
		\[
			l = \sqrt{(x_B-x_A)^2 + (y_B-y_A)^2 + (z_B-z_A)^2}
		\]
		W czterowymiarowym układzie współrzędnych odległość to:
		\[
			l = c^2\cdot t^2 - \Delta x^2 - \Delta y^2 - \Delta z^2
		\]
		\subsection{Stałość interwału czasoprzestrzennego} % (fold)
		\label{sub:sta_o_interwa_u_czasoprzestrzennego}
			\[
				c^2\cdot\Delta t'^2 - \Delta x'^2 - \Delta y'^2 - \Delta z'^2 = c^2\cdot \Delta t^2 - \Delta x^2 - \Delta y^2 - \Delta z^2
			\]
		% subsection sta_o_interwa_u_czasoprzestrzennego (end)
	% section teoria (end)

		\section{Zadanie 5} % (fold)
	\label{sec:zadanie_5}
		\paragraph{Treść} W tym samym miejscu korony słonecznej w odstępie $12~s$ nastąpiły dwa wybuchy. UFO poruszające się ze stałą prędkością względem słońca zarejestrowało te zdarzenia w odstępie $13~s$.
			\subparagraph{a)}Ile wynosi odległość przestrzenna $\Delta l$ między wybuchami w układzie związanym z poruszającą się rakietą?
			\subparagraph{b)}Jaką wartość i jaki kierunek ma wektor prędkości rakiety.
		\paragraph{Rozwiązanie}
		\subparagraph{a)}
		\[
			c^2\Delta t'^2 - \Delta x'^2 = c^2\Delta t^2 - \Delta x^2
		\]
		\[
			c^2\cdot\Delta t'^2 -\Delta x'^2 = c^2\cdot \Delta t^2 - \Delta x^2
		\]
		\[
			c^2(\Delta t^2 - \Delta t'^2) = \Delta x^2 - \Delta x'^2 = l^2
		\]
		\[
			c^2(13^2-12^2) = l^2
		\]
		\[
			c^2(169-144) = l^2
		\]
		\[
			l = c\sqrt{25}
		\]
		\[
			l = 5c = 5\cdot3\cdot10^{8} = 1.5\cdot10^9~m
		\]
		\subparagraph{b)}
		\[
			\Delta l = \frac{\Delta l'}{\gamma}
		\]
		\[
			13 = \frac{12}{\gamma}
		\]
		\[
			\gamma = \frac{12}{13}
		\]
		\[
			\frac{1}{\sqrt{1-\frac{v^2}{c^2}}} = \frac{12}{13}
		\]
		\[
			\sqrt{1-\frac{v^2}{c^2}} = \frac{12}{13}
		\]
		\[
			1-\frac{v^2}{c^2} = \frac{149}{169}
		\]
		\[
			v = c\sqrt{1-\frac{144}{169}} = 0.385~c
		\]
	% section zadanie_5 (end)
	\section{Zadanie 6 (domowe)} % (fold)
	\label{sec:zadanie_6_}
		\paragraph{Treść}Dwie cząstki o jednakowych prędkościach $0.75~c$ poruszają się po jednej prostej i padają na tarcze. Jedna z nich uderzyła w tarczę o $\Delta t = 10^{-8}~s$ później niż druga. Obliczyć odległość między cząstkami w locie w układzie odniesienia związanym z nimi.
		\paragraph{Rozwiązanie} 
		\begin{eqnarray*}
			\Delta s^2 & = & c^2\Delta t_1^2 - \Delta x_1^2\\
			\Delta s^2 & = & c*2\Delta t_2^2 - \Delta x_2^2\\
			\Delta x_1^2 & = & 0\\
			\Delta t_2^2 & = & 0\\
			c\Delta t_1 &=& \Delta x_2\\
			3\cdot10^8\frac{m}s \cdot 10^{-8}~s & = & \Delta x_2\\
			\Delta x_2 & = & 3~m
		\end{eqnarray*}
		\begin{flushright}
			Podziękowania dla Pana Dyrka.
		\end{flushright}
	% section zadanie_6_ (end)
% chapter 15_wrze_nia_2011_textit (end)

\chapter{16 września 2011 -- \textit{Prawo Archimedesa}} % (fold)
\label{cha:16_wrze_nia_2011_textit}
	\section{Prawo Archimedesa} % (fold)
	\label{sec:prawo_archimedesa}
		\paragraph{Treść} Na wszystkie ciała zanurzone w cieczy bądź gazie działa siła wyporu skierowana pionowo do góry i równa co do wartości ciężarowi wypartej cieczy (wypartego gazu).\\
		$v_c$ -- objętość ciała zanurzonego
		\[
			F_W = m_c \cdot g = V_Z \cdot \varrho _c \cdot g
		\]
	% section prawo_archimedesa (end)
% chapter 16_wrze_nia_2011_textit (end)

\chapter{20 września 2011 -- \textit{Prawo Archimedesa w zadaniach}} % (fold)
\label{cha:20_wrze_nia_2011_textit}
	\section{Zadanie 1} % (fold)
	\label{sec:zadanie_1}
		\paragraph{Treść}W wodzie o gęstości $1000\frac{kg}{m^3}$ pływa korek o gęstości $700\frac{kg}{m^3}$. Oblicz stosunek części zanurzonej do części wynurzonej.
		\paragraph{Rozwiązanie}
		\begin{eqnarray*}
			\varrho_k &=& 700\frac{kg}{m^3}\\
			\varrho_w &=& 1000\frac{kg}{m^3}\\
			\frac{V_z}{V_w} & = & ?\\
			g\cdot V_z \cdot \varrho_w &=& m\cdot g\\
			V_z\cdot \varrho_w &=& \varrho_s \cdot V_{ck}\\
			V_{ck} &=& V_z+V_w\\
			V_z\cdot \varrho_w &=& \varrho_k \cdot V_z + \varrho_k\cdot V_w\\
			V_z(\varrho_w-\varrho_k) &=& V_w\cdot\varrho_k\\
			\frac{V_z}{V_w} &=& \frac{\varrho_k}{\varrho_w-\varrho_k}\\
			\frac{700\frac{kg}{m^3}}{300\frac{kg}{m^3}} &=& \frac73
		\end{eqnarray*}
	% section zadanie_1 (end)
	\section{Zadanie 2} % (fold)
	\label{sec:zadanie_2}
		\paragraph{Treść}Ciało w wodzie waży $Q_1$, a w nafcie $Q_2$. Gęstość wody $\varrho_W$, a gęstość nafty $\varrho_N$. Oblicz objętość ciała.
		\paragraph{Rozwiązanie}
		\[
			\left\{\begin{array}{ccc}
				Q_1 &=& Q-F_{w1}\\
				Q_2 &=& Q-F_{w2}
			\end{array}\right.
		\]
		\begin{eqnarray*}
			Q_1-Q_2 &=& F_{w2} - F_{w1}\\
			Q_1-Q_2 &=& V_c\cdot\varrho_n\cdot g-V_c\cdot\varrho_w\cdot g\\
			Q_1-Q_2 &=& V_c\cdot g(\varrho_n-\varrho_w)\\
			V_c &=& \frac{Q_1-Q_2}{g\cdot(\varrho_n-\varrho_w)}\\
		\end{eqnarray*}
	% section zadanie_2 (end)
	\section{Zadanie 3} % (fold)
	\label{sec:zadanie_3}
		\paragraph{Treść}Ciężar ciała w powietrzu wynosi $17N$, a w oleju $15N$. Oblicz gęstość materiału, z którego wykonano ciało $800\frac{kg}{m^3}$.
		\paragraph{Rozwiązanie}
		\begin{eqnarray*}
			\varrho_m &=& ?\\
			\varrho_o &=& 800\frac{kg}{m^3}\\
			Q &=& 17N\\
			Q_1 &=& 15N\\
			Q-Q_1 &=& F_w\\
			F_w &=& V_c\cdot g\cdot \varrho_o\\
			2N &=& V_c\cdot g\cdot \varrho_o\\
			V_c &=& \frac{2N}{g\cdot \varrho_o}\\
			V_c &=& \frac{2kg\frac{m}{s^2}}{10\frac{m}{s^2}\cdot800\frac{kg}{m^3}}\\
			V_c &=& \frac1{1000}m^3\\
			\varrho &=& \frac{m}{V_c}\\
			\varrho &=& 1.7\cdot4000\frac{kg}{m^3}\\
			\varrho &=& 6800 \frac{kg}{m^3}
		\end{eqnarray*}
	% section zadanie_3 (end)
	\section{Zadanie 4 (domowe)} % (fold)
	\label{sec:zadanie_4}
		\paragraph{Treść}Sześcian o krawędzi $0.2m$ wykonany z drewna o gęstości $\varrho_d = 600\frac{kg}{m^3}$ zanurzono w wodzie. Pod sześcianem do dolnej ściany przymocowano stalowy ciężarek ($\varrho_s$) o takiej masie, że górna ściana sześcianu znajduje się na wysokości powierzchni wody. Jaka była masa ciężarka.
	% section zadanie_4 (end)
	\section{Zadanie 5 (domowe)} % (fold)
	\label{sec:zadanie_5_}
		10 z moodle'a.
	% section zadanie_5_ (end)
% chapter 20_wrze_nia_2011_textit (end)
\chapter{22 września 2011 -- \textit{Paradoks hydrostatyczny, naczynia połączone}} % (fold)
\label{cha:22_wrze_nia_2011_textit}
	\section{Ciśnienie} % (fold)
	\label{sec:ci_nienie}
		\[
			p = \frac{F}{s} \left[\frac{N}{m^2} = Pa\right]
		\]
		\[
			1atm = 1.013\cdot 10^5Pa
		\]
		\[
			1at = 10^5Pa = 1bar
		\]
		\[
			1mmHg = 1 Tr \mathrm{(tor)} = 133 Pa
		\]
		\[
			p = \frac{F_g}{S} = \frac{m\cdot g}{S} = \frac{S\cdot h\cdot \varrho_w\cdot g}{S} = \varrho_w\cdot g\cdot h
		\]
	% section ci_nienie (end)
	\section{Paradoks hydrostatyczny} % (fold)
	\label{sec:paradoks_hydrostatyczny}
		Mamy cebrzyk z wodą oraz cienką rurkę, napełnione do tego samego poziomu wodą. Z powyższego wzoru wynika, że ciśnienie działające na spód cebrzyka i na rurkę są takie same.

		Jeśli mamy trzy naczynia połączone o różnych kształtach (połączone na samym dole), to wlewając wodę otrzymamy w każdym naczyniu taką samą wysokość słupa cieczy.

		Do znajdywania gęstości nieznanych cieczy używa się rurek w kształcie litery U. Najpierw wlewamy tam wodę. Teraz wlewamy nieznaną ciecz, niech to będzie olej. Wtedy przecinamy linią rurkę na tej wysokości, na której stykają się dwie ciecze. Z tego mamy, że
		\begin{eqnarray*}
			\varrho_1\cdot g\cdot h_1 &=& \varrho_w\cdot g\cdot h_2\\
			\varrho_1 &=& \varrho_w\cdot\frac{h_2}{h_1}
		\end{eqnarray*}
		\subsection{Zadanie domowe} % (fold)
		\label{sub:zadanie_domowe}
			\[
				h_1 = 13.5cm
			\]
			\[
				h_2 = 12cm
			\]
			Policz gęstość oleju.
			\paragraph{Odpowiedź} Gęstość oleju wynosi $888\frac{kg}{m^3}$.
		% subsection zadanie_domowe (end)
	% section paradoks_hydrostatyczny (end)
% chapter 22_wrze_nia_2011_textit (end)

\chapter{29 września 2011 -- \textit{Naczynia połączone - zadania}} % (fold)
\label{cha:29_wrze_nia_2011_textit}
	\section{Zadania} % (fold)
	\label{sec:zadania}
	\subsection{Zadanie 1} % (fold)
	\label{sub:zadanie_1}
		\paragraph{Treść}Do naczynia połączonego w kształcie litery U nalano wody i nafty. Suma długości obu słupów wody i nafty wynosi $h = h_1+h_2 = 0.9m$. Jaka jest wysokość słupów poszczególnych cieczy. Gęstość nafty wynosi $800\frac{kg}{m^3}$.
	% subsection zadanie_1 (end)
	\subsection{Zadanie 2} % (fold)
	\label{sub:zadanie_2}
		Mamy rurkę o przekroju $S = 12cm^2 = 12\cdot10^{-4}m^2$, do którego wlano rtęć o gęstości $\varrho_{Hg} = 13600\frac{kg}{m^3}$. Gdzieśtam jeszcze jest jakiś klocek, woda\dots
	% subsection zadanie_2 (end)
	\subsection{Zadanie 3} % (fold)
	\label{sub:zadanie_3}
		\paragraph{Treść} Do wiadra w kształcie cylindra o średnicy $30cm$ wlano $15.7l$ wody. Jakie jest ciśnienie hydrostatyczne na wysokości $5cm$ od dna wiadra. Gęstość wody $1000\frac{kg}{m^3}$.
	% subsection zadanie_3 (end)
	\subsection{Zadanie domowe} % (fold)
	\label{sub:zadanie_domowe}
		\paragraph{Treść} Do rurki w kształcie litery U nalano rtęci, następnie do lewego ramienia dolano pewną ilość wody. Stwierdzono, że dolny poziom rtęci znajdował się na wysokości $h_1 = 38.5 cm$, górny na wysokości $h_2 = 41.6 cm$, górny poziom wody na wysokości $h_3 = 80.7 cm$. Jaki jest ciężar właściwy rtęci?
		\paragraph{Rozwiązanie} Nadpiszmy sobie zmienne:
		\begin{eqnarray*}
			h_1 &=& 41.6-38.5 = 3.1\\
			h_2 &=& 80.7 - 38.5 = 42.2\\
			\varrho_w &=& 1000\frac{kg}{m^3}\\
			\varrho_1\cdot h_1 &=& \varrho_2\cdot h_2\\
			3.1\cdot \varrho_{Hg} &=& 1000\cdot 42.2\\
			3.1\cdot\varrho_{Hg} &=& 42200\\
			\varrho_{Hg} &=& \frac{42200}{3.1}\\
			\varrho_{Hg} &=& 13612\frac{kg}{m^3}\\
			\gamma_{Hg} &=& \varrho_{Hg}\cdot g\\
			\gamma_{Hg} &=& 13612\cdot10\\
			\gamma_{Hg} &=& 136120\frac{N}{m^3}
		\end{eqnarray*}
	% subsection zadanie_domowe (end)
	% section zadania (end)
% chapter 29_wrze_nia_2011_textit (end)
\chapter{30 września 2011 -- \textit{Prawo Pascala}} % (fold)
\label{cha:30_wrze_nia_2011_textit}
	\section*{Zadanie} % (fold)
	\label{sec:zadanie}
		\paragraph{Treść}Z jakiej maksymalnej głębokości studni można wyciągnąć wodę, używając pompy ssąco-tłoczącej?
		\paragraph{Rozwiązanie} \[
			h = \frac{10^5Pa}{10^4} = 10m
		\]
	% subsection zadanie (end)
	\section{Prawo Pascala} % (fold)
	\label{sec:prawo_pascala}
		\bf \begin{center}
			Ciśnienie cieczy lub w gazie rozchodzi się jednakowo we wszystkich kierunkach.
		\end{center} \rm
		\paragraph{Kula Pascala} Mamy strzykawkę, zakończoną kulą. Znajdują się w niej otwory, przez które może przepływać woda. Jeśli nalejemy wody i wytworzymy ciśnienie w strzykawce, rozchodzi się ono jednakowo we wszystkich kierunkach, a z dziurek wypływa jednakowo woda.
		\paragraph{Nurek Kartezjusza} W cylindrze mamy zamknięte naczynie. Cylinder jest wypełniony wodą, a naczynie ma dziurę na dole. Cylinder jest przesłonięty błoną, na którą możemy naciskać. Wytwarzając dodatkowo ciśnienie na membranę sprawiamy, że nurek jest jednakowo ze wszystkich stron mocniej naciskany, przez co woda jest wpychana do nurka.

		Wykorzystujemy to prawo na przykład w hamulcach hydraulicznych lub prasie hydraulicznej. Wywierając ciśnienie na jednej z powierzchni tłoka $p = \frac{F_1}{S_1}$ wytwarzamy ciśnienie na drugiej, większej powierzchni o większej sile $p = \frac{F_2}{S_2}$.
		\[
			\frac{F_1}{S_1} = \frac{F_2}{S_2}
		\]
		Prasa hydrauliczna to \textbf{maszyna prosta}. Maszyna prosta charakteryzuje się tym, że nie da się w niej zaoszczędzić energii.
		\subsection{Zadanie 1} % (fold)
		\label{sub:zadanie}
			\paragraph{Treść}
			W prasie hydraulicznej średnica tłoka wynosi $1.6cm$, średnica prasy $32cm$. Ramię siły $60cm$, ramię tłoka $10
			cm$. Jaka jest siła wywierana na prasę, jeżeli robotnik działa siłą $12kG$. ($1kG = 10N$)
			\paragraph{Rozwiązanie} 
			\begin{eqnarray*}
				F_1\cdot 10cm &=& F_2\cdot 60cm\\
				F_1 &=& F_2\cdot6\\
				F_1 &=& 12\cdot10N\cdot 6\\
				F_1 &=& 
			\end{eqnarray*}
		% subsection zadanie (end)
	% section prawo_pascala (end)
% chapter 30_wrze_nia_2011_textit (end)
\chapter{4 października 2011 -- \textit{}} % (fold)
\label{cha:4_pa_dziernika_2011_textit}

% chapter 4_pa_dziernika_2011_textit (end)
\end{document}
