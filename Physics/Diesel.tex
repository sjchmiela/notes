\documentclass [a4paper, 12pt]{article}
\usepackage{fullpage}
\usepackage[utf8]{inputenc}
\usepackage{polski}
\usepackage{hyperref}
\usepackage[usenames,dvipsnames]{color}
\hypersetup{
    bookmarks=true,         % show bookmarks bar?
    unicode=false,          % non-Latin characters in Acrobat’s bookmarks
    pdftoolbar=true,        % show Acrobat’s toolbar?
    pdfmenubar=true,        % show Acrobat’s menu?
    pdffitwindow=false,     % window fit to page when opened
    pdfstartview={FitH},    % fits the width of the page to the window
    pdftitle={Fizyka},    % title
    pdfauthor={Stanisław Chmiela},     % author
    pdfsubject={Sprawność silnika Diesla},   % subject of the document
    pdfcreator={Stanisław Chmiela},   % creator of the document
    pdfproducer={Stanisław Chmiela}, % producer of the document
    pdfkeywords={fizyka} {diesel}, % list of keywords
    pdfnewwindow=false,      % links in new window
    colorlinks=true,       % false: boxed links; true: colored links
    linkcolor=BrickRed,          % color of internal links
    citecolor=PineGreen,        % color of links to bibliography
    filecolor=RawSienna,      % color of file links
    urlcolor=MidnightBlue      % color of external links
}
\usepackage[pdftex]{graphicx}
\usepackage{wrapfig}
\usepackage{float}
\usepackage{amsmath}
\linespread{1.3}
\author{Stanisław Chmiela}
\title{Sprawność silnika Diesla}
\begin{document}
\maketitle
\section{Rysunek}
\begin{figure}[h]
\centering
\includegraphics[width=10cm]{CyklDiesla_pV}
\end{figure}
\section{Wyprowadzenie}
Na odcinku $(1)$ -- izobarycznym -- gaz pobiera ciepło:
\[
    Q_3 = nC_p \Delta T = nC_p (T_3-T_2)
\]
Odcinek $(3)$, odcinku izochorycznym, gaz oddaje ciepło:
\[
    Q_4 = \Delta U_{(2,4)} = nC_v \Delta T = nC_v(T_1-T_4)
\]
Jako że $T_1 < T_4$, powinniśmy ustalić, że:
\[
    |Q_4| = nC_v(T_4-T_1)
\]
Sprawność obliczymy sobie ze wzoru
\[
    \eta = 1-\frac{|Q_4|}{Q_3} = 1-\frac{nC_v(T_4-T_1)}{nC_p(T_3-T_2)} = 1-\frac{C_v(T_4-T_1)}{C_p(T_3-T_2)}
\]
Dalej:
\[
    \eta = 1-\frac{C_v(\frac{p_2}{p_3} - \frac{p_1}{p_3})}{C_p(\frac{V_2}{V_3} - \frac{V_1}{V_3})} =1-{\frac{C_v}{C_p}}{{\left({\frac{V_2}{V_3}}\right)^{\kappa}-\left({\frac{V_1}{V_3}}\right)^{\kappa}}\over{{{V_2}\over{V_3}}-{{V_1}\over{V_3}}}}=1-{\frac{C_v}{C_p}}\left({\frac{V_2}{V_3}}\right)^{\kappa-1}{{1-\left({{V_1}\over{V_2}}\right)^{\kappa}}\over{1-{\frac{V_1}{V_2}}}}
\]
\section{Bibliografia}
\begin{enumerate}
    \item \url{http://www.ftj.agh.edu.pl/\~wolny/Wcbcd86b7b34db.htm}
    \item \url{http://www.if.pw.edu.pl/\~bibliot/archiwum/adamczyk/WykLadyFO/FoWWW\_23.html}
    \item \url{http://osilek.mimuw.edu.pl/index.php?title\=PF\_Modu\%C5\%82\_8}
    \item \url{http://pl.wikibooks.org/wiki/Fizyka\_statystyczna/Cykle\_(obiegi)\_termodynamiczne\#Sprawno.C5.9B.C4.87\_cyklu\_Diesla}
\end{enumerate}
\end{document}
