\documentclass [a4paper, 11pt, oneside]{book}
\usepackage{fullpage}
\usepackage[utf8]{inputenc}
\usepackage{polski}
\usepackage{hyperref}
\usepackage[usenames,dvipsnames]{color}

\hypersetup{
    bookmarks=true,
    unicode=false,
    pdftoolbar=true,
    pdfmenubar=true,
    pdffitwindow=false,
    pdfstartview={FitH},
    pdftitle={Historia},
    pdfauthor={Stanisław Chmiela},
    pdfsubject={Notatki - historia},
    pdfcreator={Stanisław Chmiela},
    pdfproducer={Stanisław Chmiela},
    pdfkeywords={historia} {notatki},
    pdfnewwindow=false,
    colorlinks=true,
    linkcolor=BrickRed,
    citecolor=PineGreen,
    filecolor=RawSienna,
    urlcolor=MidnightBlue
}
\usepackage[pdftex]{graphicx}
\usepackage{wrapfig}
\usepackage{float}
\usepackage{amsmath}
\linespread{1.3}
\author{Stanisław Chmiela}
\title{Historia}
\begin{document}
\maketitle
\tableofcontents
\part{Semestr I}
\chapter{5 września 2011 -- \textit{Lekcja organizacyjna}}
\section{Podręcznik}
    Brązowy, część III, wydawnictwo Nowa Era, Robert Śniegocki, zakres rozszerzony. Potem kolejne tomy z Nowej Ery.
\section{Edysk}
    Notatki z lekcji znajdują się na ,,edysku''.
    \begin{description}
    \item[Adres] \href{http://www.edysk.pl/log.html}{http://www.edysk.pl/}
    \item[Nick] \texttt{matbiolfiz}
    \item[Hasło] \texttt{*************} (\texttt{tekstynalekcje}).
    \item[Przeglądarki] Nie działa u mnie w Chromie, za to w Firefoksie już tak.
    \end{description}
\section{Oceny}
    Oceny semestralną i roczną wylicza się ze średniej ważonej.
\section{Ocenianie}
    \subsection{Sprawdziany}
        Gdy się było nieobecnym na sprawdzianie, na pierwszej lekcji od razu pisze się sprawdzian. Będą one w formie testu. 2 tygodnie czasu na poprawę. Oceny ze sprawdzianów ,,ważą'' 10.
    \subsection{Odpowiedź ustna}
        Raczej maksimum dwie oceny w ciągu roku. Nie podlegają poprawie, ich waga to 8.
    \subsection{Kartkówki}
        Rzadko, niezapowiedziana, z 3 ostatnich lekcji. 8.
    \subsection{Teksty}
        Waga: 8.
    \subsection{Aktywność}
        Waga: 7. Można je zdobyć za przeprowadzenie lekcji, referatu, prezentacji, dyskusję, zgłaszanie się na lekcji.
    \subsection{Zadania domowe}
        Waga: 6.
    \subsection{Praca z mapą}
        Waga: 5.
    \subsection{Udział w olimpiadach}
        Waga: 10.
    \subsection{Nieprzygotowania}
        Dwa nieprzygotowania, dwa braki zadań.
\section{Działalność pozalekcyjna}
    \subsection{Klub Miłośników Krakowa}
        \href{http://klubmk.v-lo.krakow.pl/}{http://klubmk.v-lo.krakow.pl/}
    \subsection{Projekty historyczno--turystyczne}
        Jura, Polski Spisz, szlak cysterski, szlak łemkowski
    \subsection{Olimpiady, konkursy}
        \begin{itemize}
            \item Olimpiada Historyczna
            \item Olimpiada Mediewistyczna
            \item Olimpiada Wiedzy o Prawach Człowieka
            \item Konkurs ,,Czy znasz Kraków?'' (indeksy tylko kl. drugie)
            \item Konkurs plastyczny ,,Krąg''
        \end{itemize}
\section{Konsultacje dla uczniów}
    Raz w tygodniu w piątek, 16:00, sala 32.

\chapter{7 września 2011 -- \textit{Świat na początku XIX wieku}}
    Wiek XIX przynosi duże przewartościowanie, nagłe i duże zmiany.
    \section{Różnice w czasie pojawiania się \textbf{rewolucji przemysłowej}}
        Nowe czasy przynoszą maszyny. W Anglii rewolucja zaczyna się w połowie XVIII wieku. W tym czasie powstała maszyna parowa, następnie Woolf ją udoskonalił. Żelazo upowszechniło się, a w XVI wieku (Anglia) zaczyna się \textbf{rewolucja agrarna}. Dzięki specjalizacji i produkcji na rynku pojawił się przemysł spożywczy, konserwy\dots

        Przy okazji zaczęła się moda na nawożenie mineralne. Wbrew tezie \textbf{Thomasa Malthusa} o niemożliwości wyżywienia ludzkości pojawiły się nadwyżki żywności. David Ricardo wprowadził ,,spiżowe prawo płacy'', mówiące by nie płacić zbyt wiele. Nagle zaczął się gwałtowny przyrost ludności, a manufaktury (ręczna produkcja) przerodziły się w fabryki (automatyczna produkcja).
    \section{Proces uwłaszczania chłopów}
        Stało się to dzięki wojnom napoleońskim. Występował tylko na zachodzie i przez to pogłębił się dualizm wschód--zachód. Chłop wolny mógł się rozwinąć, zasilić fabryki, przemysł. Dzięki wyzyskiwaniu takich ludzi przemysł bardzo szybko sie rozwinął.
    \section{Postępy w komunikacji}
        Maszyna parowa w statkach od 1807 roku (Robert Fulton). Najpierw pojawiały się z kołami, dopiero później ze śrubami okrętowymi. Długo funkcjonowały statki żaglowe (klipery) i żaglowo--parowe.

        Samochody na parę i gaz -- nieudane eksperymenty. Powstały też lokomotywy w 1814 roku (Georg Stephenson). Uzyskiwała ona prędkość $6\frac{km}h$. Koleje stały się bardzo popularne np. w krajach niemieckich (w Krakowie od 1847 roku).

        Wystąpiła globalizacja handlu i nadprodukcja i szukanier rynków zbytu poza Europą.

    \section{Przemiany społeczne i początki ruchu robotniczego} % (fold)
    \label{sec:przemiany_spo_eczne_i_pocz_tki_ruchu_robotniczego}
        Zanik społeczeństwa stanowego -- powstanie \textbf{społeczeństwa klasowego}. Powstały nowe warstwy społeczeństwa: robotnicy i burżuazja.

        \subsection{Położenie robotników} % (fold)
        \label{sub:po_o_enie_robotnik_w}
            Robotnicy nie mieli zabezpieczeń socjalnych (odszkodowań, rent i ubezpieczeń), bardzo niski wiek zatrudnienia (od 5 lat). Oprócz tego dostawali niskie pensje i złe warunki mieszkalne.
        % subsection po_o_enie_robotnik_w (end)
        
        \subsection{Ruch robotniczy} % (fold)
        \label{sub:ruch_robotniczy}
            Robotnicy zaczęli tworzyć związki, strajki oraz ideologie. Powstał na przykład ruch luddystów, niszczących maszyny.
        % subsection ruch_robotniczy (end)

        \subsection{Socjalizm utopijny} % (fold)
        \label{sub:socjalizm_utopijny}
            \paragraph{Utopia} coś idealnego, wymarzonego, pięknego, ale nieosiągalnego.
            \paragraph{Socjalizm utopijny}
            Nazwa ustroju powstała od dzieła Thomasa More'a ,,Utopia'' (1516 r.). Założenia socjalistów:
                \begin{itemize}
                    \item Należy zrównoać społeczeństwo i sprawiedliwie dzielić dobra między ludzi (zanik stanów i klas),
                    \item Wyrzekali się siły
                    \item Dążyli do pokojowych zmian,
                    \item Postulat zerwania z egoistycznym liberalizmem,
                    \item Kierowanie się altruizmem.
                \end{itemize}
            \paragraph{Robert Owen} w swoich fabrykach podwyższał pensje i warunki pracy, przegrał konkurencję z firmami produkującymi tańsze towary. Z bogatego fabrykanta stał się bankrutem. Uznany za twórcę idei spółdzielności.
            \paragraph{Charles Fourier} -- twórca falansterów -- wspólnot robotniczych (wspólne mieszkania, stołówki, drobiazgowy podział pracy). Byli tam przedstawiciele wszystkich zawodów, co gwarantowało samowystarczalność. Wspólnoty tworzył głównie na terenie USA. Akcja zakończyła się klęską. Uznany za twórcę terminu ,,feminizm'', dążył do emancypacji kobiet.
            \paragraph{Henri Saint--Simon} -- podział społeczeństwa na pszczoły (robotnicy) i trutnie (burżuazja). Tylko pszczołom należy się zapłata. Postulował oddanie społeczeństwu przemysłu i ziemi rolnej. Dochody miał rozdzielać bank centralny.
        % subsection socjalizm_utopijny (end)

        \subsection{Współpraca robotników z wigami przeciwko torysom (Anglia)} % (fold)
        \label{sub:wsp_praca_robotnik_w_z_wigami}
            Wigowie to liberałowie, torysi to konserwatyści. Coraz więcej ludzi dostawało się do parlamentu. W parlamencie działały dwie takie partie, wigowie i torysowie. Pierwsi byli ludźmi z wyższych sfer. 
        % subsection wsp_praca_robotnik_w_z_wigami (end)

        \subsection{Reforma wyborcza (Anglia)} % (fold)
        \label{sub:reforma_wyborcza}
            W 1832 roku rozszeszyły się prawa wyborcze na przemysłowców i nowe miasta przemysłowe w oparciu o cenzus majątkowy (odebranie ich ,,zgniłym miasteczkom'').
        % subsection reforma_wyborcza (end)
    % section przemiany_spo_eczne_i_pocz_tki_ruchu_robotniczego (end)

\chapter{14 września 2011 -- \textit{Kongres wiedeński}} % (fold)
\label{cha:14_wrze_nia_2011_textit}
    \section{Wprowadzenie} % (fold)
    \label{sec:wprowadzenie}
        Kongres wiedeński trwał od 1814 roku do 1815. Miał bardzo nieformalny charakter obrad (,,tańczący kongres'' od wielu balów i towarzyskich spotkań). Nigdy go nie zwołano i nigdy go nie zakończono. Pierwszy kongres, który po wojnach napoleońskich miał sytuację uspokoić i zaprowadzić ład.
    % section wprowadzenie (end)
    \section{Reprezentacje państw} % (fold)
    \label{sec:reprezentacje_pa_stw}
        Prym wiodły państwa, które pokonały Napoleona. Chcieli go upokorzyć i zaprowdzić ład. Stąd nie dziwią nas pierwsze cztery reprezentacje:
        \begin{itemize}
            \item Wielka Brytania -- M.S.Z. Robert Stewart Castlereagh
            \item Rosja -- Aleksander I
            \item Austria -- Franciszek I i kanclerz Klemens von Metternich
            \item Prusy -- Fryderyk Wilhelm III
            \item Francja -- M.S.Z. Charles Talleyrand (!)
        \end{itemize}
        Natomiast nie-wiadomo-skąd na kongres został zaproszony Talleyrand. Przed rewolucją francuską był biskupem, w czasie rewolucji był po stronie rewolucji, przyszedł Napoleon -- stanął za Napoleonem, a po Napoleonie zostawił go i przeszedł na stronę zwycięską. Natomiast został zaproszony też dlatego, by Francja nie czuła się zostawiona, by nie tworzyć kolejnego konfliktu. Względy dyplomatyczne. Po wojnie władzę przejęła dynastia Burbonów, więc aby dobrze się kojarzyła, ze spokojem zaproszono za ich czasów na kongres przedstawiciela.
    % section reprezentacje_pa_stw (end)
    \section{Główne zasady} % (fold)
    \label{sec:g_wne_zasady}
        \begin{description}
            \item[Restauracja] powrót do ustrojów i terytoriów sprzed 1789 r. (pominięcie kryterium narodowości),
            \item[Legitymizm] powrót odsuniętych od władzy dynastii,
            \item[Równowaga sił] zapobieganie dominacji państw w Europie (wymyślona przez Anglików).
        \end{description}
    % section g_wne_zasady (end)
    \section{Decyzje} % (fold)
    \label{sec:decyzje}
        \subsection{Wielka Brytania} % (fold)
        \label{sub:wielka_brytania}
            \begin{itemize}
                \item uzyskanie korzyści już w czasie wojen (z wielu wojen WB wychodziła najmniej zniszczona z racji położenia),
                \item zniszczenie konkurencyjnych flot (jedyna flota na morzach -- brytyjska),
                \item uzyskanie wysp i terytoriów należących do Holandii (Mauritius, Cejlon, Kraj Przylądkowy) oraz Malty Joannitów i duńskiego Helgolandu.
            \end{itemize}
        % subsection wielka_brytania (end)
        \subsection{Rosja} % (fold)
        \label{sub:rosja}
            \begin{itemize}
                \item uzyskanie Finlandii od Szwecji i Besarabii od Turcji,
                \item wpływy w nowopowstałym Królestwie Polskim.
            \end{itemize}
        % subsection rosja (end)
        \subsection{Austria} % (fold)
        \label{sub:austria}
            \begin{itemize}
                \item otrzymała przewodnictwo w Związku Niemieckim (dynastia Habsburgów).
            \end{itemize}
        % subsection austria (end)
        \subsection{Prusy} % (fold)
        \label{sub:prusy}
            \begin{itemize}
                \item z ziem polskich odzyskały tylko Wielkopolskę i Gdańsk (nie za dużo ziem polskich, bo sprawialiśmy problemy, będąc niegrzecznymi, tworząc powstania),
                \item na zachodzie ok. $\frac13$ Saksonii, Westfalię, Nadrenię, Pomorze Szwedzkie.
            \end{itemize}
        % subsection prusy (end)
        \subsection{Francja} % (fold)
        \label{sub:francja}
            \begin{itemize}
                \item w 1814 r. powrót do granic 1792 r. (zostawili im dużo terytoriów, żeby utwardzić pozycję Habsburgów),
                \item w 1815 r. (po ,,stu dniach'', kampanii Napoleona po ucieczce z Elby) do granic z 1790 r. (traci Sabaudię),
                \item otoczenie ,,kordonem sanitarnym'' (państwami sprzymierzonymi z wygranymi) przez inne państwa.
            \end{itemize}
        % subsection francja (end)
        \subsection{Włochy} % (fold)
        \label{sub:w_ochy}
            \begin{itemize}
                \item likwidacja Królestwa Włoch (swoistego zjednoczenia)
                \item ponowne rozdrobnienie: Państwo Kościelne, Królestwo Obojga Sycylii, Królestwo Sardynii, Parma (Maria Luiza), Modena, Toskania (trzy ostatnie pod rządami członków dynastii Habsburgów)
            \end{itemize}
        % subsection w_ochy (end)
        \subsection{Niemcy} % (fold)
        \label{sub:niemcy}
            \begin{itemize}
                \item rozwiązanie Związku Reńskiego $\rightarrow$ Związek Niemiecki z cesarzem Austrii na czele -- próba odzyskania znaczenia przez Habsburgów
                \item do związku należało 34 państw i kilka wolnych miast -- granice pokrywały się w przybliżeniu z granicami I Rzeszy (bez Niderlandów), należał do niego król Anglii (król Hanoweru)
            \end{itemize}
        % subsection niemcy (end)
        \subsection{Inne} % (fold)
        \label{sub:inne}
            \begin{itemize}
                \item nowe państwo -- Królestwo Zjednoczonych Niderlandów (Holandia, Belgia, Luksemburg) (dotychczas Francja chciała zabierać terytoria, zwłaszcza Belgii; był to wielki błąd ze strony kongresu, państwa te były bardzo różne, a zwłaszcza Belgowie),
                \item narzucenie neutralności Szwajcarii (żeby Francja nie mogła się z nią sprzymierzyć i nie mieć otwartej drogi do Europy),
                \item Dania zmuszona oddać Norwegię Szwecji (dynastia Bernadot) (Norwegowie odzyskują niepodległość w 1905 r.),
                \item powołanie i działalność Świętego Przymierza (Austria, Rosja, Prusy) -- na straży \textit{ancien regime} (starego porządku), ,,koncert mocarstw''.
                \begin{itemize}
                    \item przystąpiły do niego liczne państwa Europy za wyjątkiem Anglii (chcieli się odosobnić od Europy, bo przynosiła im same problemy, poza tym ustroje państw Świętego Przymierza bardzo różniły się od ustroju Anglii) oraz Turcji (im z kolei przeszkadzała religia państw Świętego Przymierza) i Państwa Kościelnego (inna hierarchia w religii).
                \end{itemize}
            \end{itemize}
        % subsection inne (end)
    % section decyzje (end)
% chapter 14_wrze_nia_2011_textit (end)
\chapter{19 września 2011 -- \textit{Ustalenia kongresu wiedeńskiego wobec ziem polskich}} % (fold)
\label{cha:19_wrze_nia_2011_textit_ustalenia_kongresu_wiede_skiego_wobec_ziem_polskich}
    \section{Sprawa polska na kongresie} % (fold)
    \label{sec:sprawa_polska_na_kongresie}
        \subsection{Strach przed dążeniami niepodległościowymi Polaków} % (fold)
        \label{sub:strach_przed_d_eniami_niepodleg_o_ciowymi_polak_w}
            Teoretycznie ziemie polskie powinny zostać rozdane Prusom, Rosji i Austrii, jednak tak się nie stało. Decydenci bojąc się dążeń niepodległościowych stworzyli coś na kształt Państwa.
        % subsection strach_przed_d_eniami_niepodleg_o_ciowymi_polak_w (end)
        \subsection{Chęć przejęcia terytorium Księstwa Warszawskiego i utrzymania armii polskiej przez cara Aleksandra I} % (fold)
        \label{sub:ch_przej_cia_terytorium_ksi_stwa_warszawskiego_i_utrzymania_armii_polskiej_przez_cara_aleksandra_i}
            Rosja bardzo chciała dostać Księstwo Warszawskie. Miało ono świetne wojsko, jedne z lepszych na świecie. Zjawiło się ono w Rosji, idąc z Napoleonem na nią. Wszyscy po stronie Polaków giną, nikną, natomiast Wojsko Polskie, biedne, jako jedyne wraca z wojny z rozwiniętymi sztandarami i szczęśliwe. Rosja w tym momencie gra altruistycznego przyjaciela. Ta sama armia później uderza na Rosję.
        % subsection ch_przej_cia_terytorium_ksi_stwa_warszawskiego_i_utrzymania_armii_polskiej_przez_cara_aleksandra_i (end)
        \subsection{Spół wokół ziem polskich grozi wojną (Rosję poparł tylko król Prus)} % (fold)
        \label{sub:sp_wok_ziem_polskich_grozi_wojn_}
            Państwa nie były zadowolone z takiego układu ziem -- naruszało to balans sił w Europie.
        % subsection sp_wok_ziem_polskich_grozi_wojn_ (end)
    % section sprawa_polska_na_kongresie (end)
    \section{Postanowienie o nowym podziale ziem polskich} % (fold)
    \label{sec:postanowienie_o_nowym_podziale_ziem_polskich}
        \subsection{Przemianowanie Księstwa Warszawskiego na Królestwo Polskie} % (fold)
        \label{sub:przemianowanie_ksi_stwa_warszawskiego_na_kr_lestwo_polskie}
            Przy okazji Wielkopolska przechodzi do rąk Prus (nie chciały dużo ziem, obawiając się powstań). Kongresówkę (Królestwo Polskie) związano z Rosją unią personalną.
        % subsection przemianowanie_ksi_stwa_warszawskiego_na_kr_lestwo_polskie (end)
        \subsection{Powstaje Wolne Miasto Kraków} % (fold)
        \label{sub:powstaje_wolne_miasto_krak_w}
            Zwane również Rzeczpospolitą Krakowską. Było to zrobione po to, by Rosja nie dostała dwóch dużych miast -- Warszawy i Krakowa. Zatem stworzono Wolne Miasto Kraków. Naruszyłoby do balans sił.
        % subsection powstaje_wolne_miasto_krak_w (end)
        \subsection{Wielkie Księstwo Poznańskie} % (fold)
        \label{sub:wielkie_ksi_stwo_pozna_skie}
            Powstaje Wielkie Księstwo Poznańskie na terenach Austrii.
        % subsection wielkie_ksi_stwo_pozna_skie (end)
    % section postanowienie_o_nowym_podziale_ziem_polskich (end)

    \section{Wielkie Księstwo Poznańskie} % (fold)
    \label{sec:wielkie_ksi_stwo_pozna_skie}
        \subsection{Ustrój} % (fold)
        \label{sub:ustr_j}
            \begin{itemize}
                \item stanowiło część Prus, ale autonomiczną (ciągle ograniczoną), wskazuje na to godło, na którym godło polskie (biały orzeł na czerwonym polu) jest na środku większego czarnego orła,
                \item urząd namiestnika (Antoni Radziwiłł)
            \end{itemize}
        % subsection ustr_j (end)
        \subsection{Uwłaszczenie chłopów} % (fold)
        \label{sub:uw_aszczenie_ch_op_w}
            Księstwo Poznańskie było jedynym kawałkiem Polski pod zaborem, gdzie uwłaszczono chłopów w 1823 roku, ,,regulacja''. Chłopi mogli otrzymać użytkowaną ziemię, jeżeli za nią zapłacili lub oddali jej część panu. Dzięki temu ziemie otrzymywali tylko bogaci chłopi posiadający sprzężaj, wozy. Uzyskanej ziemi nie można było już dzielić (zapobiegało to rozdrobnieniu). W efekcie powstały duże, produkujące na rynek gospodarstwa chłopskie. Wzrosła ich wydajność (chłopi pracowali tylko dla siebie).

            Panowie za otrzymane pieniądze ulepszali swoje gospodarstwa. Chłopi, których nie było stać na wykup ziemi przenosili się do miast, zasilając przemysł. Szlachta była zadowolona: albo dostawała ziemię albo pieniądze.

            W Wielkopolsce dzięku temu nie doszło do antagonizmu chłopi -- szlachta. Poza tym wydajność pól znacznie wzrosła. Zbierano dużo więcej plonów niż pod innymi zaborami, wydajność ogromnych pól była dużo większa niż małych pasemek.
        % subsection uw_aszczenie_ch_op_w (end)
    % section wielkie_ksi_stwo_pozna_skie (end)
    \section{Wolne Miasto Kraków (Rzeczpospolita Krakowska)} % (fold)
    \label{sec:wolne_miasto_krak_w_}
        \subsection{Ustrój} % (fold)
        \label{sub:ustr_j}
            W 1818 roku uzyskała konsytucję, mimo to byli tu \textbf{rezydenci}, przedstawiciele trzech mocarstw. Władza wykonawcza zwała się Senatem Rządzącym (prezesem senatu Stanisław Wodzicki).
        % subsection ustr_j (end)
        \subsection{Granice} % (fold)
        \label{sec:granice}
            Granice przebiegały na południu tak jak Wisła, na zachód do Mysłowic, przez moment północna granica płynęła wraz z Przemszą, a następnie na ukos na południowy wschód pochłaniały Kocmyrzów. Powstała strefa wolnocłowa -- silny rozwój handlu. Problem był w tym, że towary można było wwozić bez cła, natomiast wywozić już nie. Dzięki temu rozwinął się przemyt.

            Uniwersytet Jagielloński uzyskał autonomię. Chłopów oczynszowano, a sam Kraków miasto bardzo przebudowano (burzenie murów średniowiecznych oraz ratusza, kościołów; założenie Plant). Jedyny kawałek muru pozostawiono przy ulicy Floriańskiej, a podanym powodem był taki, by nie podwiewało kwiaciarkom na Floriańskiej spódnic, widok byłby nieobyczajny.

            Ratusz był trochę podniszczony, postanowiono, że zburzą stary, a następnie wybudują nowy. Lata minęły, a nowy ratusz nie powstał. Ostatnio Niemcy w czasie drugiej wojny światowej chcieli go wybudować, ale na szczęście tego nie zrobili (brzydki styl).
        % subsection granice (end)
    % section wolne_miasto_krak_w_ (end)
    \section{Królestwo Polskie} % (fold)
    \label{sec:kr_lestwo_polskie}
        \subsection{Godło} % (fold)
        \label{sub:god_o}
            Podobne jest do godła Wielkiego Księstwa Poznańskiego, na piersi dwugłowego szarego orła umieszczono orła polskiego.
        % subsection god_o (end)
        \subsection{Ustrój} % (fold)
        \label{sub:ustr_j}
            Królestwo było związane z Rosją unią personalną (car Aleksander I królem polskim). W 1815 roku ustanowiono okrojoną konstytucję (przypominającą konstytucję Księstwa Warszawskiego).
        % subsection ustr_j (end)
        \subsection{Autonomia} % (fold)
        \label{sub:autonomia}
            \begin{itemize}
                \item własny parlament, wojsko, urzędy, prawo, oświatę, politykę gospodarczą,
                \item rola króla była jednak bardzo duża (mianował namiestnika, inicjatywa ustawodawcza, prawo weta, wydawanie dekretów z pominięciem parlamentu itd.),
                \item władza wykonawcza -- Rada Stanu,
                \item parlament (król, senat, izba poselska).
            \end{itemize}
        % subsection autonomia (end)
        \subsection{Liberalne elementy konstytucji} % (fold)
        \label{sub:liberalne_elementy_konstytucji}
            Na 128 posłów było 51 depytowanych wybieranych przez zgromadzenia gminne i 77 przez sejmiki szlacheckie. Prawa wyborcze dostało 100 tysięcy osób. Był niski cenzus majątkowy przy wyborach do sejmu (należy płacić ,,jaki bądź'' podatek). \textbf{W teorii} respektowano prawa naturalne, panowała wolność osobista, słowa, wyznania, równość wobec prawa i nietykalność majątkowa.
        % subsection liberalne_elementy_konstytucji (end)
    % section kr_lestwo_polskie (end)
% chapter 19_wrze_nia_2011_textit_ustalenia_kongresu_wiede_skiego_wobec_ziem_polskich (end)
\chapter{21 września 2011 -- \textit{Geneza powstania listopadowego}} % (fold)
\label{cha:21_wrze_nia_2011_textit}
    \section{Tekst do przeczytania} % (fold)
    \label{sec:tekst_do_przeczytania}
        Na 3 października należy przeczytać tekst ,,I o czym tu mówić'' Jarosława Czubatego. Znajduje się on na e--dysku.
    % section tekst_do_przeczytania (end)
    Powstanie listopadowe było jedynym powstaniem, które miało szansę wygrać. Niestety zaprzepaściliśmy tą szansę, z własnej winy. Istnieje pogląd, że powstania wywoływali sami Rosjanie. Co ciekawe, pomiędzy kolejnymi powstaniami przerwa to dokładnie 30 lat. Działo się tak, ponieważ wtedy przychodziło nowe, żądne wolności pokolenie, stare pokolenie zapominało o represjach popowstaniowych, natomiast po pewnym czasie przychodziła odwilż represyjna i można było myśleć o kolejnym powstaniu.
    \section{Łamanie konstytucji} % (fold)
    \label{sec:_amanie_konstytucji}
        \subsection{Zaufani ludzie w urzędach} % (fold)
        \label{sub:zaufani_ludzie_w_urz_dach}
            Najwyższe urzędy w rękach zaufanych ludzi cara, namiestnikiem generał Józef Zajączek (brał udział w różnych powstaniach, jednak teraz stanął po stronie cara), naczelnym wodzem -- wielki książę Konstanty (brat cara, surowy dowódca, ale dbał o wojsko).
        % subsection zaufani_ludzie_w_urz_dach (end)
        \subsection{Działalność Nikołaja Nowosilcowa} % (fold)
        \label{sub:dzia_alno_niko_aja_nowosilcowa}
            Chciał pokazać carowi, że studenci są groźni. Szukał powstań, dzięki temu pokazując carowi, że sam jest ważny. Jednak ich protesty nie miały większego znaczenia.
        % subsection dzia_alno_niko_aja_nowosilcowa (end)
        \subsection{Przerwanie obrad sejmu} % (fold)
        \label{sub:przerwanie_obrad_sejmu}
            W latach 1820 - 1825 car nie zwoływał sejmu, bo po co -- sam wydawał dekrety.
        % subsection przerwanie_obrad_sejmu (end)
    % section _amanie_konstytucji (end)
    \section{Opozycja legalna -- liberałowie} % (fold)
    \label{sec:opozycja_legalna}
        Taką opozycją byli liberałowie, dążyli do przestrzegania konstytucji, pilnowali wywiązywania się z obietnic cara. Zwani byli ,,kaliszanami'', bo większość z nich pochodziła z okolic Kalisza. Prowadzili czasopismo ,,Orzeł Biały''. Przedstawicielami byli np. Wincenty i Bonawentura Niemojowscy.
    % section opozycja_legalna (end)
    \section{Opozycja nielegalna} % (fold)
    \label{sec:opozycja_nielegalna_}
        Były to organizacje spiskowe studenckie, wojskowe. W Warszawie np. UW - Panta Kojna, Związek Wolnych Polaków. W Wilnie Filareci, Filomaci i inne.

        Natomiast ważną organizacją było Wolnomularstwo Narodowe (rok 1819), popularne wśród oficerów. Praktycznie wszyscy, którzy coś znaczyli w państwie należeli do tej organizacji. Uważali, że mają korzenie w średniowiecznych czasach, byli tajemniczy. Oddziały masonów były w wielu krajach, nie tylko w Polsce. Należał na przykład Walerian Łukasiński.

        Było też Towarzystwo Patriotyczne, przygotowywało ono powstania, miało strukturę karbonarską. Jednak uwięziono Łukasińskiego, co osłabiło opozycję. Umarł w więzieniu w 1868 roku.

        Interesującym pomysłem była współpraca polskiej opozycji z opozycją rosyjską (dekabrystami). Oni niestety byli nieudolni, pod presją wydali Polaków.

        W 1828 roku powstał Spisek podchorążych.
    % section opozycja_nielegalna_ (end)
    \section{Pośrednie i bezpośrednie przyczyny wybuchu} % (fold)
    \label{sec:po_rednie_i_bezpo_rednie_przyczyny_wybuchu}
        \begin{itemize}
            \item niewywiązywanie się cara z obietnic i brak poszanowania konstytucji,
            \item wprowadzenie cenzury,
            \item areszt domowy niektórych posłów,
            \item precedens Grecji (udane powstanie przy pomocy mocarstw), walczyli oni przeciwko Turkom, cała Europa stanęła za nimi, niestety za nami nie,
            \item ,,słońce lipca'' -- udana rewolucja lipcowa we Francji w 1830 roku, usunęli króla Karola X Burbona i powołali Ludwika Filipa). Liczyliśmy na pomoc od nich, jednak nie otrzymaliśmy jej,
            \item powstanie Belgów przeciwko Holendrom poparte przez Francję,
            \item niechęć naszej armii do interwencji w obronie zasad kongresu wiedeńskiego (car rzekomo miał ją użyć na zachodzie),
            \item rola literatury romantycznej (Mickiewicz),
            \item wykrycie spisku przez policję przyspiesza wybuch powstania.
        \end{itemize}
    % section po_rednie_i_bezpo_rednie_przyczyny_wybuchu (end)
    \begin{center}
        \bf Zadanie domowe!
    \end{center}
% chapter 21_wrze_nia_2011_textit (end)
\chapter{3 października 2011 -- \textit{Powstanie listopadowe}} % (fold)
\label{cha:3_pa_dziernika_2011_textit}
    \section{Cechy} % (fold)
    \label{sec:cechy}
        \begin{itemize}
            \item spontaniczność akcji powstańczej \textbf{29 listopada 1830 r.}
            \item ucieczka Konstantego i wycofanie się Rosjan
            \item powstanie Rządu Tymczasowego (Rada Administracyjna, Lelewel, Czartoryski)
            \item dyktatorem \textbf{Józef Chłopicki} (stary generał z dużym autorytetem, mimo to niewidzący sensu w walce)
            \begin{itemize}
                \item generałowie nie chcieli walczyć
                \item misja pojednawcza Druckiego-Lubeckiego (poszedł do cara, by ujednać łaskę)
                \item powstrzymywanie działań powstańczych przez Chłopickiego
                \item fiasko rozmów z carem Druckiego-Lubeckiego $\longrightarrow$ dymisja Chłopickiego
            \end{itemize}
            \item \textbf{25 stycznia 1830 r.} -- detronizacja \textbf{Mikołaja I} przez sejm
        \end{itemize}
    % section cechy (end)
    \section{Przebieg} % (fold)
    \label{sec:przebieg}
        \begin{enumerate}
            \item Szukanie sprzymierzeńców po ulicach Warszawy.
            \item Atak na Arsenał w Warszawie.
            \item Wysłanie misji pojednawczej do cara (fiasko).
            \item Detronizacja Chłopickiego, postawienie na walkę.
            \item Walki
            \begin{enumerate}
                \item \textbf{Luty 1831 r.} -- zwycięstwo pod \textbf{Stoczkiem} i przegrana (?) pod \textbf{Grochowem} (bój o Olszynkę Grochowską)
                \item \textbf{Wiosna 1831 r.} -- ofensywa mimo straty Chłopickiego
                \begin{itemize}
                    \item nowym wodzem \textbf{Jan Skrzynecki} -- ostrożny, spowalniał działania
                    \item wielka rola doradcza \textbf{pułk. Ignacego Prądzyńskiego} -- (świetny plan ataku na magazyny w Siedlcach)
                    \item rozbicie Rosjan pod \textbf{Wawrem}, \textbf{Dębem Wielkim}, \textbf{Iganiami}
                    \item na skutek kunktatorstwa Skrzyneckiego nie zdobyto Siedlec
                    \item 26 maja 1831 r. -- dobra zasadzka Prądzyńskiego, natomiast Skrzynecki plan zmodyfikował, przez co przegraliśmy -- klęska pod \textbf{Ostrołęką} -- przełom w walkach
                \end{itemize}
            \end{enumerate}
            \item Klęska powstania
            \begin{itemize}
                \item manewr \textbf{Paskiewicza} -- zajście Warszawy od zachodu
                \item po Skrzyneckim wodzem \textbf{Henryk Dembiński}
                \item upadek rządu $\rightarrow$ dyktatorem \textbf{Jan Krukowiecki} ,,grabarz powstania''
                \item błędy polskiego dowództwa (gen. Ramorino, 20 tysięcy żołnierzy wysłanych pod Lublin)
                \item \textbf{6 września} -- szturm na Warszawę (przez Wolę) -- wycofanie wojsk
                \item Maciej Rybiński -- ostatni dowódca, zaprzestanie walk
                \item rozproszenie armii polskiej i przejście granic pruskiej, austriackiej i RP krakowskiej $\rightarrow$ początek emigracji
            \end{itemize}
        \end{enumerate}
    % section przebieg (end)
% chapter 3_pa_dziernika_2011_textit (end)

\chapter{5 października 2011 -- \textit{Ziemie polskie w latach 1831-1846}} % (fold)
\label{cha:5_pa_dziernika_2011_textit}
    \section{Represje po powstaniu} % (fold)
    \label{sec:represje_po_powstaniu}
        Powstanie listopadowe upadło. Oczywiste się stało, że będą represje wobec Królestwa Polskiego
        \begin{itemize}
            \item zsyłki na Syberię (ludzie dostawali prace) i konfiskaty majątków
            \item wcielanie do armii rosyjskiej pozostałych żołnierzy
            \item kontrybucja (kary pieniężne)
            \item \textbf{Statut Organiczny} zamist konstytucji (zniesienie wojska, sejmu, koronacji na króla, odrębnych granic)
            \item likwidacja UW, Uniwerstytetu Wileńskiego i Towarzystwa Przyjaciół Nauk (uniwersytety zażewiem buntu)
            \item walka z kościołem katolickim i unickim (prawosławni pod władzą papieża), (likwidacja na wschodzie)
            \item \textbf{noc paskiewiczowska} -- rządy Iwana Paskiewicza w Warszawie (budowa cytadeli)
        \end{itemize}
        W zaborze pruskim zaczęła się germanizacja oraz \textbf{praca organiczna}, czyli przedsięwzięcia gospodarcze, oświatowe, naukowe i kulturalne.
    % section represje_po_powstaniu (end)
    \section{Działalność emisariuszy} % (fold)
    \label{sec:dzia_alno_emisariuszy}
        \paragraph{Emisariusze} wspomagali i kierowali działalnością spiskową. Józef Zaliwski w 1831 roku stworzył akcję \textbf{Zemsty Ludu}, niestety zgłosiło się tylko paręnaście osób i szybko został złapany.
    % section dzia_alno_emisariuszy (end)
    \section{Plany nowego powstania} % (fold)
    \label{sec:plany_nowego_powstania}
        W całej Polsce w trzech zaborach na raz miało w \textbf{lutym 1846 roku} wybuchnąć powstanie. Planowanym wodzem miał być \textbf{Ludwik Mierosławski}, który niestety został aresztowany.

        Powstanie jednak powstało tylko w Rzeczypospolitej Krakowskiej. Dlatego, że:
        \begin{itemize}
            \item dość duża swoboda działań na terenie RP Krakowskiej
            \item nie było powstania listopadowego, ludność nie była zrażona do walki
        \end{itemize}
        W Rzeczypospolitej w tych czasach jako jedynej chłopi nie odrabiali pańszczyzny, tylko płacili czynsz.

        18 lutego walkę rozpoczynają pierwsze oddziały w Galicji. 20/21 lutego 1846 roku nastąpił atak na wojska austriackie. Austriacy nie wiedząc jak wygląda ogół powstania, wycofują się z Krakowa. Powstał Rząd Narodowy, który ogłosił Manifest, który będzie respektowany na terenie całej Polski. Był ważny, ponieważ w nim znajdowała się zapowiedź uwłaszczenia chłopów i zniesienia pańszczyzny. Celem było zdobycie chłopów do powstania. Jednak nie udało się to tak łatwo, ponieważ nie wierzyli obiecankom.

        Dyktatorem powstania zostaje \textbf{Jan Tyssowski}, natomiast prawdziwym przywódcą był \textbf{Edward Dembowski}. Wyzwolił miejscowości i miast RP Krakowskiej (Chrzanów, Wieliczka). Niestety nie było poparcia ze strony chłopów (z wyjątkiem Rzeczypospolitej Krakowskiej). Pod Gdowem powstańcy ponieśli klęskę.

        Dembowski zauważa, że w takim razie spróbuje dojść do chłopów poprzez religię. Zorganizował procesję, wychodząca z Krakowa na Podgórze, a później dalej. Jednak zamiast chłopów na Podgórzu spotkali Austriaków, którzy zmasakrowali procesję.

        2 marca Tyssowski składa władzę, powstanie wygasa, składa broń za granicą pruską. W PRLu zwano ten epizod Rewolucją Krakowską.
    % section plany_nowego_powstania (end)
    \section{Rabacja w Galicji} % (fold)
    \label{sec:rabacja_w_galicji}
        \paragraph{Rabacja} powstanie chłopów (przeciwko szlachcie).

        Spowodowane antagonizmem chłopi-szlachta (pańszczyzna, podatki). Na wsi było przeświadczenie, że Austria, car są dobrzy, bo wszystkie reguły musieli egzekwować szlachice.

        W czasie powstania w RP Krakowskiej Austriacy płacili za szlacheckie głowy chłopom. (Wyolbrzymiony mit, działo się tak bardzo rzadko.)

        W trakcie rabacji zniszczono około 500 dworów, zabito ok. 1000 osób. Chłopi atakowali oddziały powstańcze pod Tarnowem i dwory (Tarnowskie i Sądeckie). Jakub Szela za poprowadzenie chłopów do walki dostał piękny dwór na Bukowinie.
    % section rabacja_w_galicji (end)
    \section{Skutki powstania w RP} % (fold)
    \label{sec:skutki_powstania_w_rp}
        \begin{enumerate}
            \item likwidacja Wolnego Miasta Krakowa (wcielenie do Austrii)
            \item początek germanizacji (urzędy, szkoły)
            \item język niemiecki na UJ
            \item zrozumienie konieczności uwłaszczenia chłopstwa
            \item wzrost popularności postaw lojalnistycznych
        \end{enumerate}
    % section skutki_powstania_w_rp (end)
% chapter 5_pa_dziernika_2011_textit (end)
\chapter{10 października 2011 -- \textit{Wiosna ludów}} % (fold)
\label{cha:10_pa_dziernika_2011_textit}
Szereg walk, powstań na przestrzeni wieku.
    \section{Przyczyny} % (fold)
    \label{sec:przyczyny}
        \subsection{Społeczne} % (fold)
        \label{sub:spo_eczne}
            Rzemieślinicy, robotnicy, chłopi domagali się poprawy warunków życia.
        % subsection spo_eczne (end)
        \subsection{Ustrojowe} % (fold)
        \label{sub:ustrojowe}
            Chęć zniesienia monarchii, \emph{ancien regime}, chęć wprowadzenia republiki, konstytucji itp.
        % subsection ustrojowe (end)
        \subsection{Inne} % (fold)
        \label{sub:inne}
            \begin{itemize}
                \item dążenie do \textbf{niepodległości} (Węgrzy, Polacy)
                \item chęć \textbf{zjednoczenia} (Włosi, Niemcy)
                \item nieurodzaj w rolnictwie -- \textbf{głód}
                \item \textbf{kryzys} ekonomiczny
            \end{itemize}
        % subsection inne (end)
    % section przyczyny (end)
    \section{Francja} % (fold)
    \label{sec:francja}
        \subsection{Przyczyny} % (fold)
        \label{sub:przyczyny}
        \begin{itemize}
            \item rządy Ludwika Filipa po rewolucji lipcowej (1830 -- 1848)
            \begin{itemize}
                \item faworyzowanie burżuazji (hasło ,,bogaćcie się'') i krzewienie liberalizmu gospodarczego. Powstają fortuny miesczańskie (\emph{nuworysz}) $\rightarrow$ stąd przydomek ,,król mieszczański''
            \end{itemize}
        \end{itemize}
        % subsection przyczyny (end)
        \subsection{Przebieg rewolucji} % (fold)
        \label{sub:przebieg_rewolucji}
            W ciągu trzech dni monarchia upada -- król na emigracji. Francja staje się \textbf{II republiką} (luty 1848). Uchwalono konstytucję. Do kampanii wyborczej na prezydenta włącza się \textbf{Ludwik Napoleon Bonaparte}, syn Ludwika, brata Napoleona I Bonaparte. W grudniu 1848 roku Ludwik wygrywa, przeprowadza zamach stanu w 1851 roku, a rok później w wyniku kolejnego plebiscytu ogłasza się \textbf{cesarzem Francuzów} (Napoleon III).

            Napoleon III wprowadził liberalizm w gospodarce, sprawną administrację, policję i urzędy.
        % subsection przebieg_rewolucji (end)
    % section francja (end)
    \section{Kraje niemieckie} % (fold)
    \label{sec:kraje_niemieckie}
        Fala protestów przechodzi przez kraje niemieckie (żądania reform, wolności prasy, konstytucji, liberalizacji, demokratyzacji). W Austrii Ferdynand I abdykuje na rzecz \textbf{Franciszka Józefa I} (1848 -- 1916). Nadaje on nową konstytucję, aby znieść ją w 1851 roku.

        W Prusach zamieszki w Berlinie w marcu 1848, Fryderyk Wilhelm IV obiecuje nadanie konstytucji.

        We Franfurcie nad Menem zebrali się przedstawiciele krajów niemieckich, aby obradować nad zjednoczeniem Niemiec. Powstała koncepcja \textbf{wielkich Niemiec} (z Austrią) i \textbf{małych} (bez Austrii). Wielkie Niemcy dawałyby hegemonię Habsburgom i wplątywały Niemców w problemy etniczne Austrii. Król pruski nie zgadza się zostać królem zjednoczonych Niemiec (nie chce brać \emph{korony z błota} i \emph{świńskiej}), a parlament zostaje rozpędzony.
    % section kraje_niemieckie (end)
    \section{Węgry} % (fold)
    \label{sec:w_gry}
        Węgrzy wykorzystują zamieszki w Wiedniu i tworzą obszar autonomiczny, a w 1849 ogłaszają niepodległość. Austria wysyła armię w celu stłumienia dążeń węgierskich. Gdyby nie pomoc Rosji, Węgrzy wygraliby. Na czele powstania węgierskiego staje Lajos Kossuth, poetą Sandora Pet\"{o}fi. Rosja wkracza na Węgry w ramach Świętego Przymierza i obrony \emph{ancien regime}. Węgrzy mimo znacznej pomocy Polaków zostają pokonani przez Iwana Paskiewicza.
    % section w_gry (end)
    \section{Polacy w Wiośnie Ludów} % (fold)
    \label{sec:polacy_w_wio_nie_lud_w}
        \subsection{Powstanie wielkopolskie 1848 r.} % (fold)
        \label{sub:powstanie_wielkopolskie_1848_r_}
            \begin{itemize}
                \item impuls dany przez Wiosnę Ludów w Prusach i Austrii
                \item walki w Berlinie $\rightarrow$ amnestia dla polskich więźniów (Mierosławski, Libelt)
                \item nadzieje na wojnę Rosji z rewolucyjną Europą i sojusz z Prusami
                \item umowa z Prusami w Jarosławcu:
                \begin{itemize}
                    \item spolszczenie administracji na $\frac23$ Księstwa Poznańskiego
                    \item zgoda na 3 tys. wojska polskiego
                \end{itemize}
                \item zerwanie umowy przze Prusy -- administracja na $\frac13$ terenów i likwidacja wojska
                \item Mierosławski rozpoczyna walkę -- 29 kwietnia -- 9 maja 1848
                \item rozproszenie oddziałów i znieszczenie ich przez armię pruską
            \end{itemize}
        % subsection powstanie_wielkopolskie_1848_r_ (end)
        \subsection{Galicja} % (fold)
        \label{sub:galicja}
            Sporo demonstracji, sporo protestów, nic poważniejszego. Gubernator Franz Stadion na terenie Galicji postanowił znieść pańszczyznę w Galicji (aby przeciągnąć chłopów na stronę cesarza).
        % subsection galicja (end)
        \subsection{Udział Polaków w walkach na Węgrzech przeciwko Austriakom i Rosjanom} % (fold)
        \label{sub:udzia_polak_w_w_walkach_na_w_grzech_przeciwko_austriakom_i_rosjanom}
            W latach 1848 -- 1849 r. Bardzo ważną postacią był Józef Bem, Legiony Polskie (Józef Wysocki). Dowódcy polscy byli również we Włoszech i w Niemczech.
        % subsection udzia_polak_w_w_walkach_na_w_grzech_przeciwko_austriakom_i_rosjanom (end)
    % section polacy_w_wio_nie_lud_w (end)
% chapter 10_pa_dziernika_2011_textit (end)

\chapter{17 października 2011 -- \textit{Powstanie styczniowe}} % (fold)
\label{cha:17_pa_dziernika_2011_textit}
    Lekcje na sprawdzian z podręcznika: $1,2,3,5,6,8,9,10,13$

    \section{Sytuacja w Królestwie przed powstaniem} % (fold)
    \label{sec:sytuacja_w_kr_lestwie_przed_powstaniem}
        \subsection{Odwilż posewastopolska} % (fold)
        \label{sub:odwil_posewastopolska}
            Przykład Finlandii pobudza nadzieje (stopniowe nadawanie autonomii). Jednak Rosjanie nie chcieli nam jej dać. Gdy Aleksander II przyjechał do Warszawy powiedział ,,\emph{żadnych marzeń panowie}''.
        % subsection odwil_posewastopolska (end)
        \subsection{Okres manifestacji patriotycznych} % (fold)
        \label{sub:okres_manifestacji_patriotycznych}
            W 1830 roku dorosło nowe pokolenie, niepamiętające klęski powstania. 30 lat później nastąpił początek manifestacji, pierwsze ofiary. W marcu 1861 początek \textbf{żałoby narodowej}. Okazywaliśmy w ten sposób jedność. Denerwowało to Rosjan dlatego, że nie mogli za to nas ukarać. (W czasie PRLu uczniowie wpinali do koszuli oporniki.)
        % subsection okres_manifestacji_patriotycznych (end)
        \subsection{Początek konspiracji} % (fold)
        \label{sub:subsection_name}
            \subsubsection{Podział na białych i czerwonych} % (fold)
            \label{ssub:podzia_na_bia_ych_i_czarnych}
                Biali uważali, że walka tak, ale w bliżej nieokreślonej przyszłości przy pomocy innych państw. Stosowali metody jawne i dyplomatyczne.

                Czerwoni chcieli walczyć już, szybko.
            % subsubsection podzia_na_bia_ych_i_czarnych (end)

            Ciekawa postać \textbf{Aleksandra Wielopolskiego}, lojalisty -- naczelnik nowego Rządu Cywilnego. Będąc lojalnym wobec Rosji doprowdził do polonizacji urzędów. W styczniu 1863 roku zorganizował \textbf{brankę}, pobór do wojska w celu rozbicia organizacji spiskowej. Przez to ludzie mieli do wyboru iść do wojska rosyjskiego albo rozpocząć powstanie.
        % subsection subsection_name (end)
    % section sytuacja_w_kr_lestwie_przed_powstaniem (end)
    \section{Powstanie} % (fold)
    \label{sec:powstanie}
        \subsection{Wybuch} % (fold)
        \label{sub:wybuch}
            Czerwoni przejmują inicjatywę i rozpoczynają powstanie 22 stycznia 1863 roku -- 6 tysięcy żołnierzy. Wydali \textbf{Manifest Tymczasowego Rządu Narodowego}, w którym obiecywali chłopom uwłaszczenia, szlachcie na ziemię zapłaci państwo, bezrolni za udział w powstaniu otrzymają działki z Dóbr Narodowych. Szlachta doszła do wniosku, że jeśli chłopom się nie da ziemii, żadne powstanie nie ma szans. Do tego nie udało im się zdobyć Płocka (planowanej siedziby władz). Za to Rosja wycofywała się z wielu terenów.
        % subsection wybuch (end)
        \subsection{Władze powstańcze} % (fold)
        \label{sub:w_adze_powsta_zce}
            Ludwik Mierosławski pierwszym dyktatorem powstania, po pierwszych klęskach wyjeżdża z kraju. Następnie Marian Langiewicz, klęski w Małopolsce.
        % subsection w_adze_powsta_zce (end)
        \subsection{Przebieg walk} % (fold)
        \label{sub:przebieg_walk}
            Biali traktowali powstanie jako demonstrację zbrojną. Mieliśmy nieregularną, ochotniczą, słabo uzbrojoną armię. Przez to prowadziliśmy partyzancki charakter walk (ok. 1200 bitew i potyczek): Węgrów, Małogoszcz, Żyrzyn, Miechów, Opatów. Poza Kongresówką też były walki, a z innych zaborów dostawaliśmy wielu ochotników. Powstała sieć państwa podziemnego (prasa, sądy, policja, podatki).

            Zyskaliśmy znaczne poparcie rewolucjonistów, ochotników.
        % subsection przebieg_walk (end)
        \subsection{Romuald Traugutt} % (fold)
        \label{sec:romuald_traugutt}
            W październiku 1863 roku dyktatorem zostaje Romuald Traugutt, były płk. armii carskiej, związany z białymi. Dążył do pełnej realizacji dekretu uwłaszczeniowego. Przeorganizował armię celem przetrwania zimy i wiosennej ofensywy. Niestety polskie oddziały na początku 1864 roku zostały rozbite.

            W marcu 1864 roku car wydał dekret o uwłaszczeniu chłopów, odszkodowanie dla ziemian ma zapłacić państwo, a chłopi płacić na ten cel specjalny podatek. Uwłaszczenie chłopów uzależniono od rozbicia partyzantów.

            Przez to chłopi byli wrodzy i bierni wobec powstania. W kwietniu 1864 aresztowano Traugutta, zmarł chwilę później. Na wiosnę następnego roku został rozbity ostatni oddział księdza Brzóski.

            Godło powstańcze składało się z Orła, Pogoni oraz Archanioła Michała.
        % subsection romuald_traugutt (end)
    % section powstanie (end)
% chapter 17_pa_dziernika_2011_textit (end)
\chapter{19 października 2011 -- \textit{Wojna secesyjna}} % (fold)
\label{cha:19_pa_dziernika_2011_textit}
    \section{Geneza} % (fold)
    \label{sec:geneza}
        Wojna toczyła się pomiędzy dwoma częściami Stanów Zjedonoczonych -- północną i południową.

        Głównym powodem jest nierównomierny rozwój gospodarczy. Północ -- rozwinięty przemysł, południowa -- rolnictwo. Poza tym była różnica polityczna, demokraci byli za niewolnictwem (południe), republikanie przeciwko (północ). Każdy stan mógł sobie wybrać czy chcieli utrzymać niewolnictwo i abolicjonizm.

        W 1860 roku odbyły się wybory i rozgorzał spór wokół niewolnictwa. Prezydentem został Abraham Lincoln, republikanin. Stany południowe nie zgodziły się, robiąc secesję z prezydentem Jeffersonem Davisem. Rok później rozpoczęła się wojna secesyjna:
        \begin{itemize}
            \item północ -- południe,
            \item jankesi (ćwok, północ) -- konfederaci.
        \end{itemize}
    % section geneza (end)
    \section{Przebieg} % (fold)
    \label{sec:przebieg}
        Armie różniły się w stopniu przygotowania. Południe było obeznane z bronią, natomiast północ czasem w ogóle nie dotykała broni wcześniej. Stany północne zarządziły nabór do armii (za symboliczną kwotę każdy mężczyzna miał dostać 60 ha ziemi uprawnej).

        Przełom nastąpił w 1863 roku, podczas \textbf{bitwy pod Gettysburgiem}. Generał Lee (konfederaci) przegrał, każąc południowcom wpakować się na pastwiska, gdzie zostali rozstrzelani przez armaty jankesów. Od tej bitwy górą została północ. Generał Sherman zarządził wojnę totalną, palenie i niszczenie wszystkiego.

        Południe \textbf{kapituluje w Appomatox w 1865 roku}.
    % section przebieg (end)
    \section{Skutki} % (fold)
    \label{sec:skutki}
        \begin{itemize}
            \item XIII poprawka do konstytucji
            \item nowe rodzaje broni (karabiny wielostrzałowe -- Winchester, Colt)
            \item pancerniki (Virginia -- południe, Monitor -- północ) -- potwornie drogie, niesterowalne
        \end{itemize}
    % section skutki (end)
% chapter 19_pa_dziernika_2011_textit (end)
\chapter{2 listopada 2011 -- \textit{Zjednoczenie Niemiec}} % (fold)
\label{cha:2_listopada_2011_textit}
    \section{Przeszłość Niemiec} % (fold)
    \label{sec:przesz_o_niemiec}
        \begin{itemize}
            \item Istnienie I Rzeszy 962 -- 1806
            \item Związek Reński 1806 -- 1814 (tworzy go Napoleon, upada wraz z Napoleonem)
            \item Związek Niemiecki 1815 -- 1866 (35 państw i 4 miasta)
            \item Próby oddolnego zjednoczenia Niemiec w czasie Wiosny Ludów (parlament frankfurcki)
        \end{itemize}
    % section przesz_o_niemiec (end)
    \section{Prusy po Wiośnie Ludów} % (fold)
    \label{sec:prusy_po_wio_nie_lud_w}
        Prusy chciały ekonomicznego zjednoczenia Niemiec, w 1834 r. powstał \textbf{Związek Celny} -- wolny handel w państwach niemieckich z wyjątkiem Austrii i Badenii. Dzięki temu zniosły się bariery celne, gospodarcze, ekonomiczne. Najwięcej w tym kierunku działał \textbf{Wilhelm I}. Oprócz niego na tapetę wysunął się kanclerz \textbf{Otto von Bismarck}. Chciały zjednoczyć Niemcy za pomocą armii. Żołnierze byli dobrze wykształceni, często mówi się, że wojny nie wygrały Prusy, tylko nauczyciele. Poza tym armia była dowożona na miejsce walki pociągiem.
    % section prusy_po_wio_nie_lud_w (end)
    \section{Rywalizacja prusko -- austriacka} % (fold)
    \label{sec:rywalizacja_prusko_austriacka}
        Powstały dwie koncepcje z Niemiec: wielkie (\emph{wszystkie państwa z Austrią}) i małe (\emph{bez Austrii}). Południowe kraje katolickie poparły Austrię, północne (luterańskie) nie popierały.

        Według Bismarcka Niemców może zjednoczyć tylko ,,krew i żelazo''. Znaczyło to:
        \begin{description}
            \item[krew] wspólna krew
            \item[żelazo] jak nie chce dalej, to siłą
        \end{description}
    % section rywalizacja_prusko_austriacka (end)
    \section{Wojna prusko -- austriacka (1866 r.)} % (fold)
    \label{sec:wojna_prusko_austriacka_}
        Po stronie Prus stanęły państwa północnoniemieckie i Włochy. Decydującą bitwą była \textbf{bitwa pod Sadową} w Czechach. Prusy rozbiły Austrię. Na mocy pokoju w Pradze i Wiedniu Austria oddała Włochom Wenecję. Związek Niemiecki zostaje rozwiązany. Wyeliminowano Austrię jako konkurentkę do zjednoczenia.
    % section wojna_prusko_austriacka_ (end)
    \section{Utworzenie Związku Północnoniemieckiego (1867 r.)} % (fold)
    \label{sec:utworzenie_zwi_zku_p_nocnoniemieckiego_}
        W jego skład weszły państwa na północ od rzeki Men. Związek był federacją (wspólny monarcha, parlament, rząd). Na czele związku stał król pruski, powołujący kanclerza. Zawiązano sojusz wojskowy związku z państwami południowoniemieckimi. Bismarck szuka pretekstu do wojny z Francją. Dzięki temu państwa południowe stałyby się zagrożone i wspólnie w czasie wojny zjednoczyliby się wszyscy.
        \subsection{Wojna prusko -- francuska (1870 -- 1871 r.)} % (fold)
        \label{sub:wojna_prusko_francuska_}
            Po abdykacji Izabeli II hiszpańskie Kortezy ofiarowały koronę Leopoldowi Hohenzollernowi. W związku z tym Francja poczuła się otoczona zewsząd Hohenzollernami i wystosowała \textbf{depeszę emską}. Bismarck tak zredagował depeszę, że jest ona obraźliwa wobec Francji (posła francuskiego miał obrazić król pruski Wilhelm I). Przez to Francja wypowiedziała wojnę Prusom popartym przez inne państwa niemieckie, obawiające się o swój byt. Szybko oblężono Francuzów w Metz, Francja skapitulowała pod Sedanem.

            Francja oddała Prusom Alzację i Lotaryngię. Poza tym musiała zapłacić kontrybucję. Południowe państwa w końcu przystąpiły do Związku Niemieckiego. W Sali Lustrzanej w Wersalu \textbf{proklamowano Cesarstwo Niemieckie 18 stycznia 1871 roku}. Od tego czasu Wilhelm I został cesarzem, a Bismarck kanclerzem II Rzeszy.
        % subsection wojna_prusko_francuska_ (end)
    % section utworzenie_zwi_zku_p_nocnoniemieckiego_ (end)
    \section{Powstanie Austro-Węgier} % (fold)
    \label{sec:powstanie_austro_w_gier}
        Po przegranej pod Sadową Austria postanowiła dać Węgrom trochę autonomii. W 1867 roku powstała monarchia dualistyczna. Cesarz Austrii został królem Węgier. Stało się tak, ponieważ Austria zdawała sobie sprawę, że jeśli Węgrzy zbuntowaliby się, nie poradziliby sobie z powstaniem.

        Były dwie odrębne konstytucje i dwa parlamenty. Węgrzy byli bardzo dumni ze swojego parlamentu. Poza tym były osobne granice. Do Węgier należała Chorwacja, Slawonia, Słowacja, do Austrii Polska, Galicja, Czechy i inne. Węgry dostały też bardzo duży dział Siedmiogrodu (Transylwanii).
    % section powstanie_austro_w_gier (end)
% chapter 2_listopada_2011_textit (end)
\chapter{9 listopada 2011 -- \textit{Nowe ideologie polityczne XIX w.}} % (fold)
\label{cha:9_listopada_2011_textit}
    \section{Rozwój systemów demokratycznych} % (fold)
    \label{sec:rozw_j_system_w_demokratycznych}
        W Anglii została \textbf{zwiększona liczba uprawnionych do głosowania}. Monarchie absolutne przerodziły się w \textbf{monarchie konstytucyjne}. Zaczęły powstawać pierwsze republiki (Francja). Pozakładano różne cenzusy, jak np. majątkowy, wykształcenia i zawodu. W Austro -- Węgrach powstał system kurialny -- każda klasa głosowała osobno. Chłopi wybierali $x$ posłów, $y$ mieszczanie, $z$ burżuazja\dots Był trochę niesprawiedliwy pod tym względem, że burżuazja (mimo małej swej ilości) wybierała mniej więcej tyle ile kupa chłopów.
    % section rozw_j_system_w_demokratycznych (end)
    \section{Powstawanie partii} % (fold)
    \label{sec:powstawanie_partii}
        Powstał system dwubiegunowy (prawica -- lewica), którzy przetrwał do dziś. Geneza podziału jest prosta. W czasie rewolucji francuskiej po prawej stronie siedli posłowie nastawieni konserwatywnej, a po lewej radykaliści.

        Zaczęły powstawać \textbf{masowe partie}, mające program, strukturę terytorialną, władze, statut, kampanie wyborcze, czasopisma, formalni członkowie, określony elektorat.
    % section powstawanie_partii (end)
    \section{Ewolucja dawnych idei} % (fold)
    \label{sec:ewolucja_dawnych_idei}
        \subsection{Liberalizm} % (fold)
        \label{sub:liberalizm}
            Polegał na zgodzie na powszechne prawo wyborcze bez cenzusów (demokraci). John Stuart Mill postuluje za zwiększeniem osłon socjalnych przez państwo $\rightarrow$ \textbf{liberalizm socjalny}. Mówił on, że ludziom biednym, którym się w życiu nie powiodło należy pomagać zasiłkami itd.
        % subsection liberalizm (end)
        \subsection{Konserwatyzm} % (fold)
        \label{sub:konserwatyzm}
            Zbliżali się do liberałów, uznał za konieczne zmiany, ale w wydaniu ewolucyjnym (stopniowe nadawanie praw wyborczych, uwalnianie rynku) -- Beniamin Disraeli.
        % subsection konserwatyzm (end)
    % section ewolucja_dawnych_idei (end)
    \section{Ruch socjalistyczny i komunistyczny} % (fold)
    \label{sec:ruch_socjalistyczny_i_komunistyczny}
        Duża ilość robotników i ludności niezadowolonej z jakości życia, niskich pensji, braku osłon socjalnych. Przez to zaczęły powstawać \textbf{partie socjalistyczne}, chcące różnych zasiłków, pomocy itd. Zarzewie ruchów było w Niemczech u \textbf{Karola Marksa} i \textbf{Fryderyka Engelsa}. W Niemczech były poteżne fabryki, bardzo szybko się rozwijało, przez to było dużo robotników. Niezadowolonych.

        W 1847 roku powstaje Związek Komunistów. Później zostaje wydany ,,\emph{Manifest komunistyczny}'' i ,,Kapitału''. Znane cytaty:
\begin{center}\emph{
        ,,Widmo komunizmu krąży po Europie''}


        \emph{,,Proletariusze wszystkich krajów łączcie się''}
        \end{center}
        \begin{flushright}
\textbf{Zadanie domowe: Przeczytać ,,\emph{Manifest komunistyczny}'' z edysku.}
        \end{flushright}
    % section ruch_socjalistyczny_i_komunistyczny (end)
% chapter 9_listopada_2011_textit (end)
\chapter{14 listopada 2011 -- \textit{Nowe ideologie polityczne XIX w.}} % (fold)
\label{cha:14_listopada_2011_textit}
    U Marksa występuje bardzo mocny ateizm -- ,,\emph{religia opium ludu}''. Mimo to po pewnym czasie w krajach socjalistycznych pozwalano wrócić do religii. Dawało to im nadzieję, którą pokładali w Bogu i zaświatach, dzięki temu nie buntowali się za bardzo.
    \section{Anarchizm} % (fold)
    \label{sec:anarchizm}
        Stworzony przez Piotra Kropotkina oraz Michała Bakunina. Mówił, że społeczeństwo uzyska wolność, gdy pozbędzie się państwa, a ludzie powinni łączyć się w wolne wspólnoty. Organizowali oni zamachy na polityków (np. na cesarzową Elżbietę). Skrajny anarchizm nazwano \emph{nihilizmem}.
    % section anarchizm (end)
    \section{Socjalizm} % (fold)
    \label{sec:socjalizm}
        W Niemczech w 1891 powstała najstarsza na świecie partia socjalistyczna SPD -- Socjaldemokratyczna Paria Niemiec. W Wielkiej Brytanii nigdy nie było rewolucji komunistycznej. Socjaliści po prostu przeszli do działań parlamentarnych, stąd \emph{Partia Pracy}.

        W Rosji natomiast jako że było mało robotników, ruch robotniczy był mały. Socjalistyczna partia w Rosji powstaje dopiero w 1898 roku \textbf{Socjaldemokratyczna Partia Robotnicza Rosji}. Już wtedy na tapecie pojawia się Lenin.

        Jednak po pewnym czasie nastąpił rozłam w ruchu robotniczym. Tam, gdzie socjaliści mogli robić co chcieli legalnie (jak np. Wielka Brytania) panowała \textbf{socjaldemokracja}. Na przykład Eduard Bernstein twierdził, że kapitalizm można zmienić na drodze legalnych, ewolucyjnych działań (np. drogą parlamentarną) $\rightarrow$ rewizjoniści. Karl Kautsky naciskał na reformy uzyskane drogą demokratyczną $\rightarrow$ reformiści. Powstał też nurt \textbf{komunistyczny}, jak np. bolszewicy.
    % section socjalizm (end)
    \section{Chadecja} % (fold)
    \label{sec:chadecja}
        Chrześcijańska demokracja twierdziła, że konieczny jest sprawiedliwy ład społeczny w państwie. Chcieli tak pokierować rządami, by niwelować różnice w bogactwie, ale z poszanowaniem własności prywatnej. Charakteryzowali się chęcią zmian na drodze legalnej, parlamentarnej bez rewolucji.
    % section chadecja (end)
    \section{Nacjonalizm} % (fold)
    \label{sec:nacjonalizm}
        \begin{quote}
        %\begin{center}
            \textbf{,,Narody? Cóż to takiego? Ja mam tylko poddanych.''}
        %\end{center}
        \begin{flushright}
            \emph{Franciszek I}
        \end{flushright}
        \end{quote}
        Wzrastała świadomość narodowa, zwiększała się rola wspólnej historii oraz przeżyć. Kościół również był czynnikiem narodotwórczym (np. na Ukrainie, w Polsce). Zaczęto się odwoływać do teorii Darwina (przetrwają tylko narody lepiej przystosowane, silniejsze). Powstały teorie krajów wyższych i niższych (Joseph Gobineau). \textit{Szowę} w Francji wymyślił szowinizm. Nasilił się antysemityzm (sprawa Dreyfusa) i ksenofobia.
        \begin{quote}
        %\begin{center}
            \textbf{,,Świadomość narodowa dla narodu jest tym, czym kościec dla człowieka, a tylko człowiek chory czuje, że ma kości''}
        %\end{center}
        \begin{flushright}
            \emph{Bernard Shaw}
        \end{flushright}
        \end{quote}
    % section nacjonalizm (end)
% chapter 14_listopada_2011_textit (end)
\chapter{21 listopada 2011 -- \textit{Ziemie polskie na przełomie XIX i XX wieku}} % (fold)
\label{cha:21_listopada_2011_textit}
    \section{Sytuacja w zaborze pruskim} % (fold)
    \label{sec:sytuacja_w_zaborze_pruskim}
        Ludziom żyło się dobrze, zwłaszcza ekonomicznie. Były duże, nowoczesne gospodarstwa, produkujące na rynek. Przez to uwłaszczenie miało korzystny wpływ na rozwój rolnictwa. Również na Górnym Śląsku rozwinął się przemysł.

        Niestety źle się żyło pod względem kulturalnym -- \textbf{germanizacja za ,,żelaznego kanclerza''}. Bismarck walczył z Kościołem katolickim (\textbf{Kulturkampf}), odsuwano duchownych od szkolnictwa. Mimo starań Niemców, Polacy tym mocniej wierzyli i zaczynali kooperować z katolikami niemieckimi. Ta akcja Bismarckowi nie wyszła.

        \textbf{Rugi pruskie} -- wysiedlenie Polaków, nie będących obywatelami pruskimi (pochodzili z zaboru rosyjskiego, po powstaniach).

        W 1901 roku wystąpił protest dzieci i rodziców we Wrześni. Powstała ,,Rota'' Marii Konopnickiej.

        \textbf{Wóz Drzymały}

        \textbf{Solidaryzm narodowy} -- słaba działalność partii robotniczych i chłopskich, silna pozycja endecji.
    % section sytuacja_w_zaborze_pruskim (end)
    \section{Sytuacja w zaborze rosyjskim} % (fold)
    \label{sec:sytuacja_w_zaborze_rosyjskim}
        Mimo że chłopi dostali tam ziemię, by nie protestowali, nie byli zadowoleni, ponieważ pola były bardzo małe.

        \paragraph{Rusyfikacja}
        \begin{description}
            \item[nazwa] Zamiast Królestwo Polskie $\rightarrow$ \textbf{Kraj Przywiślański}
            \item[szkoły] Lekcje po rosyjsku, rozmowy na przerwach po rosyjsku
        \end{description}
    % section sytuacja_w_zaborze_rosyjskim (end)
    \section{Sytuacja w zaborze austriackim} % (fold)
    \label{sec:sytuacja_w_zaborze_austriackim}
        Bardzo ciężko się żyło, bieda. Jednak było dość dużo wolności. Rząd nie inwestował w te tereny. Obszar był ewentualnym mięsem armatnim dla Rosji. Do tego gospodarstwa były rozdrobnione (przez źle przeprowadzone uwłaszczenie w 1848 r.). Było też prawo propinacji dla ziemian, oraz silny antysemityzm.

        Galicja powoli uzyskiwała autonomię na różnych płaszczyznach. W 1861 roku powstał \textbf{Sejm Krajowy, Wydział Krajowy}. \textbf{Nie było cenzury.} Dzięki temu rozkwitła literatura oraz sztuka (Matejko, Chełmoński, Kossakowie, Wyspiański, Przybyszewski, Boy-żeliński). Urzędy oraz administracja były polonizowane. Byli nawet autorzy, obarczający winą za rozbiory naród polski -- zanarchizowany i zacofany, odrzucali walkę o niepodległość, a skłonność do konspiracji -- \textbf{liberum conspiro} -- uważali za tyleż zgubne, co liberum veto.
    % section sytuacja_w_zaborze_austriackim (end)
% chapter 21_listopada_2011_textit (end)
\chapter{23 listopada 2011 -- \textit{Polskie partie polityczne w XIX wieku}} % (fold)
\label{cha:23_listopada_2011_textit}
    \section{Przyczyny powstawania partii} % (fold)
    \label{sec:przyczyny_powstawania_partii}
        W zaborze rosyjskim praktycznie tworzenie partii było praktycznie niemożliwe. Życie społeczne było demokratyzowane. Pomiędzy klasami trwały lekkie ,,walki''. Generowało to niezadowolenie. Głównym, źródłowym problemem był brak niepodległości.
    % section przyczyny_powstawania_partii (end)
    \section{Partie socjalistyczne -- problemy robotników, proletariatu} % (fold)
    \label{sec:partie_socjalistyczne_problemy_robotnik_w_proletariatu}
        \subsection{Nurt internacjonalistyczny} % (fold)
        \label{sub:nurt_internacjonalistyczny}
            \subsubsection{Wielki Proletariat} % (fold)
            \label{ssub:wielki_proletariat}
                Założona w 1882 roku przez Ludwika Waryńskiego. Chcieli wprowadzić komunizm na świecie, postawić kwestie społeczne na pierwszym miejscu. \textbf{Kwestia niepodległości była pomijana} (po co tworzyć państwo polskie, wystarczy stworzyć jedno wielkie europejskie państwo), (program brukselski -- zakładana i wymyślana w Brukseli).

                Później jednak partia się rozbiła, a Waryński zmarł w więzieniu rosyjskim. Części zwały się \textbf{II Proletariat} oraz \textbf{Związek Robotników Polskich}.

                W 1893 roku powstała Socjaldemokracja Królestwa Polskiego, później przerodziła się w Socjaldemokrację Królestwa Polskiego i Litwy (1900 r.). Reprezentantami była Róża Luksemburg oraz Julian Marchlewski. Program jak wyżej, współpraca z bolszewikami.
            % subsubsection wielki_proletariat (end)
        % subsection nurt_internacjonalistyczny (end)
        \subsection{Nurt niepodległościowy} % (fold)
        \label{sub:nurt_niepodleg_o_ciowy}
            \subsubsection{PPS} % (fold)
            \label{ssub:pps}
                W 1893 roku, przedstawicielami Józef Piłsudski oraz Stanisław Wojciechowski. Niepodległość stawiali na pierwszym miejscu, później w programie przemiany socjalne (program paryski).
            % subsubsection pps (end)
            \subsubsection{PPSD} % (fold)
            \label{ssub:ppsd}
                Polska Partia Socjalno-Demokratyczna Galicji i Śląska Cieszyńskiego, w Galicji (opróćz niej PPS zaboru pruskiego).
            % subsubsection ppsd (end)
        % subsection nurt_niepodleg_o_ciowy (end)
    % section partie_socjalistyczne_problemy_robotnik_w_proletariatu (end)
    \section{Problemy związane z brakiem państwa, niepodległości, walka o przetrwanie narodu $\mathbf\rightarrow$ partie nacjonalistyczne} % (fold)
    \label{sec:problemy_zwi_zane_z_brakiem_pa_stwa_niepodleg_o_ci_walka_o_przetrwanie_narodu_mathbfrightarrow_partie_nacjonalistyczne}
        \subsubsection{Liga Polska} % (fold)
        \label{sub:liga_polska}
            Liga Polska (1887), Zygmunt Miłkowski, walka o niepodległość
        % subsubsection liga_polska (end)
        \subsubsection{Liga Narodowa} % (fold)
        \label{ssub:liga_narodowa}
            Założona w roku 1893, potoczną nazwą była ,,Narodowa Demokracja'' (endecja). Duży wkład Romana Dmowskiego, który uważał, że najistotniejsze są interesy narodu polskiego (\textbf{egoizm narodowy}). Chciał podniesienia poziomu kulturalnego i gospodarczego narodu polskiego, antysemityzm, walczyła przeciwko walce klasowej, która niestety poróżniała Polaków. Zwalczali komunistów. Nie skupiali się na walce z zaborcą.
        % subsubsection liga_narodowa (end)
    % section problemy_zwi_zane_z_brakiem_pa_stwa_niepodleg_o_ci_walka_o_przetrwanie_narodu_mathbfrightarrow_partie_nacjonalistyczne (end)
    \section{Problemy wsi $\mathbf{\rightarrow}$ partie chłopskie} % (fold)
    \label{sec:problemy_wsi_mathbf}
        Ważną rolę odegrał ks. Stanisław Stojałowski, wydawał czasopisma ,,Wieniec'' i ,,Pszczółka''. Gazety wychodziły w zaborze austriackim. Były to dwie gazety, ponieważ wydawali je naprzemiennie co 2 tygodnie, dzięki temu nie płacąc podatku.

        W Rzeszowie w 1895 roku stworzony \textbf{Stronnictwo Ludowe}. W 1903 roku \textbf{Polskie Stronnictwo Ludowe}, reprezentowane przez Jana Stapinskiego. Głosiło hasła ekonomiczne, tanie kredyty dla chłopów, uregulowanie serwitutów, powszechne prawo wyborcze.

        Później rozdzieliło się na:
        \begin{description}
             \item[PSL -- Lewicę] żądało dalszych radykalnych reform, Stapiński
             \item[PSL -- ,,Piast''] (od nazwy czasopism), umiarkowane, mniejsze żądania, Wincenty Witos
         \end{description}
    % section problemy_wsi_mathbf (end)
    \section{Kto się lubił, kto nie lubił} % (fold)
    \label{sec:kto_si_lubi_kto_nie_lubi_}
        \colorbox{SpringGreen}{NDecja i chłopskie partie -- lubili się, razem głosili chrześcijańskie wartości itd.}\\
        \colorbox{WildStrawberry}{Piłsudski vs. Dmowski -- walka.}
    % section kto_si_lubi_kto_nie_lubi_ (end)
% chapter 23_listopada_2011_textit (end)
\chapter{28 listopada 2011 -- \textit{Pierwsza wojna światowa c.d.}} % (fold)
\label{cha:pierwsza_wojna_wiatowa_c_d_}
    \section{Przystąpienie USA do wojny} % (fold)
    \label{sec:przyst_pienie_usa_do_wojny}
        Na początku prezydent Woodrow Wilson izolował się. Jednak po zatopieniu wielu amerykańskich statków przez Niemcy, w tym pasażerskiej Lusitanii, zdecydowali się na interwencję. Bali się gospodarczych konsekwencji -- spadek importu i eksportu. Przegrana ententy przyniosłyby ogromne straty firmom w USA.

        Niemcy, będąc cwane, chcąc odwieść USA od wojny rozmawiały z Meksykiem o planie napaści na USA. Robili to poprzez linię telegraficzną, którą Stany Zjednoczone bez problemu podsłuchały.

        Amerykanie nie posiadali armii lądowej, bo po co. Na lądzie zaszkodzić im mógł tylko Meksyk, który i tak był już dawno uciszony. Powolnie więc przygotowali armię lądową, by wysłać ją do Europy.
    % section przyst_pienie_usa_do_wojny (end)
    \section{Wpływ rewolucji w Rosji na bieg wojny} % (fold)
    \label{sec:wp_yw_rewolucji_w_rosji_na_bieg_wojny}
        Lenin pragnął zakończyć wojnę $\rightarrow$ podpisał pokój z państwami centralnymi w \textbf{Brześciu Litewskim -- 3. marca 1918 r.} Wtedy Rosja odeszła z wojny światowej. Dzięki temu Niemcy mogli przerzucić wszystkie wojska ze wschodu na front zachodni. Uderzyli zatem z wielką siłą na Francję i Paryż. Dzięki pomocy Stanów udało się odeprzeć atak. Później Hitler pomstował na ten rozejm jako na ,,\emph{cios w plecy}''.

        Rosja oddała Litwę, część Białorusi, Inflanty, Estonię oraz Królestwo Polskie pod okupację.

        Koncepcja ,,Mitteleuropy''.
    % section wp_yw_rewolucji_w_rosji_na_bieg_wojny (end)
    \section{Działania wojenne w roku 1918} % (fold)
    \label{sec:dzia_ania_wojenne_w_roku_1918}
        Ofensja trójprzymierza we Francji, podeszli pod Paryż. Nad Marną w lipcu udało się powstrzymać Niemców. USA w końcu ruszyli pełną parą z pomocą.

        Niemcy zaczęli się wycofywać z Francji i Belgii.
        \subsection{Powody przegrania wojny przez Niemcy} % (fold)
        \label{sub:powody_przegrania_wojny_przez_niemcy}
            \begin{description}
                \item[Kapitulacja Bułgarii i Turcji] -- Bułgaria we wrześniu, miesiąc później Turcja. Odsłonęło to Austro-Węgry od strony Bałkan i Morza Czarnego.
                \item[Rozpad Austro-Węgier] -- Na przełomie października i listopada niepodległość ogłaszają: Czechosłowacja, Galicja, Chorwacja (powstaje państwo SHS), Węgry. 3 listopada Austro-Węgry podpisują zawieszenie broni. Cesarz i król Karol I abdykuje.
                \item[Sytuacja w Niemczech] -- wraz z końcem października marynarze z Kilonii odmawiają walki. Wybuchł bunt poparty przez robotników, dzięki temu rozpoczęła się rewolucja socjaldemokratyczna (władzę przejmuje SPD). Wilhelm II abdykował, Niemcy stały się republiką. Nowy rząd podpisał zawiezsenie broni \textbf{11 listopada 1918 r. w Compiegne} (w wagonie kolejowym).

                Powstała legenda o ,,ciosie w plecy'', zadanym przez socjalistów niepokonanej armii niemieckiej.
            \end{description}
        % subsection powody_przegrania_wojny_przez_niemcy (end)
    % section dzia_ania_wojenne_w_roku_1918 (end)
    \section{Nowe rodzaje broni oraz środków} % (fold)
    \label{sec:nowe_rodzaje_broni_oraz_rodk_w}
        \begin{itemize}
            \item karabiny maszynowe
            \item granaty ręczne
            \item samoloty (na początku tylko do działań rozpoznawczych)
            \item sterowce
        \end{itemize}
    % section nowe_rodzaje_broni_oraz_rodk_w (end)
% chapter pierwsza_wojna_wiatowa_c_d_ (end)
\chapter{14 grudnia 2011 -- \textit{Rewolucja w Rosji}} % (fold)
\label{cha:rewolucja_w_rosji}
    \section{Tematy na sprawdzian} % (fold)
    \label{sec:tematy_na_sprawdzian}
        Numery tematów z podręcznika: 20, 21, 22, 26, 27, 28.
    % section tematy_na_sprawdzian (end)
    \section{Przyczyny wybuchu rewolucji w 1917 roku} % (fold)
    \label{sec:przyczyny_wybuchu_rewolucji_w_1917_roku}
        Klęski w I wojnie światowej zachwiały wiarą w państwo i cara. Rosja straciła wiele ziem i wiele ludzi. Przyszedł kryzys gospodarczy, głód na wsi z braku mężczyzn, którzy poszli do wojska\dots Poza tym był Rasputin, przez którego spadł poziom zaufania do rodziny carskiej. Był on intrygantem, rozpustnikiem, rzekomo uzdrowił syna cara. W końcu pojmali go i chcieli zabić, jednak on się nie dawał. Czego nie próbowali, on ,,odżywał''. W końcu im się udało, oczywiście nie przysporzyło to wielbicieli carskiej rodzinie.

        Ludzie żądali demokratycznych i socjalnych reform. Jednak car był niezdecydowany, słaby i bierny wobec problemów kraju. W Dumie, armii urzędach pojawiały się spiski antycarskie.
    % section przyczyny_wybuchu_rewolucji_w_1917_roku (end)
    \section{Wybuch i przebieg rewolucji lutowej} % (fold)
    \label{sec:wybuch_i_przebieg_rewolucji_lutowej}
        \subsection{I faza -- próba budowy republiki} % (fold)
        \label{sub:i_faza_pr_ba_budowy_republiki}
            Wojsko poparło demonstrantów, krórzy na wiosnę 1917 roku wyszli na ulice Piotrogrodu. Wszystko przeszło mniej więcej bez krwawych walk, car zmuszony został do abdykacji, władzę przejął Rząd Tymczasowy złożony z robotników i Dumy. Premierem został Georgij Lwow. Później premierem został Aleksander Kiereński. Niestety dla nich zaplanowano wybory powszechne na listopad 1917 r. To pół roku mogli wykorzystać bolszewicy, którzy zgromadzili poparcie.

            Lenin zaczął dużo obiecywać, przy pomocy Niemców przybył ze Szwajcarii. Jego obiecanki zwane były tezami kwietniowymi, m. in. głosił, że cała władza zostanie oddana robotnikom. Powiedział, że zakończy wojnę i nada ziemię chłopom. Tych trzech rzeczy wszyscy chcieli i dlatego poszli za nim.
        % subsection i_faza_pr_ba_budowy_republiki (end)
        \subsection{II faza -- radykalna} % (fold)
        \label{sub:ii_faza_radykalna}
            Lenin nakłonił w Piotrogrodzie część wojska do przejścia na stronę komunistów. 7 listopada 1917 roku bolszewicy opanowali Pałac Zimowy, czyli siedzibę Rządu Tymczasowego. Kiereński uciekł. Lenin doprowadził do pokoju z państwami centralnymi -- 3 marca 1918 w Brześciu.

            Komuniści ponieśli klęskę w wyborach do konstytuanty\dots
        % subsection ii_faza_radykalna (end)
        \subsection{III faza -- dyktatura} % (fold)
        \label{sub:iii_faza_dyktatura}
            
        % subsection iii_faza_dyktatura (end)
    % section wybuch_i_przebieg_rewolucji_lutowej (end)
% chapter rewolucja_w_rosji (end)
\chapter{19 grudnia 2011 -- \textit{Sprawa polska w trakcie I wojny}} % (fold)
\label{cha:sprawa_polska_w_trakcie_i_wojny}
    \section{Orientacje polityczne na ziemiach polskich} % (fold)
    \label{sec:orientacje_polityczne_na_ziemiach_polskich}
        Polacy spodziewając się zbrojnego konfliktu wiążą się z poszczególnymi państwami ościennymi w celu uzyskania korzyści. Wydarzenia w ,,\emph{kotle bałkańskim}'' mogłyby rozpocząć wojnę (aneksja Bośni i Hercegowiny w 1908 r., wojny bałkańskie).
        \subsection{Orientacja prorosyjska (pasywistyczna)} % (fold)
        \label{sub:orientacja_prorosyjska}
            Zwolennikiem był Roman Dmowski, NDecja. Liczył na wygraną Rosji i wcielenie przez nią reszty ziem polskich oraz nadanie szerokiej autonomii. Rosja ulegała powoli demokratyzacji, poza tym łączono nas z Rosją jako słowiańskim bratem. Głównym argumentem Dmowskiego było, że rusyfikacja nie jest taka groźna jak germanizacja. Było tak, ponieważ w Niemczech germanizacja była tak przestrzegana, tak uporządkowana, że była skuteczna.
        % subsection orientacja_prorosyjska (end)
        \subsection{Orientacja proaustriacka (aktywistyczna)} % (fold)
        \label{sub:orientacja_proaustriacka_}
            Zwolennikami byli politycy galicyjscy, lojaliści, po trochu Piłsudski. Liczyli na wygraną Austro-Węgier, głównym wrogiem była Rosja. W Austrii upatrywano najłagodniejszego zaborcy, ponieważ mieliśmy tam największą autonoimę. Liczono nawet na stworzenie trialistycznej monarchii Austro-Polsko-Węgier.
        % subsection orientacja_proaustriacka_ (end)
        \subsection{Orientacja proniemiecka} % (fold)
        \label{sub:orientacja_proniemiecka}
            Nie było takiej. Twardo stali na stanowisku, że nie ma szans na wolną Polskę.
        % subsection orientacja_proniemiecka (end)
        \subsection{Kierunek powstańczy} % (fold)
        \label{sub:kierunek_powsta_czy}
            Zwolennikiem był Piłsudski, Kazimierz Sosnkowski, Marian Kukiel. Celem było wywołanie powstania w zaborze rosyjskim przy pomocy Austro-Węgier, a następnie stworzenie niepodległego państwa w momencie osłabienia walczących krajów.
        % subsection kierunek_powsta_czy (end)
        \subsection{Kierynek socjalistyczn-rewolucyjny} % (fold)
        \label{sub:kierynek_socjalistyczn_rewolucyjny}
            SDKPiL, PPS-Lewica $\rightarrow$ od 1918 roku KPRP.\\
            Odczucali hasła walki o niepodległość, dążyli do internacjonalistycznej rewolucji.
        % subsection kierynek_socjalistyczn_rewolucyjny (end)
    % section orientacje_polityczne_na_ziemiach_polskich (end)
    \section{Działalność Piłsudskiego 1914 -- 1915} % (fold)
    \label{sec:dzia_alno_pi_sudzkiego_1914_1915}
        Wybuch wojny i początek tworzenia armii polskiej u boku Austro-Węgier. W Krakowie powstała \textbf{1 Kompania Kadrowa} (145 -- 168 osób). Ruszyli w kierunki Kielc, próbując wywołać powstanie.

        Powstał pomysł utworzenia \textbf{Legionów Polskich} 16 sierpnia 1914 roku. Wyszedł od Juliusza Leo. Piłsudski pomaga w utworzeniu, tymczasem nadal myśląc o powstaniu powołał tajną \textbf{POW} -- \textbf{Polską Organizację Wojskową}.

        \paragraph{Jaki błąd popełnił Piłsudski wiążąc się z państwami centralnymi?} Obrał złego sprzymierzeńca. Ale co by było, gdyby Piłsudski utworzył Legiony u boku Rosji? Ano prawdopodobnie zostałyby rozwiązane.
        \paragraph{Dlaczego w historiografii polskiej umniejszono rolę endecji?} Ponieważ Piłsudski później doszedł do władzy.
        \paragraph{Maciejówka} -- nowy typ czapki, okrągła, nosili ją robotnicy i chłopi. Została wprowadzona do armii za czasów Piłsudskiego. W ten sposób okazał sympatię tym grupom ludzi, chcąc ich wciągnąć do armii.
    % section dzia_alno_pi_sudzkiego_1914_1915 (end)
    \section{Działania państw centralnych} % (fold)
    \label{sec:dzia_ania_pa_stw_centralnych}
        \subsection{Akt 5 listopada 1916 r.} % (fold)
        \label{sub:akt_5_listopada_1916_r_}
            Był to manifest dwóch cesarzy. Zapowiedziano utworzenie wojska polskiego -- Polnische Wermacht. Obiecano powstanie Królestwa Polskiego (granice wielką niewiadomą) w sojuszu z państwami centralnymi.

            W lipcu 1917 roku Piłsudski namówił żołnierzy do odmowy złożenia przysięgi. Państwa centralne za to rozwiązały I i III Brygadę, a żołnierzy internowali w Beniaminowie i Szczypiornie (stąd nazwa ,,szczypiorniści'').

            We wrześniu 1917 powołano Radę Regencyjną, zapowiedź utworzenia królestwa. Powstał rząd polski.

            Niemcy zezwalają przejąć Ukrainie Chełmszczyznę i część Podlasia.
        % subsection akt_5_listopada_1916_r_ (end)
    % section dzia_ania_pa_stw_centralnych (end)
% chapter sprawa_polska_w_trakcie_i_wojny (end)
\chapter{2 stycznia 2012 -- Sprawa polska w trakcie I wojny cd.} % (fold)
\label{cha:2_stycznia_2012_sprawa_polska_w_trakcie_i_wojny_cd_}
    25 grudnia 1916 roku car Mikołaj II wygłosił orędzie o stworzeniu Polski z ziem zabranych z własną władzą ustawodawczą i wojskiem.

    15 listopada 1917 r. bolszewicy wydali dekret po obaleniu Rządu Tymczasowego, mówiący o czymś, czego nie zdążyłem zapisać.

    29 sierpnia 1918 roku -- dekret bolszewickiej \textbf{Rady Komisarzy Ludowych} o anulowaniu traktatów rozbiorowych (ziemie polskie i tak były pod okupacją państw centralnych). Zero pokrycia, czysta propaganda.

    \section{Działania ententy na zachodzie} % (fold)
    \label{sec:dzia_ania_ententy_na_zachodzie}
        U boku Francji walczy na początku wojny kampania bajończyków (od miasta Bayonne) $\rightarrow$ uległa rozbiciu. W sierpniu 1917 roku Dmowski założył \textbf{Komitet Narodowy Polski}, działający w Paryżu, uznawany za rząd polski. W połowie 1917 roku powstała tzw. \textbf{błękitna armia} (od koloru mundurów), później dowodzona przez Hallera.

        Po 30 marca 1917 roku zdecydowano o stworzeniu trzech polskich korpusów.

        8 stycznia roku następnego Woodrow Wilson wygłosił orędzie, bardzo ważne orędzie. W punkcie 13 powiedział o konieczności powstania niepodległej Polski w granicach etnicznych z dostępem do morza.

        Ignacy Paderewski przed każdym występem fortepianowym opowiadał o Polsce, dzięki czemu nagłośnił polską sprawę w USA.
    % section dzia_ania_ententy_na_zachodzie (end)
% chapter 2_stycznia_2012_sprawa_polska_w_trakcie_i_wojny_cd_ (end)
\chapter{4 stycznia 2012 -- \textit{System wersalski}} % (fold)
\label{cha:4_stycznia_2012_textit}
    \section{Konferencja pokojowa w Paryżu} % (fold)
    \label{sec:konferencja_pokojowa_w_pary_u}
        Symboliczne jej daty to 18.01 -- 28.06.1919. Przewodniczącymi Rady Najwyższej byli ministrowie spraw zewnętrznych i szefowie mocarstw USA, Anglii, Francji, Włoch oraz Japonii. Najważniejsza była tzw. ,,Wielka czwórka''. Nie było reprezentacji Rosji (z powodu separatystycznego pokoju oraz widma bolszewizmu). Odmienne zdania Francji i Anglii wobec Polski i Niemiec. Francja chciała nam dać dużo swobody, żebyśmy byli duzi i mocni, ale Anglia miała bzika na punkcie ,,balance of power'' i nie chciała byśmy byli za mocni.

        Niemcy podpisały traktat wersalski 28 czerwca 1919 roku.
        \subsection{Postanowienia wobec Niemiec} % (fold)
        \label{sub:postanowienia_wobec_niemiec}
            \subsubsection{Terytorialne} % (fold)
            \label{ssub:terytorialne}
                \begin{itemize}
                    \item utrata wszystkich kolonii,
                    \item Alzacja i Lotaryngia wcielona do Francji,
                    \item okręgi Eupen, Malmedy i Moresnet do Belgii,
                    \item Wielkopolska i cz. Pomorza Gdańskiego do Polski,
                    \item Gdańsk i Kłajpeda -- wolne miasta pod zarządem Ligi
                    \item na Górnym Śląsku, Mazurach, Hulczynie (Czechosłowacja) przeprowadzono plebiscyty,
                    \item zakazano łączyć się z Austrią
                \end{itemize}
            % subsubsection terytorialne (end)
            \subsubsection{Wojskowe} % (fold)
            \label{ssub:wojskowe}
                \begin{itemize}
                    \item armia tylko zawodowa (100 tys.),
                    \item internowanie floty niemieckiej i jej zatopienie przez Niemców
                \end{itemize}
            % subsubsection wojskowe (end)
            \subsubsection{Ekonomiczne} % (fold)
            \label{ssub:ekonomiczne}
                \begin{itemize}
                    \item nałożenie reparacji wojennych,
                    \item 20 mld marek w złocie do maja 1921 roku
                \end{itemize}
            % subsubsection ekonomiczne (end)
            \subsubsection{,,Ukaranie'' zbrodniarzy wojennych} % (fold)
            \label{ssub:_ukaranie_zbrodniarzy_wojennych}
                \begin{itemize}
                    \item sądy niemieckie wydały wyroki na kilka miesięcy więzienia,
                    \item cesarz Wilhelm II schronił się w Holandii (ku uciesze ententy)
                \end{itemize}
            % subsubsection _ukaranie_zbrodniarzy_wojennych (end)
        % subsection postanowienia_wobec_niemiec (end)
    % section konferencja_pokojowa_w_pary_u (end)
% chapter 4_stycznia_2012_textit (end)
\chapter{9 stycznia 2012 -- \textit{Początki państwa polskiego}} % (fold)
\label{cha:9_stycznia_2012_textit}
    \section{Sytuacja międzynarodowa} % (fold)
    \label{sec:sytuacja_mi_dzynarodowa}
        \begin{itemize}
            \item Rosja -- wojna domowa
            \item Niemcy -- rewolucja, upadek cesarstwa\dots
            \item Austro-Węgry -- rozkład monarchii w październiku 1918 r.
            \item Ententa -- uznała konieczność powstania wolnej Polski (orędzie Wilsona)
        \end{itemize}
    % section sytuacja_mi_dzynarodowa (end)
    \section{Sytuacja na ziemiach polskich -- ośrodki władzy} % (fold)
    \label{sec:sytuacja_na_ziemiach_polskich_o_rodki_w_adzy}
        \begin{description}
            \item[Polska Komisja Likwidacyjna] -- przejęła władzę w Krakowie, a 31.10.1918 spiskowcy rozbroili wojska austriackie (por. Antoni Stawarz)
            \item[Rada Regencyjna] -- powołana i popierana przez rządy Austrii i Niemiec. Nie ufano jej. 
            \item[Tymczasowy Rząd Ludowy Republiki Polskiej w Lublinie] -- Powstał w wyzwolonym przez POW Lublinie.
            \item[Naczelna Rada Ludowa w Poznaniu] -- zaplecze endecji i chadecji.
            \item[Komitet Narodowy Polski w Paryżu] -- pod egidą endecji.
        \end{description}
        \subsection{Przejęcie władzy przez Józefa Piłsudskiego} % (fold)
        \label{sub:przej_cie_w_adzy_przez_j_zefa_pi_sudskiego}
            Po uwięzieniu go w Magdeburgu stał się bohaterem, nikt już o nim, że spiskuje. Wkrótce zwolniono Piłsudskiego z więzienia w Magdeburgu. Wrócił do Warszawy 10 listopada 1918 roku. Rada Regencyjna szybko zrezygnowała z władzy nad wojskiem i przekazała mu ją -- \textbf{11 listopada 1918 roku}. Udało się rozbroić Niemców (akcja zbiegła się z rozejmem w Compiegne). Rozwiązano Radę Regencyjną, przekazano Piłsudskiemu władzę cywilną. Powołał rząd z Jędrzejem Moraczewskim na czele, powierzył on Piłsudskiemu funkcję \textbf{Tymczasowego Naczelnika Państwa}, dającą niemal dyktatorską władzę. Nowej rzeczywistości nie uznawali komuniści -- KPRP -- powstała w 1918 roku z połączenia PPS-Lewicy i SDKPiL.
        % subsection przej_cie_w_adzy_przez_j_zefa_pi_sudskiego (end)
    % section sytuacja_na_ziemiach_polskich_o_rodki_w_adzy (end)
    \section{Powstanie wielkopolskie} % (fold)
    \label{sec:powstanie_wielkopolskie}
        Wynik powstańczych planów POW, Sokoła i harcerzy.

        Paderewski przybył do Poznania w grudniu 1918 roku -- demonstracje Polaków. Praktycznie było świetnie przygotowane, powstańcy byli dobrze uzbrojeni, mundury, wszystko. Walki uliczne z Niemcami przeradzają się w zorganizowane powstanie, które ogarnia cały obszar Wielkopolski (dow. Józef Dowbor-Muśnicki).

        16.02.1919 rozejm w Trewirze.
    % section powstanie_wielkopolskie (end)
    \section{Decyzje Ententy w sprawie granic Polski} % (fold)
    \label{sec:decyzje_ententy_w_sprawie_granic_polski}
        Brak decyzji w sprawie granic na wschodzie, tzw. ,,otwarte drzwi''. Co sobie zdobędziemy, to nasze. Anglia trzymała dystans wobec nas. Stworzono projekt plebiscytów. Wielkopolska z Bydgoszczem i Pomorzem Grańskim wejdą w skład Polski, a Gdańsk stałby się wolnym miastem pod zarządem Ligi Narodów.
    % section decyzje_ententy_w_sprawie_granic_polski (end)
% chapter 9_stycznia_2012_textit (end)
\chapter{11 stycznia 2012 -- \textit{Odbudowa Polski cd.}} % (fold)
\label{cha:11_stycznia_2012_textit}
    \section{Powstania śląskie} % (fold)
    \label{sec:powstania_l_skie}
        Śląsk włączył do Polski Mieszko I. Później Kazimierz Wielki zrzekł się go w XIV wieku. Dopiero po pierwszej wojnie światowej zaczęliśmy się o niego ubiegać.
        \subsection{I powstanie śląskie} % (fold)
        \label{sub:i_powstanie_l_skie}
            Wybuchło w sierpniu 1919 roku. Spontanicznie, na skutek zwalniania przez niemieckich przedsiębiorców polskich pracowników, upadło po tygodniu walk. Ententa zdecydowała o usunięciu wojsk niemieckich ze Śląska. Wmaszerowały tam i pilnowały porządku.
        % subsection i_powstanie_l_skie (end)

        Tymczasem Niemcy i Polacy potajemnie dozbrajali swoje organizacje na Śląsku.
        \subsection{II powstanie śląskie} % (fold)
        \label{sub:ii_powstanie_l_skie}
            Niemcy nasilali powoli ataki na Polaków, korzystali z obecności bolszewików pod Warszawą. Znów trwało tylko tydzień, a Polacy wywalczyli utworzenie niemiecko-polskiej policji.
        % subsection ii_powstanie_l_skie (end)
        \subsection{Plebiscyt na Śląsku} % (fold)
        \label{sub:plebiscyt_na_l_sku}
            20 marca 1921 roku strona polska przegrała. Np. dlatego, że udział wzięli Niemcy przybyli z zachodu.
        % subsection plebiscyt_na_l_sku (end)
        \subsection{III powstanie śląskie} % (fold)
        \label{sub:iii_powstanie_l_skie}
            Odbyło się w okresie \textit{maj -- czerwiec 1921}. Dowódcą wojsk był Maciej Mielżyński, a całością powstania Wojciech Korfanty. Potajemnie dostali pomoc od Polaków. Powstańcom początkowo dobrze szło, doszli do Odry. W końcu passa się skończyła i przegraliśmy w boju o Górę Św. Anny. Jednak ententa pomogła nam i zagroziła Niemcom wojną.
        % subsection iii_powstanie_l_skie (end)
        \subsection{Podział Górnego Śląska} % (fold)
        \label{sub:podzia_g_rnego_l_ska}
            W październiku 1921 roku dostaliśmy 29\% terenów plebiscytowych z miastami: Chorzów, Katowice, Mysłowice, Tychy, Pszczyna, Rybnik. Więc dostaliśmy większość hut, kopalń i zakładów.
        % subsection podzia_g_rnego_l_ska (end)
    % section powstania_l_skie (end)
    \section{Granica północna} % (fold)
    \label{sec:granica_p_nocna}
        Postanowiono zrobić plebiscyt na północ od Gdańska, w Prusach Wschodnich. Natomiast w Wolnym Mieście Gdańsku byliśmy troszku szykanowani i nieszanowani, było nas za mało. Tak samo plebiscyt źle poszedł. Tylko 3 gminy dostały dołączone do Polski. Dodatkowo problem był taki, że bolszewicy będąc pod Warszawą rzucali cień na losy Polski. Gdyby upadł dostaliby się pod władzę bolszewików. Zatem wszyscy głosowali za Niemcami, stabilnymi itd.
    % section granica_p_nocna (end)
    \section{Granica południowa i południowo-wschodnia} % (fold)
    \label{sec:granica_po_udniowa_i_po_udniowo_wschodnia}
        W listopadzie 1918 ustalono linię demarkacyjną na Śląsku Cieszyńskim między Polską, a Czechosłowacją według kryterium etnicznego. Planowano zrobić plebiscyt, ale wojska czeskie zerwały porozumienie i zajęli większość tegoż wyżej wspomnianego Śląska.

        Na mocy decyzji Rady Ambasadorów ententy 28 lipca 1920 roku Czechosłowacja otrzymała:
        \begin{itemize}
            \item Zaolzie (teren za rzeką Olzą), zamieszkiwany przez 100 tysięcy Polaków.
            \item I coś jeszcze.
        \end{itemize}

        W 1923 roku Galicja Wschodnia została przyznana Polsce. Wcześniej, do 1919 zajęły ją wojska polskie, usunąwszy wojska Zachodnioukraińskiej Republiki Ludowej.

        Rok 1923 uznawany jest za koniec gmerania przy granicach.
    % section granica_po_udniowa_i_po_udniowo_wschodnia (end)
% chapter 11_stycznia_2012_textit (end)
\chapter{18 stycznia 2012 -- \textit{Walka o wschodnią granicę cd.}} % (fold)
\label{cha:18_stycznia_2012_textit}
    \section{Bolszewicy pod Warszawą} % (fold)
    \label{sec:bolszewicy_pod_warszaw_}
        Położenie militarne Polski było bardzo niekorzystne.

        Premier Władysław Grabski poprowadził rozmowy z ententą w Spa (Belgii). Lloyd Georg zadządał dla siebie możliwości mediacji z bolszewikami oraz uznania granicy wschodniej Polski na Bugu (linia Curzona). Pod naciskiem Francji Polska pogodziła się z utratą Zaolzia.
    % section bolszewicy_pod_warszaw_ (end)
    \section{Nowy rząd} % (fold)
    \label{sec:nowy_rz_d}
        Powołano nowy rząd -- Rząd Obrony Narodowej -- premierem został Wincenty Witos z PSL ,,Piast''.
    % section nowy_rz_d (end)
    \section{Bitwa} % (fold)
    \label{sec:bitwa}
        Rosjanie planowali jeszcze raz zastosować manewr Paskiewicza, jednak polscy matematycy złamali szyfr i wiedzieliśmy. Generał Edward Rydz-Śmigły nad rzeką Wieprz szykował uderzenie z zaskoczenia na południowe skrzydło bolszewików.

        Bitwa rozegrała się w dniach 15 -- 16 sierpnia 1920 roku. Najcięższe boje na przedpolach stolicy rozegrały się pod Radzyminem. Ważna była kontrofensywa znad Wieprza. Była to jedna z bardziej decydujących bitew w dziejach świata.

        Bohaterem bitwy stał się ksiądz Skorupka, który wywołał impuls porywający żołnierzy do ataku.

        Dobiliśmy armię Rosjan w bitwie nad Niemnem we wrześniu 1920. Rozbiliśmy armię Budionnego pod Lwowem pod Zadwórzem (polskie Termopile, 300 polskich żołnierzy).

        \paragraph{Pokój} ryski, 18 marca 1921. Granicę ustalono na linii rzek Dźwina-Zbrucz. Zrezygnowano z wcielenia Białorusi do Polski, ponieważ endecja się nie zgodziła -- Białoruś nie zagłosowałaby na endecję w wyborach. Bolszewicy obiecali wypłacić odszkodowania i zwrócić dzieła sztuki (praktycznie nic nie dostaliśmy). Rozwiązano oddziały białoruskie i ukraińskie walczące przy boku Polski.
    % section bitwa (end)
% chapter 18_stycznia_2012_textit (end)
\chapter{23 stycznia 2012 -- \textit{Ciąg dalszy}} % (fold)
\label{cha:23_stycznia_2012_textit}
    \section{Zajęcie Wilna przez Polskę} % (fold)
    \label{sec:zaj_cie_wilna_}
        Nie można było zająć Wilna siłą, bo ententa by zobaczyła, że Polska jest agresorem i nic nie szanuje. Za namową Piłsudskiego gen. Lucjan Żeligowski pozoruje bunt i rusza z wojskiem na Wilno -- zajmuje je w październiku 1920 roku. Następnie powołał państwo Litwa Środkowa. Zrobiono wybory, a wybrany sejm zadecydował, że chce być włączony do Polski. I wszystko było w porządku (nie dla Litwinów). Zatem wypowiedziała Polsce wojnę.
    % section zaj_cie_wilna_ (end)
% chapter 23_stycznia_2012_textit (end)
\chapter{23 stycznia 2012 -- \textit{Pierwsze lata II RP}} % (fold)
\label{cha:23_stycznia_2012_textit}
    \section{Ukształtowanie się władz centralnych} % (fold)
    \label{sec:ukszta_towanie_si_w_adz_centralnych}
        Wybory do Sejmu Ustawodawczego 26 stycznia 1919 r. Wielki sukces w wyborach odniosła prawica oraz ludowcy, wygrała prawica z endecją -- $42.7\%$. KPRP zbojkotowała wybory.
    % section ukszta_towanie_si_w_adz_centralnych (end)
    \section{Analiza Konstytucji marcowej z 17 marca 1921 r.} % (fold)
    \label{sec:analiza_konstytucji_marcowej_z_17_marca_1921_r_}
        Prezydent miał mieć słaby wpływ na rządu, nie miał prawa weta. Rząd i premier mieli mieć inicjatywę ustawodawczą, jednak podporządkowani mieli być parlamentowi (sejmowi i senatowi), miało to zagwarantować przewagę nad władzą wykonawczą.

        Wybory prezydenckie odbyły się 9 grudnia 1922. Wybierał sejm i senat jako Zgromadzenie Narodowe. Nie stanął do nich Józef Piłsudski. Wygrał je Gabriel Narutowicz. Endecja i chadecja atakowały go za kosmopolityzm, ateizm, masonerię itp. Dlatego Eligiusz Niewiadomski zabił go w zamachu.

        Dlatego 20 grudnia prezydentem został Stanisław Wojciechowski (współtwórca PPS, dawny współpracownik Piłsudskiego). Zaakceptowała go większość partii.
    % section analiza_konstytucji_marcowej_z_17_marca_1921_r_ (end)
% chapter 23_stycznia_2012_textit (end)
\chapter{25 stycznia 2012 -- \textit{Problemy II RP}} % (fold)
\label{cha:25_stycznia_2012_textit}
    \begin{quote}
        \begin{center}
            \bf
            \textcolor{WildStrawberry}{,,Przewrót majowy'' do przeczytania na środę.}
        \end{center}
    \end{quote}
    \section{Problemy etniczno-religijne} % (fold)
    \label{sec:problemy_etniczno_religijne}
        \subsection{Struktura zawodowa} % (fold)
        \label{sub:struktura_zawodowa}
            Państwo było rolnicze, $55\%$ ludzi to byli rolnicy. Gospodarka się nie kręciła, bo większość z nich produkowała dla siebie, nie na rynek. $27\%$ stanowili robotnicy, $11\%$ drobnomieszczaństwo, inteligencja $5.5\%$ (bardzo bardzo mało). $1\%$ to burżuazja. Połowę z burżuazji stanowiło ziemiaństwo.
        % subsection struktura_zawodowa (end)
        \subsection{Narodowość} % (fold)
        \label{sub:narodowo_}
            Polacy stanowili niecałe $70\%$. Na Wołyniu liczba ta stanowiła nawet $16.7\%$. Jest to dość mała liczba. Największą miejszością byli Ukraińcy ($14\%$), Żydzi stanowili $8\%$. Połowę z tego Białorusini, tyle samo Niemcy. Litwini wraz z innymi mniejszościami stanowili pozostały $1\%$.
        % subsection narodowo_ (end)
        \subsection{Religie} % (fold)
        \label{sub:religie}
            \paragraph{Grekokatolicy} Unia prawosławia i rzymskokatolicyzmu.
            \paragraph{Żydzi} Polska była największym skupiskiem Żydów w Europie. Zamieszkiwali głównie miasta i miasteczka byłego zaboru rosyjskiego i austriackiego. Używali języka polskiego i jidysz. Przed wojną w polsce był mocny antysemityzm. Bardzo jasnym przykładem jest \emph{numerus clausus} na uczelniach. Była to zamknięta liczba ile Żydów można przyjąć na uczelnię.
            \paragraph{Łemkowie, Bojkowie i Huculi} byli popierani przez rząd polski dla osłabienia Ukraińców.
                \subparagraph{Łemkowie} w dużej mierze ludzie grekokatoliccy, pochodzenia rusko-wołoskiego. Część z nich powróciła do prawosławia w 1926 r. Zamieszkiwali tereny od Pienin (Beskid Niski i Sądecki) po Bieszczady. Zostali usunięci ze swych siedzib za pomoc dla UPA (Ukraińska Powstańcza Armia) w 1947 roku w ramach akcji ,,Wisła'' i przeniesieni na ziemie zabrane Niemcom.
        % subsection religie (end)
    % section problemy_etniczno_religijne (end)
% chapter 25_stycznia_2012_textit (end)
\chapter{5 marca 2012 -- \textit{Faszyzm we Włoszech}}
\section{,,Wybuch demokracji'' w Europie po I wojnie światowej}
	Monarchie upadały, powstawały republiki oraz silna władza ustawodawcza. Społeczeństwa jednak były słabo przygotowane do demokracji, było mało wykształcone, dużo kryzysów ekonomicznych, bezrobocia, poczucie klęski. Wywołało to \textbf{rozczarowanie demokracją}. Działo się tak dlatego, że demokracja funkcjonuje dobrze dopiero po jakimś czasie. A czego ludzie szukają, gdy jest im źle? Sposobu na zmianę.
	\subsection{Sytuacja we Włoszech}
		Po wojnie Włochy miały bardzo małe korzyści. Marzyły o potężnych koloniach w Afryce, a tu tylko Istria, Pd. Tyrol i Fiume. Dodatkowo przyszedł kryzys gospodarczy -- inflacja, bezrobocie (zwolnieni żołnierze).

		Wojna zabrała ze sobą też dużo pieniędzy oraz ludzi.

	\subsection{Narodziny faszyzmu}
		Powodem były słabe rządy, partie kanapowe oraz korupcja. Pojawiła się w tym momencie silna partia komunistów, \textbf{Włoska Partia Komunistyczna}. Komuniści byli słabi w Polsce, ponieważ kojarzył się z niedobrymi rzeczami.

		Korzenie faszyzmu leżą w wojskowo-kombatanckim Związku Weteranów (Fasci di Combattiamento) $\Rightarrow$ Narodowa Partia Faszystowska z Benito Mussolinim na czele.

		Połączono hasła lewicowe z nacjonalistycznymi, głoszono chęć wskrzeszenia imperium, konfiskaty dóbr Kościoła, militaryzację społeczeństwa. Czarne koszule atakowały komunistów -- swoich największych konkurentów.

		\subsubsection{Przejęcie władzy}
			Mussolini przeszedł na pozycje bardziej prawicowe, wyprowadzono kraj z równowagi (strajki, walki faszystów z komunistami)\dots Następnie 40 tysięcy ,,czarnych koszul'' wyruszyło na Rzym.

			W 1922 roku Mussolini został desygnowany na premiera przez Króla. Pełną władzę przejął w roku 1925.
		\subsubsection{,,Reformy'' faszystów}
			Próbowano stworzenie systemu korporacji pracowników i pracodawców, państwo stało się arbitrem. Wprowadzono w życie reformy socjalne (urlopy, prace publiczne itd.), unowocześniono armię, co napędzało przemysł i zmniejszało bezrobocie. Zdelegalizowano partie polityczne, chciano wprowadzić ideę \textbf{autarchii} -- samowystarczalności. Mussolini, przywódca partii przyznał sobie tytuł ,,duce'' (włoski odpowiednik F\"{o}hrera). Zachowano mimo to monarchię (monarchia jednoczyła Włochy).

			Wprowadzono cenzurę oraz tajną policję -- dawało się ludziom we znaki.
			
			Państwo faszystowskie podpisało umowę z Watykanem. (Wcześniej Państwo Kościelne przestało istnieć, a papież ogłosił się więźniem Watykanu.) Dzięki traktatom laterańskim wszystko się wyjaśniło, oficjalnie powstało Państwo Kościelne, granice zostały wytyczone itd.

			We Włoszech był niski poziom rasizmu oraz antysemityzmu.
\chapter{7 marca 2012 -- \textit{Państwo Hitlera -- film}}
\chapter{19 marca 2012 -- \textit{Kampania wrześniowa}}

\section{Polski plan wojny}
Z wiary w pomoc aliantów nie utrzymywaliśmy armii na wschodzie. Taktyka defensywna, wyczekiwanie na ofensywę na zachodzie. Po 15 dniach wojny Francja i Anglia miały dołączyć do wojny

Mieliśmy plan B, jeśli nie utrzymywalibyśmy się, wycofalibyśmy się do ,,przedmościa rumuńskiego'', obok Dniestru.

Flota polska została wycofana do Anglii. Mieliśmy dobre czołgi o dużym kalibrze, niestety mało. Samoloty polskie były przestarzałe. Inwestowano za dużo w marynarkę.

\section{Plan niemiecki}
\textbf{Blitzkrieg} -- wojna błyskawiczna, Niemcy wciskali się w luki pomiędzy naszymi oddziałami.

Wyposażeni byli w nowoczesne samoloty, wyrzucające bomby pikując. Wyposażone w syreny akustyczne.

\section{Przebieg działań wojennych}
\subsection{Bitwa graniczna w pierwszych dniach września}
1. września pancernik Schleswig-Holstein zaatakował Westerplatte, które skapitulowało 7. września. Broniliśmy Poczty Polskiej, atakowano linie telefoniczne, kolejowe, mosty\dots Pod Mokrą Wołyńska Kawaleria pokonała jednostkę pancerną.

6. czerwca armia Kraków się wycofała.

\subsection{Bitwa pod Wizną}
Pod Wizną, nad Narwią, pod dowództwem kapitana Władysława Raginisa przez 4 dni zatrzymywano grupę pancerną generała Guderiana. Bitwa zwana polskimi Termopilami.

\subsection{Obrona Warszawy}
8. września niemieckie kolumny dotarły pod Warszawę. Jako jedna z niewielu stolic europejskich broniliśmy.

\subsection{Bitwa nad Bzurą}
Podczas gdy armia niemiecka pruła na Warszawę, my zadaliśmy im duże straty na tyłach.

\subsection{Wkroczenie ZSRR na teren RP}
Mówił, że ,,bierze pod opiekę ludność''. Rydz-Śmigły wydał rozkaz, by armia wycofywała się do Rumunii, unikała z nimi walki.

17/18 września prezydent, premier i wódz naczelny opuścili kraj -- udali się do Rumunii z myślą o przerzuceniu armii do Francji. Niepotrzebnie zrobił to wódz naczelny. Szybko powrócił później jako zwykly człowiek do Polski.

\section{Zachowanie sojuszników RP}
3. września Anglia i Francja wypowiedziały wojnę Niemcom. 12 września sojusznicy postanawiają w Abbeville, iż nie zaatakują Niemiec, a jedynie ograniczą się do akcji propagandowej. Na Zachodzie rozpoczynają się Sitzkrieg i Drole de guerre.

\chapter{21 marca 2011 -- \textit{Na frontach II wojny światowej}}

\section{Atak Niemiec na kraje Skandynawskie}

9 kwietnia 1940 roku wojska niemieckie dokonały agresji na Danię i Norwegię, ponieważ oba państwa były ważne ze względów strategicznych. Zajęcie Danii umożliwiało kontrolowanie czegośtam, tak samo Norwegii.

\section{Atak Niemiec na Francję}

Eeeeeee Niemcy zrzucili żołnierzy eeeeeeee desantem w krajach eeeeeeeee Beneluksu. Eeeeeee dodatkowo przeszli przez eeeeeee góry, a potem eeeeee otoczyli alianckie oddziały. Eeeeee przez to oddziały były w eeeeee pułapce. Eeeee przez to alianci musieli uciec z Dunkierki eeeeee bardzo szybko eeeee zostawiając dużo eeeeeeeee sprzętu.

22 czerwca 1940 roku Philipp Petain podpisał zawieszenie broni. Utworzono w części nieokupowanej Państwo Francuskie z siedzibą rządu w Vichy (nowe władze kolaborowały z okupantem). Charles de Gaulle organizował armię francuską na terenach alianckich. W 1940 roku wydał dekret, odezwę, wzywającą do buntu przeciwko rządowi.

\section{Bitwa o Anglię}

Trwała od lipca do września 1940 roku. Kampania powietrzna głównie nad południową i centralną Anglią, toczona była pomiędzy niemieckim Luftwaffe, a brytyjskim RAF w okresie od 10 lipca do 31 października.

Brytyjskie myśliwce: Spitfire, Hurricane. Piloci eeeeeeeee brytyjscy mieli przewagę nad eeeeeeeeeeee pilotami eeeeeeeeeeeee niemieckimi.

Niemieckie myśliwce: Messerschmitt, bombowiec Ju-87 Stuka.

Pierwsze bombowce miały ,,eeee mały procent eeeeee trafności w eeeee cel'', bo nie nurkowały. Potem bombowce były eeeee nurkujące, eeee więc nurkowały eeeeeee, więc były niżej eeeeeee.

15 września był punktem krytycznym w bitwie o Anglię. Tego dnia nad Wielką Brytanią pojawiło się około 1000 niemieckich samolotów. Atak został odparty, duże straty Luftwaffe skłoniły Niemców do zmniejszenia częstości ataków na Anglię.

Problemem WB było mało wyszkolonych dobrych pilotów. Mimo to mieli dobre samoloty i sprzęt. Dlatego pozwolono obcokrajowcom latać na brytyjskich samolotach.

Eeeeeee najwięcej zeszczeleń miał Dywizjon 303.

Dzięki temu Churchill pewnego razu powiedział:

\begin{center}
\begin{quote}
\emph{Never have so many owed so much to so few.}

\emph{Jeszcze nigdy tak wielu nie zawdzięczało tak wiele, tak nielicznym.}
\end{quote}
\end{center}

\section{Wojna na Atlantyku}

Próbowano pokonać Wielką Brytanię poprzez odcięcie jej od dostaw zaopatrzenia, przewożonego drogą morską. Okręty podwodne wykorzystywały taktykę ,,wilczych stad''.

\end{document}
