\documentclass [a4paper, 10pt]{article}
\usepackage{fullpage}
\usepackage[utf8]{inputenc}
\usepackage{polski}
\usepackage{hyperref}
\usepackage[usenames,dvipsnames]{color}
\hypersetup{
    bookmarks=true,         % show bookmarks bar?
    unicode=false,          % non-Latin characters in Acrobat’s bookmarks
    pdftoolbar=true,        % show Acrobat’s toolbar?
    pdfmenubar=true,        % show Acrobat’s menu?
    pdffitwindow=false,     % window fit to page when opened
    pdfstartview={FitH},    % fits the width of the page to the window
    pdftitle={Zadanie domowe},    % title
    pdfauthor={Stanisław Chmiela},     % author
    pdfsubject={Zadanie domowe},   % subject of the document
    pdfcreator={Stanisław Chmiela},   % creator of the document
    pdfproducer={Stanisław Chmiela}, % producer of the document
    pdfkeywords={matematyka} {zadanie domowe}, % list of keywords
    pdfnewwindow=false,      % links in new window
    colorlinks=true,       % false: boxed links; true: colored links
    linkcolor=BrickRed,          % color of internal links
    citecolor=PineGreen,        % color of links to bibliography
    filecolor=RawSienna,      % color of file links
    urlcolor=MidnightBlue      % color of external links
}
\usepackage[pdftex]{graphicx}
\usepackage{wrapfig}
\usepackage{float}
\usepackage{tikz}
\usepackage{amsfonts}
\usepackage{multicol}
\usetikzlibrary{shapes,snakes,trees}
\usepackage{amsmath}
\usepackage{enumerate}
\linespread{1.3}
\author{Stanisław Chmiela}
\date{11 grudnia 2012}
\title{Zadanie domowe}
\begin{document}
\maketitle

\tikzstyle{mybox} = [draw=black, fill=white, thick,
    rectangle, rounded corners, inner sep=10pt, inner ysep=10pt]
\tikzstyle{fancytitle} =[fill=black, rounded corners, text=white, font=\bfseries, right=7pt]
\begin{multicols*}{2}
\begin{tikzpicture}
\node [mybox] (box){%
    \begin{minipage}{0.4\textwidth}
        Rozpatrzmy najpierw prawdopodobieństwa wylosowania białej kuli w odpowiednich urnach:
        \begin{enumerate}[I.]
            \item $\frac3{10}$
            \item $\frac5{10} = \frac12$
            \item $\frac7{10}$
            \item $0$
        \end{enumerate}
        Prawdopodobieństwo pójścia ponumerowanymi drogami jest równe. Zatem do odpowiednich urn trafimy z takimi prawdopodobieństwami:
        \begin{enumerate}[I.]
            \item $\frac14 + \frac14 = \frac12$
            \item $\frac13\cdot\frac12 = \frac16$
            \item $\frac13\cdot\frac12 = \frac16$
            \item $\frac14$
        \end{enumerate}
        Zsumujmy wyniki:
        \[
            \frac12\cdot\frac3{10} + \frac16\cdot\frac12 + \frac16\cdot\frac7{10} + \frac14\cdot0 = \mathbf{\frac7{20}}
        \]
    \end{minipage}
};
\node[fancytitle] at (box.north west) {Zadanie 8.19};
\end{tikzpicture}

\begin{tikzpicture}
\node [mybox] (box){%
    \begin{minipage}{0.4\textwidth}
        Rozwiążmy zadanie za pomocą drzewka stochastycznego. Ale najpierw policzmy pewne wartości:
        \begin{itemize}
            \item Prawdopodobieństwo wylosowania nieforemnego cukierka: $5\%\cdot40\%+4\%\cdot35\%+2\%\cdot25\% = \frac{271}{50000}$.
            \item Prawdopodobieństwo wylosowania foremnego cukierka: $1- \frac{271}{50000} = \frac{49729}{50000}$.
        \end{itemize}
        \tikzstyle{level 1}=[level distance=2.5cm, sibling distance=2cm]
        \tikzstyle{level 2}=[level distance=2.5cm, sibling distance=1cm]
        \tikzstyle{bag} = [text width=4em, text centered]
        \begin{tikzpicture}[grow=down, sloped]
        \node[bag] {Losujmy}
            child {
                node[bag] {$F_1$}
                child {
                        node[bag] {Dobry}
                        edge from parent
                        node[above] {98\%}
                }
                child {
                        node[bag] {Zły}
                        edge from parent
                        node[above]{2\%}
                }
                edge from parent
                node[above] {$\frac{25}{100}$}
            }
            child {
                node[bag] {$F_2$}
                child {
                        node[bag] {Dobry}
                        edge from parent
                        node[above] {96\%}
                }
                child {
                        node[bag] {Zły}
                        edge from parent
                        node[above]{4\%}
                }
                edge from parent
                node[above] {$\frac{35}{100}$}
            }
            child {
                node[bag] {$F_3$}
                child {
                        node[bag] {Dobry}
                        edge from parent
                        node[above] {95\%}
                }
                child {
                        node[bag] {Zły}
                        edge from parent
                        node[above]{5\%}
                }
                edge from parent
                node[above] {$\frac{40}{100}$}
            };
        \end{tikzpicture}
        \\
        Oznaczmy zdarzenia:
        \begin{itemize}
            \item Zdarzenie $A$ -- wylosowano dobrego cukierka,
            \item Zdarzenie $B$ -- wylosowano cukierka z firmy $F_3$.
        \end{itemize}
        Odpowiedzią będzie liczba:\\
        $P(B|A) = \frac{P(B\cap A)}{P(A)} = \frac{\frac{40}{100}\cdot95\%}{\frac{49729}{50000}} = \frac{19000}{49729} \approx \mathbf{38.21\%}$.
    \end{minipage}
};
\node[fancytitle] at (box.north west) {Zadanie 8.27};
\end{tikzpicture}


\end{multicols*}

\begin{tikzpicture}
\node [mybox] (box){%
    \begin{minipage}{0.9\textwidth}
        Prawdopodobieństwo wyrzucenia kostką do gry liczby podzielnej przez 3 (a co za tym idzie losowania z urny $U_1$): $\frac26 = \frac13$. Drzewkami stochastycznymi przeanalizujmy losowania w urnach $U_1$ i $U_2$.
        \tikzstyle{level 1}=[level distance=2.5cm, sibling distance=2.5cm]
        \tikzstyle{level 2}=[level distance=2.5cm, sibling distance=1cm]
        \tikzstyle{bag} = [text width=4em, text centered]
        \begin{center}
        \begin{multicols*}{2}
        \textbf{Urna }$\mathbf{U_1}$\\
        \begin{tikzpicture}[grow=down, sloped]
        \node[bag] {$6B, 4C$}
            child {
                node[bag] {{B}\\${5B, 4C}$}        
                    child {
                        node[bag] {{B}}
                        edge from parent
                        node[above] {${\frac{5}{9}}$}
                        node[below]  {}
                    }
                    child {
                        node[bag] {C}
                        edge from parent
                        node[above] {$\frac{4}{9}$}
                        node[below]  {}
                    }
                    edge from parent 
                    node[above] {${\frac6{10}}$}
                    node[below]  {}
            }
            child {
                node[bag] {C\\$6B, 3C$}        
                child {
                        node[bag] {{B}}
                        edge from parent
                        node[above] {${\frac{6}{9}}$}
                        node[below]  {}
                    }
                    child {
                        node[bag] {C}
                        edge from parent
                        node[above] {$\frac{3}{9}$}
                        node[below]  {}
                    }
                    edge from parent 
                    node[above] {${\frac4{10}}$}
                    node[below]  {}
            };
        \end{tikzpicture}
        \\\textbf{Urna }$\mathbf{U_2}$\\
        \begin{tikzpicture}[grow=down, sloped]
        \node[bag] {$4B, 8C$}
            child {
                node[bag] {{B}\\${3B, 8C}$}        
                    child {
                        node[bag] {{B}}
                        edge from parent
                        node[above] {${\frac{3}{11}}$}
                        node[below]  {}
                    }
                    child {
                        node[bag] {C}
                        edge from parent
                        node[above] {$\frac{8}{11}$}
                        node[below]  {}
                    }
                    edge from parent 
                    node[above] {${\frac4{12}}$}
                    node[below]  {}
            }
            child {
                node[bag] {C\\$4B, 7C$}        
                child {
                        node[bag] {{B}}
                        edge from parent
                        node[above] {${\frac{4}{11}}$}
                        node[below]  {}
                    }
                    child {
                        node[bag] {C}
                        edge from parent
                        node[above] {$\frac{7}{11}$}
                        node[below]  {}
                    }
                    edge from parent 
                    node[above] {${\frac7{12}}$}
                    node[below]  {}
            };
        \end{tikzpicture}
        \end{multicols*}
        \end{center}
        \paragraph{Podpunkt a)} Pomnóżmy prawdopodobieństwo losowania z urny $U_1$ przez prawdopodobieństwo wylosowania dwóch kul białych z tej urny, dodajmy następnie iloczyn prawdopodobieństwa losowania z urny $U_2$ oraz prawdopodobieństwa wylosowania dwóch białych kul.\\
        $$\frac13\cdot\frac6{10}\cdot\frac59+\frac23\cdot\frac4{12}\cdot\frac3{11} = \mathbf{\frac{17}{99}}$$
        \paragraph{Podpunkt b)} Pomnóżmy prawdopodobieństwo losowania z urny $U_1$ przez prawdopodobieństwo wylosowania dwóch kul o różnych kolorach z urny, dodajmy następnie iloczyn prawdopodobieństwa losowania z urny $U_2$ oraz prawdopodbieństwa wylosowania dwóch kul o różnych kolorach.\\
        $$\frac13\cdot\left(\frac6{10}\cdot\frac49+\frac4{10}\cdot\frac69\right) + \frac23\cdot\left(\frac4{12}\cdot\frac8{11}+\frac7{12}\cdot\frac4{11}\right) = \mathbf{\frac{238}{495}}$$
    \end{minipage}
};
\node[fancytitle] at (box.north west) {Zadanie 8.13};
\end{tikzpicture}
\end{document}