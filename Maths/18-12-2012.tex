\documentclass [a4paper, 10pt]{article}
\usepackage{fullpage}
\usepackage[utf8]{inputenc}
\usepackage{polski}
\usepackage{hyperref}
\usepackage[usenames,dvipsnames]{color}
\hypersetup{
    bookmarks=true,         % show bookmarks bar?
    unicode=false,          % non-Latin characters in Acrobat’s bookmarks
    pdftoolbar=true,        % show Acrobat’s toolbar?
    pdfmenubar=true,        % show Acrobat’s menu?
    pdffitwindow=false,     % window fit to page when opened
    pdfstartview={FitH},    % fits the width of the page to the window
    pdftitle={Zadanie domowe},    % title
    pdfauthor={Stanisław Chmiela},     % author
    pdfsubject={Zadanie domowe},   % subject of the document
    pdfcreator={Stanisław Chmiela},   % creator of the document
    pdfproducer={Stanisław Chmiela}, % producer of the document
    pdfkeywords={matematyka} {zadanie domowe}, % list of keywords
    pdfnewwindow=false,      % links in new window
    colorlinks=true,       % false: boxed links; true: colored links
    linkcolor=BrickRed,          % color of internal links
    citecolor=PineGreen,        % color of links to bibliography
    filecolor=RawSienna,      % color of file links
    urlcolor=MidnightBlue      % color of external links
}
\usepackage[pdftex]{graphicx}
\newcommand{\mbf}[1]{\boldsymbol{\mathbf{#1}}}
\usepackage{wrapfig}
\usepackage{float}
\usepackage{tikz}
\usepackage{amsfonts}
\usepackage{multicol}
\usetikzlibrary{shapes,snakes,trees}
\usepackage{amsmath}
\usepackage{enumerate}
\linespread{1.3}
\author{Stanisław Chmiela}
\date{18 grudnia 2012}
\title{Zadanie domowe}
\begin{document}
\maketitle

\tikzstyle{mybox} = [draw=black, fill=white, thick,
    rectangle, rounded corners, inner sep=10pt, inner ysep=10pt]
\tikzstyle{fancytitle} =[fill=black, rounded corners, text=white, font=\bfseries, right=7pt]
\begin{multicols*}{2}

\begin{tikzpicture}
\node [mybox] (box){%
    \begin{minipage}{0.4\textwidth}
        Z treści zadania wynika, że prawdopodobieństwo, że Maciek trafi do tarczy wynosi $\frac9{10}$, a że Tomek trafi w tarczę $\frac{85}{100}$. Zatem prawdopodobieństwo, że nie trafi Maciek wynosi $\frac1{10}$, że nie trafi Tomek $\frac{15}{100}$. Zatem prawdopodobieństwo, że każdy z nich nie trafi wynosi $\frac1{10} \cdot \frac{15}{100} = \frac{15}{1000}$. Prawdopodobieństwo, że tarcza zostanie trafiona przynajmniej raz wynosi $1-\frac{15}{1000} = \mbf{\frac{985}{1000}}$.
    \end{minipage}
};
\node[fancytitle] at (box.north west) {Zadanie 9.24};
\end{tikzpicture}

\begin{tikzpicture}
\node [mybox] (box){%
    \begin{minipage}{0.4\textwidth}
        Oznaczmy sobie zbiory:
        \begin{itemize}
            \item $A$ -- zbiór liczb podzielnych przez 2, $\#A = \frac{6n}2 = 3n$
            \item $B$ -- zbiór liczb podzielnych przez 3, $\#B = \frac{6n}3 = 2n$
            \item $A\cap B$ -- zbiór liczb podzielnych przez 2 i przez 3, $\#(A\cap B) = \frac{6n}6 = n$.
        \end{itemize}
        Wtedy z zasady włączeń i wyłączeń $\#{A\cup B} = \#A + \#B - \#({A\cap B}) = 3n + 2n - n = \mbf{4n}$.
    \end{minipage}
};
\node[fancytitle] at (box.north west) {Zadanie 9.27};
\end{tikzpicture}

\begin{tikzpicture}
\node [mybox] (box){%
    \begin{minipage}{0.4\textwidth}
        Schemat Bernoulliego\\
        Prawdopodobieństwo sukcesu jest równe prawdopodobieństwu przegranej, czyli $\frac12$.
        \paragraph{Podpunkt a)} $P(S_8=7) + P(S_8=8) = {8\choose7}\cdot\left(\frac12\right)^8 + {8\choose8}\cdot\left(\frac12\right)^8 = \mbf{\frac{9}{256}}$
        \paragraph{Podpunkt b)} $P(S_8=0) + P(S_8=1) + P(S_8=2) + P(S_8=3) = {8\choose0}\frac12^8 + {8\choose1}\frac12^8 + {8\choose2}\frac12^8 + {8\choose3}\frac12^8 = \mbf{\frac{93}{256}}$
        \paragraph{Podpunkt c)} $P(S_8=4) + P(S_8=5) + P(S_8=6) = {8\choose4}\frac12^8 + {8\choose5}\frac12^8 + {8\choose6}\frac12^8 = \mbf{\frac{77}{128}}$.
    \end{minipage}
};
\node[fancytitle] at (box.north west) {Zadanie 10.11};
\end{tikzpicture}

\begin{tikzpicture}
\node [mybox] (box){%
    \begin{minipage}{0.4\textwidth}
        Schemat Bernoulliego\\
        Przez sukces oznaczmy sobie sumę oczek większą lub równą 9. Wtedy prawdopodobieństwo, że wygramy wynosi $\frac1{36}(4+3+2+1) = \frac{10}{36}$.\\
        $P(S_3=2) + P(S_3=3) = {3\choose2}\left(\frac{10}{36}\right)^{2}\left(\frac{26}{36}\right) + {3\choose3}\left(\frac{10}{36}\right)^{3} = \mbf{\frac{275}{1458}}$.
    \end{minipage}
};
\node[fancytitle] at (box.north west) {Zadanie 10.12};
\end{tikzpicture}

\begin{tikzpicture}
\node [mybox] (box){%
    \begin{minipage}{0.4\textwidth}
        Schemat Bernoulliego\\
        Przez sukces oznaczmy sobie sumę oczek większą lub równą 4. Wtedy prawdopodobieństwo, że wygramy wynosi $\frac1{36}(4+5+6+6+6+6) = \frac{33}{36}$.\\
        $P(S_7=3) + P(S_7=4) + P(S_7=5) + P(S_7=6) + P(S_7=7) = {7\choose3}\left(\frac{33}{36}\right)^3\left(\frac{3}{36}\right)^4 + {7\choose4}\left(\frac{33}{36}\right)^4\left(\frac{3}{36}\right)^3 + {7\choose5}\left(\frac{33}{36}\right)^5\left(\frac{3}{36}\right)^2 + {7\choose6}\left(\frac{33}{36}\right)^6\left(\frac{3}{36}\right) + {7\choose7}\left(\frac{33}{36}\right)^7 = \mbf{\frac{1327007}{1327104}}$.
    \end{minipage}
};
\node[fancytitle] at (box.north west) {Zadanie 10.14};
\end{tikzpicture}

\end{multicols*}

\end{document}

