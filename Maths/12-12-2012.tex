\documentclass [a4paper, 10pt]{article}
\usepackage{fullpage}
\usepackage[utf8]{inputenc}
\usepackage{polski}
\usepackage{hyperref}
\usepackage[usenames,dvipsnames]{color}
\hypersetup{
    bookmarks=true,         % show bookmarks bar?
    unicode=false,          % non-Latin characters in Acrobat’s bookmarks
    pdftoolbar=true,        % show Acrobat’s toolbar?
    pdfmenubar=true,        % show Acrobat’s menu?
    pdffitwindow=false,     % window fit to page when opened
    pdfstartview={FitH},    % fits the width of the page to the window
    pdftitle={Zadanie domowe},    % title
    pdfauthor={Stanisław Chmiela},     % author
    pdfsubject={Zadanie domowe},   % subject of the document
    pdfcreator={Stanisław Chmiela},   % creator of the document
    pdfproducer={Stanisław Chmiela}, % producer of the document
    pdfkeywords={matematyka} {zadanie domowe}, % list of keywords
    pdfnewwindow=false,      % links in new window
    colorlinks=true,       % false: boxed links; true: colored links
    linkcolor=BrickRed,          % color of internal links
    citecolor=PineGreen,        % color of links to bibliography
    filecolor=RawSienna,      % color of file links
    urlcolor=MidnightBlue      % color of external links
}
\usepackage[pdftex]{graphicx}
\usepackage{wrapfig}
\usepackage{float}
\usepackage{tikz}
\usepackage{amsfonts}
\usepackage{multicol}
\usetikzlibrary{shapes,snakes,trees}
\usepackage{amsmath}
\usepackage{enumerate}
\linespread{1.3}
\author{Stanisław Chmiela}
\date{12 grudnia 2012}
\title{Zadanie domowe}
\begin{document}
\maketitle

\tikzstyle{mybox} = [draw=black, fill=white, thick,
    rectangle, rounded corners, inner sep=10pt, inner ysep=10pt]
\tikzstyle{fancytitle} =[fill=black, rounded corners, text=white, font=\bfseries, right=7pt]
\begin{multicols*}{2}
\begin{tikzpicture}
\node [mybox] (box){%
    \begin{minipage}{0.4\textwidth}
        Rozwiążmy zadanie za pomocą drzewka stochastycznego.\\
        \tikzstyle{level 1}=[level distance=2.5cm, sibling distance=2.5cm]
        \tikzstyle{level 2}=[level distance=2.5cm, sibling distance=1.5cm]
        \tikzstyle{bag} = [text width=4em, text centered]
        \begin{tikzpicture}[grow=down, sloped]
        \node[bag] {Losujemy kostką}
            child {
                node[bag] {Urna $U_1$}
                child {
                        node[bag] {W}
                        edge from parent
                        node[above] {$\frac3{10}$}
                }
                child {
                        node[bag] {P}
                        edge from parent
                        node[above] {$\frac7{10}$}
                }
                edge from parent
                node[above] {$\frac26$}
            }
            child {
                node[bag] {Urna $U_2$}
                child {
                        node[bag] {W}
                        edge from parent
                        node[above] {$\frac59$}
                }
                child {
                        node[bag] {P}
                        edge from parent
                        node[above] {$\frac49$}
                }
                edge from parent
                node[above] {$\frac46$}
            };
        \end{tikzpicture}\\
        Prawdopodobieństwo wylosowania losu wygrywającego wynosi: $\frac26\cdot\frac3{10}+\frac46\cdot\frac59 = \mathbf{\frac{127}{270}}$.
    \end{minipage}
};
\node[fancytitle] at (box.north west) {Zadanie 8.21};
\end{tikzpicture}

\begin{tikzpicture}
\node [mybox] (box){%
    \begin{minipage}{0.4\textwidth}
        Rozwiążmy zadanie za pomocą drzewka stochastycznego.\\
        \tikzstyle{level 1}=[level distance=2.5cm, sibling distance=2.5cm]
        \tikzstyle{level 2}=[level distance=2.5cm, sibling distance=1.5cm]
        \tikzstyle{bag} = [text width=4em, text centered]
        \begin{tikzpicture}[grow=down, sloped]
        \node[bag] {Losujemy kostką}
            child {
                node[bag] {Pudełko $P_1$}
                child {
                        node[bag] {W}
                        edge from parent
                        node[above] {$\frac4{10}$}
                }
                child {
                        node[bag] {P}
                        edge from parent
                        node[above] {$\frac6{10}$}
                }
                edge from parent
                node[above] {$\frac26$}
            }
            child {
                node[bag] {Pudełko $P_2$}
                child {
                        node[bag] {W}
                        edge from parent
                        node[above] {$\frac49$}
                }
                child {
                        node[bag] {P}
                        edge from parent
                        node[above] {$\frac59$}
                }
                edge from parent
                node[above] {$\frac46$}
            };
        \end{tikzpicture}
        \paragraph{Podpunkt a)} Prawdopodobieństwo wylosowania losu wygrywającego wynosi:
        \[
            \frac26\cdot\frac4{10}+\frac46\cdot\frac49 = \mathbf{\frac{58}{135}}
        \]
        \paragraph{Podpunkt b)} Prawdopodobieństwo wylosowania losu przegrywającego wynosi:
        \[
            1- \frac{58}{135} = \mathbf{\frac{77}{135}}
        \]
    \end{minipage}
};
\node[fancytitle] at (box.north west) {Zadanie 8.22};
\end{tikzpicture}

\begin{tikzpicture}
\node [mybox] (box){%
    \begin{minipage}{0.4\textwidth}
        Rozpatrzmy przypadki drzewkiem stochastycznym.\\
        \tikzstyle{level 1}=[level distance=2.5cm, sibling distance=2.5cm]
        \tikzstyle{level 2}=[level distance=2.5cm, sibling distance=1.5cm]
        \tikzstyle{bag} = [text width=5em, text centered]
        \begin{tikzpicture}[grow=down, sloped]
        \node[bag] {Morela}
            child {
                node[bag] {Z Hiszpanii}
                child {
                        node[bag] {D}
                        edge from parent
                        node[above] {$99.2\%$}
                }
                child {
                        node[bag] {N}
                        edge from parent
                        node[above] {$0.8\%$}
                }
                edge from parent
                node[above] {$\frac9{20}$}
            }
            child {
                node[bag] {Z Włoch}
                child {
                        node[bag] {D}
                        edge from parent
                        node[above] {$98.2\%$}
                }
                child {
                        node[bag] {N}
                        edge from parent
                        node[above] {$1.2\%$}
                }
                edge from parent
                node[above] {$\frac{11}{20}$}
            };
        \end{tikzpicture}
        \\
        Prawdopodobieństwo kupienia niedojrzałej moreli wynosi:
        \[
            \frac9{20}\cdot0.8\% + \frac{11}{20}\cdot1.2\% = \mathbf{\frac{51}{5000}}
        \]
    \end{minipage}
};
\node[fancytitle] at (box.north west) {Zadanie 8.28};
\end{tikzpicture}\\
\phantom{Ble}\\
\phantom{cps}
\begin{tikzpicture}
\node [mybox] (box){%
    \begin{minipage}{0.4\textwidth}
        Oznaczmy sobie zdarzenia:
        \begin{itemize}
            \item Zdarzenie $F_i$ -- cukierek pochodzi z firmy $F_i$,
            \item Zdarzenie $D$ -- cukierek jest foremny,
            \item Zdarzenie $N$ -- cukierek jest zdeformowany.
        \end{itemize}
        Na podstawie treści zadania prawdopodobieństwa wynoszą:
        \begin{itemize}
            \item $P(F_3) = \frac25$
            \item $P(N|F_3) = \frac1{20}$
            \item $P(D|F_3) = \frac{19}{20}$
            \item $P(D) = 1-P(N) = 1-\left(\frac14\cdot\frac1{50}+\frac7{20}\cdot\frac1{25}+\frac25\cdot\frac1{20}\right) = \frac{961}{1000}$
        \end{itemize}

        Na podstawie twierdzenia Bayesa prawdopodobieństwo, że wylosowany cukierek, który jest dobry, pochodzi od firmy $F_3$ wynosi:\\
        $
            P(F_3|D) = \frac{P(D|F_3)\cdot P(F_3)}{P(D)} = \frac{\frac{19}{20}\cdot\frac25}{\frac{961}{1000}} = \mathbf{\frac{380}{961}}
        $
    \end{minipage}
};
\node[fancytitle] at (box.north west) {Zadanie 8.27};
\end{tikzpicture}

\end{multicols*}

\end{document}