\documentclass [a4paper, 10pt]{article}
\usepackage{fullpage}
\usepackage[utf8]{inputenc}
\usepackage{polski}
\usepackage{hyperref}
\usepackage[usenames,dvipsnames]{color}
\hypersetup{
    bookmarks=true,         % show bookmarks bar?
    unicode=false,          % non-Latin characters in Acrobat’s bookmarks
    pdftoolbar=true,        % show Acrobat’s toolbar?
    pdfmenubar=true,        % show Acrobat’s menu?
    pdffitwindow=false,     % window fit to page when opened
    pdfstartview={FitH},    % fits the width of the page to the window
    pdftitle={Zadanie domowe},    % title
    pdfauthor={Stanisław Chmiela},     % author
    pdfsubject={Zadanie domowe},   % subject of the document
    pdfcreator={Stanisław Chmiela},   % creator of the document
    pdfproducer={Stanisław Chmiela}, % producer of the document
    pdfkeywords={matematyka} {zadanie domowe}, % list of keywords
    pdfnewwindow=false,      % links in new window
    colorlinks=true,       % false: boxed links; true: colored links
    linkcolor=BrickRed,          % color of internal links
    citecolor=PineGreen,        % color of links to bibliography
    filecolor=RawSienna,      % color of file links
    urlcolor=MidnightBlue      % color of external links
}
\usepackage[pdftex]{graphicx}
\usepackage{wrapfig}
\usepackage{float}
\usepackage{tikz}
\usepackage{amsfonts}
\usepackage{multicol}
\usetikzlibrary{shapes,snakes,trees}
\usepackage{amsmath}
\usepackage{enumerate}
\linespread{1.3}
\author{Stanisław Chmiela}
\date{13 grudnia 2012}
\title{Zadanie domowe}
\begin{document}
\maketitle

\tikzstyle{mybox} = [draw=black, fill=white, thick,
    rectangle, rounded corners, inner sep=10pt, inner ysep=10pt]
\tikzstyle{fancytitle} =[fill=black, rounded corners, text=white, font=\bfseries, right=7pt]
\begin{multicols*}{2}
\begin{tikzpicture}
\node [mybox] (box){%
    \begin{minipage}{0.4\textwidth}
        Oznaczmy prawdopodbieństwa:
        \begin{itemize}
            \item Prawdopodobieństwo wystąpienia awarii monitora: $P(A_e) = \frac3{10}$
            \item Prawdopodobieństwo wystąpienia awarii klawiatury: $P(A_k) = \frac2{10}$
            \item Prawdopodobieństwo wystąpienia awarii myszy: $P(A_m) = \frac5{10}$
            \item Prawdopodobieństwo znalezienia awarii monitora: $P(Z_e) = \frac8{10}$
            \item Prawdopodobieństwo znalezienia awarii klawiatury: $P(Z_k) = \frac9{10}$
            \item Prawdopodobieństwo znalezienia awarii myszy: $P(Z_m) = \frac9{10}$
        \end{itemize}
        Prawdopodobieństwo znalezienia awarii w komputerze równe jest sumie wykrycia awarii gdy awaria wystąpiła w poszczególnych komponentach:\\
        $P(A) = P(Z_e\cap A_e) + P(Z_k\cap A_k) + P(Z_m\cap A_m) = \frac3{10}\cdot\frac8{10} + \frac2{10}\cdot\frac9{10} + \frac5{10}\cdot\frac9{10} = \mathbf{\frac{87}{100}}$.
    \end{minipage}
};
\node[fancytitle] at (box.north west) {Zadanie 8.29};
\end{tikzpicture}

\begin{tikzpicture}
\node [mybox] (box){%
    \begin{minipage}{0.4\textwidth}
        \begin{itemize}
            \item Prawdopodobieństwo, że wylosujemy mężczyznę $P(M) = \frac12$.
            \item Prawdopodobieństwo, że wylosujemy kobietę $P(K) = \frac12$.
            \item Prawdopodobieństwo, że wylosowany mężczyzna nie rozróżnia kolorów $P(D_m) = \frac5{100}$.
            \item Prawdopodobieństwo, że wylosowana kobieta nie rozróżnia kolorów $P(D_k) = \frac2{1000}$.
        \end{itemize}
        Prawdopodobieństwo, że wylosujemy osobę, która nie rozróżnia kolorów:\\
        $P(D) = P(M)\cdot P(D_m) + P(K)\cdot P(D_k) = \frac12\cdot\frac5{100} + \frac12\cdot\frac2{1000} = \mathbf{\frac{13}{500}}$.
    \end{minipage}
};
\node[fancytitle] at (box.north west) {Zadanie 8.31};
\end{tikzpicture}

\begin{tikzpicture}
\node [mybox] (box){%
    \begin{minipage}{0.4\textwidth}
        \tikzstyle{level 1}=[level distance=2cm, sibling distance=2.5cm]
        \tikzstyle{bag} = [text width=4em, text centered]
        \begin{tikzpicture}[grow=down, sloped]
        \node[bag] {Rzucamy}
            child {
                node[bag] {R}
                child {
                        node[bag] {R}
                        child {
                        node[bag] {R}
                        edge from parent
                       node[above] {$\frac12$}
                }
                        edge from parent
                       node[above] {$\frac12$}
                }
                edge from parent
                node[above] {$\frac12$}
            }
            child {
                node[bag] {O}
                child {
                        node[bag] {R}
                        edge from parent
                       node[above] {$\frac12$}
                }
                child {
                        node[bag] {O}
                                child {
                                node[bag] {R}
                                edge from parent
                                node[above] {$\frac12$}
                        }
                        child {
                                node[bag] {O}
                                edge from parent
                                node[above] {$\frac12$}
                        }
                        edge from parent
                        node[above] {$\frac12$}
                }
                edge from parent
                node[above] {$\frac12$}
            };
        \end{tikzpicture}\\
        $P(A) = \frac12 + \frac12^2 + \frac12^3 = \frac12+\frac14+\frac18 = \frac78$\\
        $P(B) = \frac12^3 + \frac12^3 = \frac18 + \frac18 = \frac28$\\
        Prawdopobieństwo, że otrzymaliśmy minimum jedną reszkę, a zarazem 3 orły lub 3 reszki (czyli że otrzymaliśmy 3 reszki):\\
        $P(A\cap B) = \frac18$.\\
        Jeśli zdarzenia są niezależne, $P(A)\cdot P(B) = P(A\cap B)$:\\
        $P(A)\cdot P(B) = \frac{7}{32} \neq \frac28 = P(A\cap B)$\\
        \textbf{Zdarzenia nie są niezależne.}
    \end{minipage}
};
\node[fancytitle] at (box.north west) {Zadanie 9.5};
\end{tikzpicture}

\begin{tikzpicture}
\node [mybox] (box){%
    \begin{minipage}{0.4\textwidth}
        \begin{itemize}
            \item $P(A) = \frac12$
            \item $P(B) = \frac{0+1+2+3+4+5}{36} = \frac{15}{36}$
        \end{itemize}
        Prawdopodobieństwo, że na pierwszej kostce wypadły co najmniej cztery oczka, a suma jest większa niż siedem sprowadza się do takich par: $(4,4), (4,5), (4,6), (5,3), (5,4), (5,5),\\ (5,6), (6,2), (6,3), (6,4), (6,5), (6,6)$. Jest ich 12. Zatem $P(A\cap B) = \frac{12}{36}$.\\
        Jeśli zdarzenia są niezależne od siebie, $P(A)\cdot P(B) = P(A\cap B)$.\\
        $P(A)\cdot P(B) = \frac12\cdot\frac{15}{36} = \frac{15}{72} \neq \frac{12}{36} = P(A\cap B)$.\\
        \textbf{Zdarzenia nie są niezależne.}
    \end{minipage}
};
\node[fancytitle] at (box.north west) {Zadanie 9.8};
\end{tikzpicture}

\end{multicols*}

\begin{tikzpicture}
\node [mybox] (box){%
    \begin{minipage}{0.9\textwidth}
        \tikzstyle{level 1}=[level distance=2cm, sibling distance=7cm]
        \tikzstyle{level 2}=[level distance=2cm, sibling distance=3cm]
        \tikzstyle{level 3}=[level distance=2cm, sibling distance=1.5cm]
        \tikzstyle{level 4}=[level distance=2cm, sibling distance=0.5cm]
        \tikzstyle{bag} = [text width=4em, text centered]
        \begin{tikzpicture}[grow=down, sloped]
        \node[bag] {Rzucamy}
            child {
                node[bag] {R}
                child {
                    node[bag] {R}
                    child {
                        node[bag] {R}
                        child {
                node[bag] {R}
                edge from parent
                node[above] {$\frac12$}
            }
            child {
                node[bag] {O}
                edge from parent
                node[above] {$\frac12$}
            }
                        edge from parent
                        node[above] {$\frac12$}
                    }
                    child {
                        node[bag] {O}
                        child {
                node[bag] {R}
                edge from parent
                node[above] {$\frac12$}
            }
            child {
                node[bag] {O}
                edge from parent
                node[above] {$\frac12$}
            }
                        edge from parent
                        node[above] {$\frac12$}
                    }
                    edge from parent
                    node[above] {$\frac12$}
                }
            child {
                node[bag] {O} %% TUTAJ
                child{
                node[bag]{R}
                child{
                node[bag]{R}
                edge from parent
                node[above]{$\frac12$}
                }
                child{
                node[bag]{O}
                edge from parent
                node[above]{$\frac12$}
                }
                edge from parent
                node[above]{$\frac12$}
                }
                child{
                node[bag]{O}
                child{
                node[bag]{R}
                edge from parent
                node[above]{$\frac12$}
                }
                child{
                node[bag]{O}
                edge from parent
                node[above]{$\frac12$}
                }
                edge from parent
                node[above]{$\frac12$}
                }
                edge from parent
                node[above] {$\frac12$}
            }
                edge from parent
                node[above] {$\frac12$}
            }
            child {
                node[bag] {O}
                    child {
                    node[bag] {R}
                        child {
                        node[bag] {R}
                        child {
                node[bag] {R}
                edge from parent
                node[above] {$\frac12$}
            }
            child {
                node[bag] {O}
                edge from parent
                node[above] {$\frac12$}
            }
                        edge from parent
                        node[above] {$\frac12$}
                    }
                    child {
                        node[bag] {O}
                        child {
                node[bag] {R}
                edge from parent
                node[above] {$\frac12$}
            }
            child {
                node[bag] {O}
                edge from parent
                node[above] {$\frac12$}
            }
                        edge from parent
                        node[above] {$\frac12$}
                    }
                    edge from parent
                    node[above] {$\frac12$}
                }
                child {
                    node[bag] {O}
                        child {
                        node[bag] {R}
                        child {
                node[bag] {R}
                edge from parent
                node[above] {$\frac12$}
            }
            child {
                node[bag] {O}
                edge from parent
                node[above] {$\frac12$}
            }
                        edge from parent
                        node[above] {$\frac12$}
                    }
                    child {
                        node[bag] {O}
                        child {
                node[bag] {R}
                edge from parent
                node[above] {$\frac12$}
            }
            child {
                node[bag] {O}
                edge from parent
                node[above] {$\frac12$}
            }
                        edge from parent
                        node[above] {$\frac12$}
                }
                    edge from parent
                    node[above] {$\frac12$}
                }
                edge from parent
                node[above] {$\frac12$}
            };
        \end{tikzpicture}\\
        $P(A) = 5\cdot\frac12^4 = 5\cdot\frac1{16} = \frac5{16}$\\
        $P(B) = 1-2\cdot\frac12^4 = 1-2\cdot\frac1{16} = 1-\frac18 = \frac78$\\
        Prawdopodbieństwo wylosowania nie więcej niż jednej reszki i jednocześnie otrzymania orła i reszki w czterech losowaniach sprowadza się do prawdopodobieństwa wylosowania dokładnie jednej reszki w czterech rzutach, a to się równa (4 różne ścieżki z jednym R):\\
        $P(A\cap B) = 4\cdot\frac12^4 = \frac14$.\\
        Jeśli zdarzenia są niezależne: $P(A)\cdot P(B) = P(A\cap B)$.\\
        $P(A)\cdot P(B) = \frac5{16}\cdot \frac78 = \frac{35}{128} \neq \frac14 = P(A\cap B)$.\\
        \textbf{Zdarzenia nie są niezależne od siebie.}
    \end{minipage}
};
\node[fancytitle] at (box.north west) {Zadanie 9.7};
\end{tikzpicture}

\end{document}

