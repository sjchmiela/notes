\documentclass [a4paper, 10pt]{article}
\usepackage{fullpage}
\usepackage[utf8]{inputenc}
\usepackage{polski}
\usepackage{hyperref}
\usepackage[usenames,dvipsnames]{color}
\hypersetup{
    bookmarks=true,         % show bookmarks bar?
    unicode=false,          % non-Latin characters in Acrobat’s bookmarks
    pdftoolbar=true,        % show Acrobat’s toolbar?
    pdfmenubar=true,        % show Acrobat’s menu?
    pdffitwindow=false,     % window fit to page when opened
    pdfstartview={FitH},    % fits the width of the page to the window
    pdftitle={Zadanie domowe},    % title
    pdfauthor={Stanisław Chmiela},     % author
    pdfsubject={Zadanie domowe},   % subject of the document
    pdfcreator={Stanisław Chmiela},   % creator of the document
    pdfproducer={Stanisław Chmiela}, % producer of the document
    pdfkeywords={matematyka} {zadanie domowe}, % list of keywords
    pdfnewwindow=false,      % links in new window
    colorlinks=true,       % false: boxed links; true: colored links
    linkcolor=BrickRed,          % color of internal links
    citecolor=PineGreen,        % color of links to bibliography
    filecolor=RawSienna,      % color of file links
    urlcolor=MidnightBlue      % color of external links
}
\usepackage[pdftex]{graphicx}
\usepackage{wrapfig}
\usepackage{float}
\usepackage{tikz}
\usepackage{amsfonts}
\usepackage{multicol}
\usetikzlibrary{shapes,snakes,trees}
\usepackage{amsmath}
\usepackage{enumerate}
\newcommand{\mathb}[1]{\boldsymbol{\mathbf{#1}}}
\linespread{1.3}
\author{Stanisław Chmiela}
\title{Matura: zestaw otwarty XIV}
\begin{document}
\maketitle

\tikzstyle{mybox} = [draw=black, fill=white, thick,
    rectangle, rounded corners, inner sep=10pt, inner ysep=10pt]
\tikzstyle{fancytitle} =[fill=black, rounded corners, text=white, font=\bfseries, right=7pt]
\begin{multicols*}{2}
\begin{tikzpicture}
\node [mybox] (box){%
    \begin{minipage}{0.4\textwidth}
       Rozpiszmy sobie wyrażenie, a następnie skorzystajmy ze wzorów redukcyjnych.\\
       $(\sin\alpha-\cos\alpha)(\sin\beta-\cos\beta) = \sin\alpha\sin\beta+\cos\alpha\cos\beta-(\sin\alpha\cos\beta+\cos\alpha\sin\beta) = \cos(\alpha-\beta) - (\sin(\alpha+\beta)) = 0.3-0.8 = \mathb{-\frac12}$.
    \end{minipage}
};
\node[fancytitle] at (box.north west) {Zadanie 1};
\end{tikzpicture}

\begin{tikzpicture}
\node [mybox] (box){%
    \begin{minipage}{0.4\textwidth}
       \paragraph{Podpunkt a)} $\sin 2x = \sin x \Leftrightarrow 2\sin x\cos x = \sin x$. Mamy dwa przypadki:
       \begin{enumerate}
           \item $\sin x = 0 \Leftrightarrow \mathb{x = 0 \lor x = \pi \lor x = 2\pi}$
           \item $\cos x = \frac12 \Leftrightarrow \mathb{x = \frac\pi3 \lor x = 2\pi-\frac\pi3 = \frac{5\pi}{3}}$
       \end{enumerate}
       \paragraph{Podpunkt b)} $\sin 2x > \sin x \Leftrightarrow 2\sin x\cos x > \sin x$. Jeśli $\sin x = 0$, nierówność nie jest spełniona. W przeciwnym wypadku $\cos x > \frac12 \Leftrightarrow \mathb{x \in \langle 0, \frac\pi3) \cup (\frac{5\pi}{3},2\pi\rangle}$.
    \end{minipage}
};
\node[fancytitle] at (box.north west) {Zadanie 2};
\end{tikzpicture}

\begin{tikzpicture}
\node [mybox] (box){%
    \begin{minipage}{0.4\textwidth}
       Na przedziale $\left(-\frac\pi2, \frac\pi2\right)$ tangens jest różnowartościowy, zatem równanie sprowadza się do $x+\frac\pi3 = \frac\pi2-x \Leftrightarrow 2x = \frac\pi6 \Leftrightarrow x = \frac\pi{12}$.

       Skorzystajmy jednak jeszcze ze wzoru redukcyjnego $\tg(\pi+x) = \tg x$. Wtedy $\tg(x+\frac\pi3) = \tg(x+\frac{4\pi}3)$. Sprawdźmy zatem kiedy $x+\frac{4\pi}3 = \frac\pi2-x \Leftrightarrow 2x = \frac\pi{2}-\frac{4\pi}3 = -\frac{5\pi}6$. W tym przypadku $x = -\frac{5\pi}{12}$.

       Zatem $\mathb{x = \left\{-\frac{5\pi}6, \frac\pi{12}\right\}}$.
    \end{minipage}
};
\node[fancytitle] at (box.north west) {Zadanie 3};
\end{tikzpicture}

\begin{tikzpicture}
\node [mybox] (box){%
    \begin{minipage}{0.4\textwidth}
       $f(x) = \cos^2 x - \sin x = 1-\sin^2 x - \sin x = -\sin^2 x - \sin x + 1$. Funkcja jest funkcją kwadratową.\\
       Podstawmy $t := \sin x$.\\
       $f(t) = -t^2 - t + 1$. Wierzchołek funkcji kwadratowej leży w punkcie $(\frac{-b}{2a}, \frac{\Delta}{4a}) = (-\frac12, \frac54)$. Zatem największą wartość funkcja przyjmuje dla $t = 0, \mathb{m = f(0) = 1}$, a najmniejszą dla $t = -\frac12, \mathb{M = f(-\frac12) = \frac54}$.
    \end{minipage}
};
\node[fancytitle] at (box.north west) {Zadanie 4};
\end{tikzpicture}

\begin{tikzpicture}
\node [mybox] (box){%
    \begin{minipage}{0.4\textwidth}
      Aby nierówność była spełniona:
      \begin{enumerate}
        \item $\cos x - \cos\frac\pi4 \ge 0 \Leftrightarrow \cos x \ge \cos\frac\pi4$ lub
        \item $\cos x = 0 \Leftrightarrow x = -\frac\pi2 \lor x = \frac\pi2$. 
      \end{enumerate}
      Rozwiązaniem pierwszej nierówności jest przedział $x \in \langle-\frac\pi4,\frac\pi4\rangle$. Łącząc te przedziały, rozwiązaniami nierówności jest zbiór liczb $\mathb{x: x\in\left\{-\frac\pi2,\frac\pi2\right\}\cup\langle-\frac\pi4,\frac\pi4\rangle}$.
    \end{minipage}
};
\node[fancytitle] at (box.north west) {Zadanie 5};
\end{tikzpicture}

\begin{tikzpicture}
\node [mybox] (box){%
    \begin{minipage}{0.4\textwidth}
      $\tg x = 4 \Rightarrow \frac{\sin x}{\cos x} = 4 \Rightarrow \sin x = 4\cos x \Rightarrow \sin x = 4\sqrt{1-\sin^2 x} \Rightarrow \sin^2 x = 16-16\sin^2 x \Rightarrow \sin^2 x = \frac{16}{17}$.\\
      Ze wzorów trygonometrycznych wiemy, że $\cos2x = 1-2\sin^2x = 1-\frac{2\cdot16}{17} = \frac{17-2\cdot16}{17} = \mathb{-\frac{15}{17}}$.
    \end{minipage}
};
\node[fancytitle] at (box.north west) {Zadanie 6};
\end{tikzpicture}

\begin{tikzpicture}
\node [mybox] (box){%
    \begin{minipage}{0.4\textwidth}
      $\tg110^\circ\cdot\tg200^\circ = \tg(90^\circ+20^\circ)\cdot\tg(180^\circ+20^\circ) = -\ctg20^\circ\cdot\tg20^\circ = -1$. Równanie zatem po przekształceniach wygląda tak:\\
      $\cos 2x = \frac12 \Leftrightarrow 2x = \frac\pi6+2k\pi \lor 2x = -\frac\pi6 + 2k\pi, k\in\mathbb{Z} \Leftrightarrow \mathb{x = \left\{\pm\frac\pi3+k\pi, k\in\mathbb{Z}\right\}}$.
    \end{minipage}
};
\node[fancytitle] at (box.north west) {Zadanie 7};
\end{tikzpicture}

\begin{tikzpicture}
\node [mybox] (box){%
    \begin{minipage}{0.4\textwidth}
      $\sin\alpha = \sin\frac12\cdot\frac\pi6 = \sqrt{\frac{1-\cos\frac\pi6}2} = \sqrt{\frac{1-\frac{\sqrt3}2}2} = \frac{\sqrt{2-\sqrt3}}2$.\\
      $\sin2\alpha = \sin\frac\pi6 = \frac12$.\\
      $\sin3\alpha = -4\sin^3x + 3\sin x = -4\cdot\frac18\left((2-\sqrt3)\sqrt{2-\sqrt3}\right) + \frac32\sqrt{2-\sqrt3} = \sqrt{2-\sqrt3}\left(-\frac12(2-\sqrt3) + \frac32\right) = \sqrt{2-\sqrt3}\left(-1+\frac{\sqrt3}2 + \frac32\right) = \sqrt{2-\sqrt3}\left(\frac12 + \frac{\sqrt3}2\right)$.\\
      $\sin\alpha + \sin2\alpha + \sin3\alpha = \frac12 + \sqrt{2-\sqrt3}\left(1 + \frac{\sqrt3}2\right)$.\\
      \\
      $\cos\alpha = \sqrt{1-\sin^2\alpha} = \sqrt{1-\frac{2-\sqrt3}4} = \frac{\sqrt{4-2+\sqrt3}}2 = \frac{\sqrt{2+\sqrt3}}2$.\\
      $2\cos\alpha + 1 = {\sqrt{2+\sqrt3}} + 1$.\\
      \\
      $\frac{\sin\alpha + \sin 2\alpha + \sin3\alpha}{2\cos\alpha + 1} = \frac{\frac12 + \sqrt{2-\sqrt3}\left(1 + \frac{\sqrt3}2\right)}{{\sqrt{2+\sqrt3}} + 1} = \frac{(\sqrt{2+\sqrt3}-1)\frac12 + (\sqrt{2+\sqrt3}-1)\sqrt{2-\sqrt3}\left(1 + \frac{\sqrt3}2\right)}{1+\sqrt3} = \frac{\frac{\sqrt{2+\sqrt3}}2 - \frac12 + (2-\sqrt3)(1+\frac{\sqrt3}2) - \sqrt{2-\sqrt3}(1+\frac{\sqrt3}2)}{1+\sqrt3} = \frac{\frac{\sqrt{2+\sqrt3}}2 - \frac12 + 2 + \sqrt 3 - \sqrt 3 - \frac32 - \sqrt{2-\sqrt3} - \frac{\sqrt{2-\sqrt3}\sqrt3}2}{1+\sqrt3} = \frac{\sqrt{2+\sqrt3} - 2\sqrt{2-\sqrt3} - \sqrt{6-3\sqrt3}}{2+2\sqrt3} = 
      \frac{(2-2\sqrt3)\left(\sqrt{2+\sqrt3} - 2\sqrt{2-\sqrt3} - \sqrt{6-3\sqrt3}\right)}{4-4\cdot3} = \frac{(2-2\sqrt3)\left(\sqrt{2+\sqrt3} - 2\sqrt{2-\sqrt3} - \sqrt{6-3\sqrt3}\right)}{-8} = -\frac18(2-2\sqrt3)(\sqrt{2+\sqrt3} - 2\sqrt{2-\sqrt3} - \sqrt3\sqrt{2-\sqrt3}) = -\frac18(2-2\sqrt3)(\sqrt{2+\sqrt3} - \sqrt{2-\sqrt3}(2+\sqrt3))$.
    \end{minipage}
};
\node[fancytitle] at (box.north west) {Zadanie 8 -- PROBLEM};
\end{tikzpicture}

\begin{tikzpicture}
\node [mybox] (box){%
    \begin{minipage}{0.4\textwidth}
      $\sin 2kx = \cos kx, k = 1005$.\\
      $\sin 2kx = 2\sin kx \cos kx = \cos kx \Leftrightarrow \cos kx = 0 \lor \sin kx = \frac12$.\\
      Rozwiązań równości $\cos x = 0$ na przedziale $\langle0,2\pi\rangle$ jest dwa. Gdy cosinus biegnie $k$ razy szybciej, rozwiązań jest $k$ razy więcej, a więc $2010$.\\
      Rozwiązań równości $\sin x = \frac12$ na przedziale $\langle 0, 2\pi\rangle$ jest dwa. Gdy sinus biegnie $k$ razy szybciej, rozwiązań jest $k$ razy więcej, a więc $2010$.\\
      W sumie rozwiązań jest $\mathb{4020}$.
    \end{minipage}
};
\node[fancytitle] at (box.north west) {Zadanie 9};
\end{tikzpicture}

\begin{tikzpicture}
\node [mybox] (box){%
    \begin{minipage}{0.4\textwidth}
      $\frac{1-\tg\alpha}{1+\tg\alpha} = \frac{\cos\alpha - \sin\alpha}{\cos\alpha + \sin\alpha}$.\\
      $(\cos^2\alpha-\sin^2\alpha)(\cos\alpha + \sin\alpha) = (1+2\sin\alpha\cos\alpha)(\cos\alpha-\sin\alpha)\\
      (\cos\alpha-\sin\alpha)(\cos\alpha+\sin\alpha)^2 = (1+\sin2\alpha)(\cos\alpha-\sin\alpha)$\\
      Nie zawsze $\cos\alpha = \sin\alpha$, zatem możemy podzielić przez ten nawias.\\
      $(\cos\alpha+\sin\alpha)^2 = \sin^2\alpha + \cos^2\alpha + 2\sin\alpha\cos\alpha = 1 + 2\sin\alpha\cos\alpha$.
    \end{minipage}
};
\node[fancytitle] at (box.north west) {Zadanie 10};
\end{tikzpicture}

\begin{tikzpicture}
\node [mybox] (box){%
    \begin{minipage}{0.4\textwidth}
      $\sin10^\circ\cdot\cos20^\circ\cdot\cos40^\circ = \frac18$\\
      $\cos20^\circ\cdot\cos40^\circ = \cos(30^\circ-10^\circ)\cdot\cos(30^\circ+10^\circ) = (\cos30^\circ\cos10^\circ + \sin30^\circ\sin10^\circ) (\cos30^\circ\cos10^\circ - \sin30^\circ\sin10^\circ) = (\frac{\sqrt3}2\cos10^\circ + \frac{\sin10^\circ}2) (\frac{\sqrt3}2\cos10^\circ - \frac{\sin10^\circ}2) = \frac34\cos^210^\circ - \frac14\sin^210^\circ$.\\
      $8\sin10^\circ(\frac34\cos^210^\circ - \frac14\sin^210^\circ) = 1\\
      2\sin10^\circ(3\cos^210^\circ-\sin^210^\circ) = 1\\
      2\sin10^\circ(3(1-\sin^210^\circ) - \sin^210^\circ) = 1\\
      2\sin10^\circ(3-3\sin^210^\circ - \sin^210^\circ) = 1\\
      2\sin10^\circ(3-4\sin^210^\circ) = 1\\
      -4\sin^310^\circ + 3\sin10^\circ = \frac12\\
      \sin30^\circ = \frac12\\
      0 = 0$.
    \end{minipage}
};
\node[fancytitle] at (box.north west) {Zadanie 11};
\end{tikzpicture}

\end{multicols*}

\end{document}