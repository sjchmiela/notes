\documentclass [a4paper, 12pt, oneside]{article}
\usepackage{fullpage}
\usepackage[utf8]{inputenc}
\usepackage{polski}
\usepackage{hyperref}
\usepackage[usenames,dvipsnames]{color}
\usepackage[pdftex]{graphicx}
\usepackage{wrapfig}
\usepackage{float}
\usepackage{amsmath}
\usepackage{amsfonts}
\linespread{1.3}
\hypersetup{
    bookmarks=true,
    unicode=false,
    pdftoolbar=true,
    pdfmenubar=true,
    pdffitwindow=false,
    pdfstartview={FitH},
    pdftitle={Różności o granicach},
    pdfauthor={Stanisław Chmiela},
    pdfsubject={Różności o granicach},
    pdfcreator={Stanisław Chmiela},
    pdfproducer={Stanisław Chmiela},
    pdfkeywords={granice} {matematyka},
    pdfnewwindow=false,
    colorlinks=true,
    linkcolor=BrickRed,
    citecolor=PineGreen,
    filecolor=RawSienna,
    urlcolor=MidnightBlue
}
\author{Stanisław Chmiela}
\title{Granice}
\begin{document}
\section{Definicja granicy}
\[
    \lim_{n\to\infty}~a_n = g\in \mathbb{R}~\Longleftrightarrow~\forall_{\varepsilon > 0}~\exists_{N_0 \in \mathbb{R}}~\forall_{n > N_0}:~|a_n - g| < \varepsilon
\]
\section{Definicja granicy w nieskończoności}
\[
    \lim_{n\to\infty}~a_n = +\infty~\Longleftrightarrow~\forall_{M \in \mathbb{R}}~\exists_{N_0 \in \mathbb{R}}~\forall_{n > N_0}:~a_n > M
\]
Przykłady ciągów bez granic:
\begin{itemize}
    \item $a_n = \sin n$
    \item $b_n = (-1)^n$
    \item $c_n = (-1)^n \cdot n$
    \item $d_n = (-1)^n \cdot \frac{3n}{2n-1}$
\end{itemize}
\section{Dowód, że granica ciągu nie jest w danym miejscu}
\[
    \lim_{n\to\infty}~\frac{5n^2 +7n -1}{3n^2 +3n +4} \neq 1 ~\Longleftrightarrow~\exists_{\varepsilon > 0}~\forall_{N_0\in\mathbb{R}}~\exists_{n\ge N_0}:~ \textcolor{Maroon}{\left|\frac{5n^2 +7n -1}{3n^2 +3n +4}-1\right| \ge \varepsilon}
\]
Ustalone $\varepsilon > 0$. Z \textcolor{Maroon}{równania} dostajemy kolejno:
\[
   \textcolor{Maroon}{ \left| \frac{5n^2 -3n^2 +7n +3n -1 -4}{3n^2 -3n +4}\right| < \varepsilon}
\]
\[
    \left| \frac{2n^2 +10n -5}{3n^2 -3n +4}\right| < \varepsilon
\]
Możemy ściągnąć moduł i pomnożyć, ponieważ wyrażenie jest dodatnie dla $n\ge 1$.
\[
    (3\varepsilon-2)n^2 +(3\varepsilon-4)n +(4\varepsilon +3) > 0
\]
Dla $\varepsilon = \frac12$ nierówność jest spełniona dla skończonej liczby elementów, a więc nie tu jest granica.
\section{Dowód, że ciąg nie może mieć dwóch granic}
Hip. $g_1 \neq g_2$ są granicami ciągu $(a_n)$.
Niech $\varepsilon = \left|\frac{g_1-g_2}{3}\right|$, $g_2 > g_1$.
Z definicji granicy:
\[
    \textcolor{Violet}{\exists_{N_0\in\mathbb{R}}~\forall_{n>N_0}:~\left|a_n - g_1\right| < \varepsilon}
\]
Zarazem:
\[
    \textcolor{WildStrawberry}{\exists_{N_1\in\mathbb{R}}~\forall_{n>N_1}:~\left|a_n - g_2\right| < \varepsilon}
\]
A więc:
\[
    \textcolor{Violet}{\forall_{n>N_0}:~a_n < g_1+\varepsilon}
\]
\[
    \textcolor{WildStrawberry}{\forall_{n>N_1}:~a_n > g_2-\varepsilon}
\]
Ponadto $\textcolor{Green}{g_1 + \varepsilon < g_2 - \varepsilon}$.
Dostajemy:
\[
    \forall_{n > \max\{N_0, N_1\}}:~\textcolor{Violet}{a_n < g_1 + \varepsilon}~\textcolor{Green}{<}~\textcolor{WildStrawberry}{g_2 + \varepsilon < a_n}
\]
\begin{flushright}Sprzeczność.\end{flushright}
\section{Granica sumy dwóch ciągów}
\textbf{Założenie:}
\[
    \lim_{n\to\infty} a_n = a\qquad\lim_{n\to\infty} b_n = b
\]
\textbf{Teza:}
\[
    \lim_{n\to\infty} (a_n+b_n) = a+b
\]
\textbf{Dowód:}
\[
    \lim_{n\to\infty} a_n = a~\Longleftrightarrow~\forall_{\varepsilon > 0}~\exists_{N_0\in\mathbb{R}}~\forall_{n>N_0}:~\textcolor{NavyBlue}{\left|a_n -a\right| < \frac\varepsilon2}
\]
\[
    \lim_{n\to\infty} b_n = b~\Longleftrightarrow~\forall_{\varepsilon > 0}~\exists_{N_1\in\mathbb{R}}~\forall_{n>N_1}:~\textcolor{Mulberry}{\left|b_n -b\right| < \frac\varepsilon2}
\]
Z tezy otrzymujemy:
\[
    \lim_{n\to\infty} (a_n+b_n) = a+b~\Longleftrightarrow~\forall_{\varepsilon > 0}~\exists_{N_2\in\mathbb{R}}~\forall_{n>N_2}:~\textcolor{RawSienna}{\left|(a_n+b_n) -(a+b)\right| < \varepsilon}
\]
\[
    \textcolor{RawSienna}{\left|a_n + b_n - a - b\right|} = \left| a_n-a + b_n-b\right|
    ~\overset{\textrm{z tw. o sumie modułów}}{\le}~
    \textcolor{NavyBlue}{\left|a_n-a\right|} + \textcolor{Mulberry}{\left|b_n-b\right|} < \textcolor{NavyBlue}{\frac\varepsilon2}+\textcolor{Mulberry}{\frac\varepsilon2} = \varepsilon
\]
\begin{flushright}
dla $n > \max\{N_0,N_1\}$
\end{flushright}
A więc:
\[
    \forall_{\varepsilon > 0}~\exists_{N_2 = \max\{N_0, N_1\}}~\forall_{n>N_2}:~ \left|a_n + b_n - (a+b)\right| < \varepsilon
\]
\section{Granica iloczynu dwóch ciągów}
\textbf{Założenie:}
\[
    \lim_{n\to\infty} a_n = a\qquad\lim_{n\to\infty} b_n = b
\]
\textbf{Teza:}
\[
    \lim_{n\to\infty} a_nb_n = ab
\]
\textbf{Dowód:}
\[
    \lim_{n\to\infty}~a_n = g\in \mathbb{R}~\Longleftrightarrow~\forall_{\varepsilon > 0}~\exists_{N_0 \in \mathbb{R}}~\forall_{n > N_0}:~\textcolor{OliveGreen}{|a_n - g| < \varepsilon}
\]
\[
    \lim_{n\to\infty}~b_n = g\in \mathbb{R}~\Longleftrightarrow~\forall_{\varepsilon > 0}~\exists_{N_1 \in \mathbb{R}}~\forall_{n > N_1}:~\textcolor{OliveGreen}{|b_n - g| < \varepsilon}
\]
Rozpiszmy sobie tezę:
\begin{align*}
    \left|a_nb_n - ab\right| &= \\
    &= \left|a_nb_n -a_nb +a_nb -ab\right|\\
    &= \left|a_n(b_n-b) +b(a_n-a)\right|\\
    &\overset{\textrm{z tw. o sumie modułów}}{\le} |a_n|\textcolor{OliveGreen}{|b_n-b|} + |b|\textcolor{OliveGreen}{|a_n-a|}\\
    &\textcolor{OliveGreen}{<} |a_n|\textcolor{OliveGreen}{\varepsilon} + |b|\textcolor{OliveGreen}{\varepsilon} \\
    &< \max\{|a-\varepsilon|, |a+\varepsilon|\}\cdot \varepsilon + |b|\cdot \varepsilon \\
    &= \varepsilon\underbrace{(\max\{|a-\varepsilon|, |a+\varepsilon|\} + |b|)}_{> 0}
\end{align*}
A zatem dla $\varepsilon > 0~\exists_{N_2 = \max\{N_0, N_1\}}:~\left|a_nb_n - ab\right| < \varepsilon\cdot c$, gdzie $c>0$, dla $n > N_2$.
\section{Ciąg ograniczony}
\textbf{Definicja:}
\[
    (a_n)\textrm{ jest ograniczony}\Longleftrightarrow \exists_{M\in\mathbb{R}}~\forall_{n\in\mathbb{N}_+}:~|a_n| < M
\]
\textbf{Założenie:}
\[
    \lim_{n\to\infty} a_n = a
\]
\textbf{Teza:}
\[
    (a_n)\textrm{ jest ograniczony}
\]
\textbf{Dowód:}
Niech $\varepsilon = 1$. Wówczas $\exists_{N_0 \in \mathbb{R}}~\forall_{n>N_0}:~\left|a_n-a\right| < 1$. A co za tym idzie:
\[
    a-1 < a_n < a+1
\]
Dalej:
\[
    \forall_{n\in\mathbb{N}_+}:~\underbrace{\min\{a_1, a_2,\dots,a_{[N_0]}, a-1\}}_{= m_1} \le a_n \le \underbrace{\max\{a_1, a_2, \dots, a_{[N_0]}, a+1\}}_{= m_2}
\]
A więc wtedy $M = \max\{|m_1|, |m_2|\}$.
\section{Granica ciągu w danej liczbie}
Mamy ciąg $a_n = \frac1{2n-17}$, chcemy udowodnić, że jego granicą jest liczba $a = 0$.\\
\textbf{Założenie:}
\[
	a_n = \frac{1}{2n-17}
\]
\textbf{Teza:}
\[
    \lim_{n\to\infty}~\frac{1}{2n-17} = 0~\Longleftrightarrow~\forall_{\varepsilon > 0}~\exists_{N_0 \in \mathbb{R}}~\forall_{n > N_0}:~\left|\frac{1}{2n-17} - 0\right| < \varepsilon
\]
\textbf{Dowód:}
Dostajemy ustalony $\varepsilon > 0$. Rozpiszmy sobie z definicji granicy.
\[
\left|\frac{1}{2n-17} - 0\right| < \varepsilon
\]
\textcolor{OliveGreen}{Dla $n\ge9$ możemy zdjąć moduł, więc zróbmy to.}
\begin{align*}
    \frac1{2n-17} & < \varepsilon\\
    1 &< \varepsilon(2n-17)\\
    1 &< 2\varepsilon n - 17\varepsilon\\
    \textcolor{WildStrawberry}{n} & > \textcolor{WildStrawberry}{ \frac{1+17\varepsilon}{2\varepsilon}}
\end{align*}
A zatem do ustalonego $\varepsilon > 0$ można dobrać $N_0 = \max\{\textcolor{WildStrawberry}{\frac{1+17\varepsilon}{2\varepsilon}}, \textcolor{OliveGreen}{9}\}$ takie, że $\forall_{n > N_0} \left|\frac{1}{2n-17} - 0\right| < \varepsilon$.
\section{Twierdzenie o trzech ciągach} % (fold)
\label{sec:twierdzenie_o_trzech_ci_gach}
    \textbf{Założenie:} Mamy trzy ciągi $(a_n)$, $(b_n)$, $(c_n)$, dla których:
    \[
        \textcolor{BlueViolet}{\exists_{N_0\in\mathbb{R}}\forall_{n>N_0}: a_n \le b_n \le c_n}
    \]
    \[
        \lim_{n\to\infty} a_n = \lim_{n\to\infty} c_n = g \in \mathbb{R} \cup \{\pm\infty\}
    \]
    \\
    \textbf{Teza:}
    \[
        \lim_{n\to\infty} b_n = g
    \]
    \\
    \textbf{Dowód:}
    Rozpiszmy sobie tezę.
    \[
        \lim_{n\to\infty}~a_n = g~\Longleftrightarrow~\forall_{\varepsilon > 0}~\exists_{N_1 \in \mathbb{R}}~\forall_{n>N_1}:~\textcolor{BrickRed}{\left|a_n-g\right| < \varepsilon}
    \]
    \[
        \lim_{n\to\infty}~c_n = g~\Longleftrightarrow~\forall_{\varepsilon > 0}~\exists_{N_2 \in \mathbb{R}}~\forall_{n>N_2}:~\textcolor{Plum}{\left|c_n-g\right| < \varepsilon}
    \]
    Przyjrzyjmy się pokolorowanym elementom i rozpiszmy je.
    \[
        \textcolor{BrickRed}{\left|a_n-g\right| < \varepsilon}
    \]
    \[
        {-\varepsilon < a_n-g < \varepsilon}
    \]
    \[
        \textcolor{Bittersweet}{g-\varepsilon < a_n} < g+\varepsilon
    \]
    Analogicznie dla $(c_n)$:
    \[
        g-\varepsilon < \textcolor{Fuchsia}{c_n < g+\varepsilon}
    \]
    Połączmy założenie z tym, co wyliczyliśmy.
    \[
        \textcolor{Bittersweet}{g-\varepsilon < a_n}~\textcolor{BlueViolet}{\le b_n \le}~\textcolor{Fuchsia}{c_n < g+\varepsilon}
    \]
    Z tego wnioskujemy, że:
    \[
        \forall_{\varepsilon > 0}~\exists_{N_3 = \max\{N_0,N_1,N_2\}}~\forall_{n>N_3}:~\left|b_n-g\right| < \varepsilon
    \]
    Co było do udowodnienia. Do $\varepsilon > 0$ dobraliśmy $N_3 = \max\{N_1,N_2\}$ takie, że $g-\varepsilon < b_n < g+\varepsilon$.
% section twierdzenie_o_trzech_ci_gach (end)
\end{document}
